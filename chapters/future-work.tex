
\chapter{Future Work and Conclusion}
\label{cha:future-work-concl}

This chapter concludes the thesis by laying out directions for future
work. The author has made some progress on some of these directions
while for others, pointers are given to related work so that the
reader can get started.

\section{A Unified Framework for Planning and Execution using
  Inaccurate Models}
\label{sec:unified-framework}

This thesis has presented two novel algorithms \cmax{} and \cmaxpp{} that update the
behavior of the planner, rather than updating the dynamics of the
model, to allow robots to complete the task despite using an
inaccurate model. \cmax{} enables the planner to stick to the
state-action space regions where the model is accurate and biases it
away from the inaccurately modeled regions. \cmaxpp{}, on the other
hand, learns model-free value estimates for inaccurately modeled
transitions and integrates them into a model-based planning procedure
with the inaccurate model. Both approaches require the inaccurate
model to be optimistic and have task-completeness guarantees. While
our experiments have shown that they work very well empirically, there
are domains where designing ``good'' optimistic models is
difficult. By good, we mean a non-trivial optimistic model (a trivial
optimistic model would be one that predicts any transition executed by
robot would complete the task) that is useful in most state-action
space regions. In such domains, updating the dynamics of the model
might be more efficient even in cases where the true dynamics does not
lie in the model class considered. This thesis has taken preliminary
steps in designing such an efficient model learning algorithm in
Chapter~\ref{CHA:TAML} where we presented \taml{} that performs better
than \cmax{} in domains where we can update the dynamics of the model
through low-dimensional parameterizations.

An important future direction would be to build upon \taml{} to design
efficient model learning algorithms that directly optimize task
performance and enable choosing models that are useful for planning
rather than prediction. Initial work in this direction has been done
in~\cite{grimm2020value, DBLP:journals/corr/abs-2106-10316,
  DBLP:conf/icml/AyoubJSWY20, DBLP:journals/corr/abs-2106-14080,
  DBLP:journals/corr/abs-2106-03273,
  DBLP:journals/corr/abs-2110-02758}.
Given such algorithms, the author envisons a
unified framework for planning and execution where the robot, at every
time step, chooses to either update the dynamics of the model from
executions (using algorithms like \taml{}) or updates the behavior of
the planner (using algorithms like \cmax{} and \cmaxpp{}.) This allows
us to combine the advantages of both family of algorithms while
retaining task completeness guarantees. One viable way of implementing
this would be by using the multi-heuristic A* (MHA*)
framework~\cite{DBLP:journals/ijrr/AineSNHL16} where we treat each
algorithm that the robot can use as a heuristic that it can follow to
reach the goal. This would involve maintaining a different set of
cost-to-go estimates for \taml{}, \cmax{} and \cmaxpp{}. Intuitively,
we prefer \cmax{} as it does not waste executions learning dynamics or
learning model-free value estimates and quickly finds an alternative
path. To encode this preference, we can design an anytime algorithm
similar to \acmaxpp{} (in Section~\ref{sec:adaptive}) where if the
cost-to-go following \cmax{} is not too worse compared to that of
\taml{} and \cmaxpp{}, we follow \cmax{}. Else, if the cost-to-go
following \cmaxpp{} is not too far from that of \taml{}, then we
follow \cmaxpp{}. If neither of those are true, then we follow
\taml{}. This encodes the preference that avoiding inaccurately
modeled transitions is easier to learn than model-free value
estimates which is easier to learn than the model dynamics. The goal is to create a unified
framework where the robot, during the course of its execution,
intelligently switches between (a) learning the true dynamics, (b)
learning a model-free value estimate, or (c)
biasing the planner away from an inaccurately modeled
transition to guarantee task completeness while reducing
the amount of real-world experience required.


\section{Online Model Learning with Misspecified Model Classes}
\label{sec:online-model-learn}

While \taml{} was a first step in the direction of online model
learning with misspecified model classes where we directly optimize
task performance rather than prediction error, the author believes
there is still a long way to go in this direction. Our main motivation
for this comes from the simulation lemma, which was first introduced
in~\cite{DBLP:journals/ml/KearnsS02}, and can be reformulated in the
undiscounted deterministic dynamics setting as follows:
\begin{lemma}[Undiscounted Deterministic Dynamics Simulation Lemma]
  Let $M, M'$ be two Markov Decision Processes with the same cost
  function. If we have a fixed 
  start state $s_0$, a deterministic policy $\pi:\statespace
  \rightarrow \actionspace$, and $M, M'$ have deterministic dynamics
  $f, f': \statespace \times \actionspace \rightarrow
  \statespace$. Then we have,
  \begin{align}
    \label{eq:23}
    J_M(\pi) &= J_{M'}(\pi) + \sum_{t=0}^\infty c(s_t^M, \pi(s_t^M)) +
               V_{M'}^\pi(s_{t+1}^M) - V_{M'}^\pi(s_t^M) \\
    &= J_{M'}(\pi) + \sum_{t=0}^\infty V_{M'}^\pi(s_{t+1}^M) -
      V^\pi_{M'}(f'(s_t^M, \pi(s_t^M)))
  \end{align}
  where $s_0^M = s_0$ and $s_t^M = f(s_{t-1}^M, \pi(s_{t-1}^M))$.
\end{lemma}

In the case where $M$ is the real world, and $M'$ is any dynamical
model that we consider, the above lemma states that the performance of
any policy $\pi$ in the real world $M$ is equal to the sum of the performance of the
policy in the model $M'$ and the \textit{model advantages} at each time step $V_{M'}^\pi(s_{t+1}^M) -
      V^\pi_{M'}(f'(s_t^M, \pi(s_t^M)))$. Thus, in order to find a
      model $M'$ that captures the performance of a policy as the same
      as that of its performance in the real world, we need to
      minimize model advantages. However, most existing works that
      perform maximum likelihood learning do not consider this
      objective function~\cite{DBLP:journals/arc/Ljung10,
        DBLP:conf/icml/AbbeelN05, DBLP:conf/icml/RossB12, 
  DBLP:journals/corr/abs-1907-02057} and instead use a prediction
error loss. To understand why, let us take a
closer look at the model advantages:
\begin{align*}
  V_{M'}^\pi(s_{t+1}^M) - V^\pi_{M'}(f'(s_t^M, \pi(s_t^M))) &\leq
                                                              L\|s_{t+1}^M - f'(s_t^M, \pi(s_t^M))\| \\
  &\leq L\|f(s_t^M, \pi(s_t^M)) - f'(s_t^M, \pi(s_t^M))\|
\end{align*}
where we assumed that the value function of policy $\pi$ in the model
$M'$ is $L$-lipschitz (any bounded function on a bounded domain is
lipschitz.) Thus, instead of optimizing the model advantages one can
optimize the prediction error which is an upper bound on the
model advantage~\cite{DBLP:conf/icml/RossB12}.

%%% Local Variables:
%%% mode: latex
%%% TeX-master: "../main"
%%% End:
