
\chapter{Task-Aware Model Learning}
\label{CHA:TAML}

\epigraph{\textit{More work is needed before planning with learned
    models can be effective. Environment models should be
    constructed judiciously with regard to both their states and
    dynamics with the goal of optimizing the planning process.}}{Rich
  Sutton and Andrew Barto (2018)}

The algorithms presented in this thesis, so
far, have not required any updates to the dynamics of the model. In
contrast, most existing methods in the literature, such
as~\cite{DBLP:journals/ml/KearnsS02, DBLP:journals/jmlr/BrafmanT02,
  DBLP:conf/atal/JongS07,
  DBLP:journals/pami/DeisenrothFR15, DBLP:conf/icml/AbbeelQN06,
  DBLP:conf/aaai/Jiang18, rastogi2018sample}, use experience
acquired from executions to update the dynamics of the model or learn
a model from scratch.
Chapters~\ref{CHA:CMAX} and~\ref{CHA:CMAXPP} have
argued that updating the dynamics of the model requires a large amount
of experience in large state spaces and can be at the expense of
completing the task. While this is generally true, there are major
advantages of updating the dynamics of the model, especially in
domains where it
is feasible to do it online, as it allows the planner to compute
solutions that exploit the true dynamics and potentially result in
solutions with very low costs. Furthermore even in application domains
where we require a large amount of experience to update the model,
the improvement in task performance from planning on a more accurate
model can outweigh the executions wasted to learn true dynamics. For
example, there might be regions in the state space where updating the
dynamics of the model can be done efficiently while in other regions
we can resort to methods that update the behavior of planner such as
\cmax{} and \cmaxpp{}. This motivates a trade-off between both sets of
approaches and understanding this trade-off can result in intelligent
use of online experience to achieve efficient planning and
execution.

%%% Local Variables:
%%% mode: latex
%%% TeX-master: "../main"
%%% End:
