\PassOptionsToPackage{svgnames,dvipsnames}{xcolor}

\documentclass[11pt]{cmuthesis}

\usepackage[Lenny]{fncychap}
\ChNameVar{\Large}

\input{packages}
\input{macros}

%\draftstamp{\today}{DRAFT}

\begin{document}
\frontmatter

\pagestyle{empty}

\title{{\bf Planning and Execution using Inaccurate Models with
    Provable Guarantees on Task Completeness}}
\author{Anirudh Vemula}
\date{February 2 2022}
\Year{2022}
\trnumber{CMU-RI-TR-22-05}

\committee{
  Maxim Likhachev, Co-Chair \\
  J. Andrew Bagnell, Co-Chair \\
  Oliver Kroemer \\
  Leslie Pack Kaelbling, \textit{Massachusetts Institute of Technology}
%   \begin{tabular}{cc}
%     & \\
% Maxim Likhachev, Co-Chair & CMU\\
% J. Andrew Bagnell, Co-Chair & CMU\\
%     Oliver Kroemer & CMU \\
%     Leslie Pack Kaelbling & MIT
% \end{tabular}
}

\support{}
\disclaimer{}

\keywords{Robotics, Planning, Reinforcement Learning, Numerical Optimization}

\maketitle

\begin{dedication}
  % For my brother, who put the fear of mediocrity in me \\
  For my mother, who sacrificed her own dreams to help me pursue
  mine
  % For my love, who believed in me and brought me back to surface
  % when I sank to the depths
\end{dedication}

\begin{abstract}
  Modern planning methods are effective in computing feasible and
optimal plans for robotic tasks when given access to accurate
dynamical models.
However, robots operating in the real
world often face situations that cannot be modeled perfectly before
execution.
Thus, we only have access to simplified but potentially inaccurate 
models.
This imperfect modeling can lead to highly suboptimal plans
or even the inability to reach the goal during execution.
Existing approaches present a learning-based solution where real-world
experience is used to learn a complex dynamical model that is
subsequently used for planning. However, this requires a prohibitively
large amount of experience over the entire state space, and can
be wasteful if we are interested in completing the task and not in
modeling the dynamics accurately. Furthermore, real robots often have
operating constraints and cannot spend hours acquiring experience to
learn dynamics.
This thesis argues that by updating the behavior of the planner and
not the dynamics of the model, we can
leverage simplified and potentially inaccurate models and
significantly reduce the amount of real-world experience needed to
provably guarantee that the robot completes the task.

We support this argument from an algorithmic perspective
by presenting two novel algorithms.
The first algorithm \cmax{} guarantees that the robot reaches the
goal using the inaccurate model without any resets. This
is achieved by biasing the planner away from transitions whose
dynamics are discovered to be inaccurately modeled during online
execution. However, \cmax{} requires strong assumptions on the
accuracy of the model used for planning and fails to improve the
quality of solution over repetitions of the same task. The second
algorithm \cmaxpp{} leverages real-world experience to improve the
quality of resulting plans over successive repetitions of a robotic
task. \cmaxpp{} achieves this by integrating model-free learning using
acquired experience with model-based planning using the potentially
inaccurate model. As a consequence of this in addition to completeness, \cmaxpp{} also
guarantees asymptotic convergence to the optimal path cost
as the number of repetitions increases under relaxed
assumptions. Crucially, both algorithms do not require any updates to
the dynamics of the model unlike any existing method for planning
using inaccurate models.

From a theoretical perspective,
this thesis presents a
performance analysis for methods
that leverage inaccurate models in optimal control of linearized systems with
quadratic costs. Our analysis shows that naively using inaccurate
models can lead to large suboptimality gaps when modeling errors are
significant, while updating the behavior of the planner during
execution, like \cmax{} and \cmaxpp{}, can substantially reduce the
suboptimality gap. The thesis concludes by exploring the paradigm of
updating the dynamics of the model and presents an algorithm \taml{} that
directly optimizes task performance rather than prediction error. We
show that in the online setting where the 
robot does not have access to any resets and collects data as it
executes, \taml{} outperforms prior 
works that either optimize a maximum likelihood objective or rely on
an offline collected dataset with good coverage.

\end{abstract}

\newgeometry{left=0.5in,right=0.5in,top=1in,bottom=1.4in}
\begin{acknowledgments}
  I believe PhD is all about the journey and not about the destination. I was
  extremely fortunate to have a
  wonderful journey
  with amazing people who have been my collaborators, confidantes,
  friends, mentors, and family. I will try my best to thank each and every one of them
  but alas, my memory often fails me. So if I forget someone below, know that I am
  lucky to have known them and they made my journey all the more fun.
  % While I leave with a doctorate in my hand, I cherish my
  % interactions with them the most and will always look back fondly on these
  % memories.

  % Advisors - Max, Drew
  First and foremost, I would like to thank my advisors Max and Drew. I always
  wanted to work with both of them, and forging this joint
  collaboration remains as the most important decision I have made in
  my PhD. They were the yin and yang in my PhD, and the body of work
  you read in this thesis was a direct result of this
  delicate balance. Max always ensured that my ideas were grounded and
  would work on a robotic system, which was especially useful as I often
  flew away with idealistic ideas. I admire his extreme patience
  during our meetings and his approach of understanding complex ideas
  from fundamentals. If not these amazing skills, I hope I learned how
  to always stay humble and listen to collaborators from him. Drew has been a
  constant source of inspiration throughout my journey and allowed me
  to chase crazy ideas purely because they were interesting. For someone
  who is a CTO of a self-driving company, his passion for pure fundamental
  research is very inspiring and has driven me to work harder on ideas
  that I was excited about. I loved all our phone calls, with me
  speaking cryptic equations while he somehow miraculously deciphers
  them, our walks on campus discussing ideas and papers, and our terse
  emails with the entire content in the subject and no body. The most
  fun projects in my PhD were a result of brainstorming with Drew, and
  for that I am extremely grateful.

  % Committee - Oliver, Leslie
  I take this opportunity to also thank my committee members,
  Oliver and Leslie. I interacted with Oliver a lot of times during my
  PhD both as an RI student who was interested in his work, and as his
  teaching assistant for two courses. I admire his straightforward way
  of thinking and insights in to the robot manipulation problem, which
  influenced my own research. Leslie has always been one of my idols
  and for someone entering into the field of robot planning and
  learning, some of her works were foundational. I am extremely lucky
  to have them both on my committee and thankful for all the feedback
  I have received from them on my thesis.

  % Apple - Martin, Chris, Humphrey, Jakub
  My journey also took me to some wonderful places besides CMU. I was
  fortunate to work with Martin Levihn, Chris Clark, Humphrey Hu, and
  Jakub Dworakowski at Apple Special Projects Group. All of them have
  been incredible mentors both professionally and personally. Their
  insights and guidance has influenced both this thesis and my future
  beyond PhD. I also worked with Vladlen Koltun and Ozan Sener at
  Intel Labs, where I learned how to tackle extremely difficult
  problems and the value of collaboration. I'd like to thank Ozan for
  his patience while I slowly grasped the
  challenges of the problem.

  % Masters Advisors - Jean, Katharina
  I would be remiss if I did not thank Jean Oh and Katharina Muelling
  who were my mentors at the start of this journey. They showed me how
  exciting research could be, and encouraged me to apply for a
  PhD. Jean is one of the nicest people I know, and I am
  fortunate to have been her student.

  % SBPL - Dhruv, Suhail, Fahad, Sung, Andrew, Tushar, Yash, Venkat, Jacky,
  % Anahita, Shivam, Ram, Oren, Vinitha
  During my time at CMU, I was a part of two wonderful labs - SBPL and
  Lairlab. I would like to thank my officemates, Dhruv Saxena, Yash Oza, and
  Tushar Kusnur for putting up with my constant scribbles on the whiteboard
  and my loud discussions. I would like to thank Dhruv especially for
  always lending an ear when I needed it. For the first 3 years of my
  PhD, I was part of a truck unloading project and worked with fellow
  SBPL members - Oren
  Salzman, Fahad Islam,
  Sung-Kyun Kim, and Andrew Dornbush, who were all insanely helpful whenever I got stuck
  and helped me learn motion planning and belief space planning
  fundamentals. Besides work, I had loads of fun with other SBPL
  friends - Venkatraman Narayanan, Jacky Liang, Anahita Mohseni Kabir,
  Shivam Vats, Ram Natarajan, Manash Pratim Das, Vinitha Ranganeni,
  Shohin Mukherjee, and 
  Rishi Veerapaneni. While SBPL was where I physically most of my time, Lairlab
  was my go to for brainstorming on machine learning research. Wen Sun
  was my close collaborator initially in my PhD, and he almost became
  my third advisor for a year. Fellow ``Bagnellians'' -  Allie Del
  Giorno, Hanzhang Hu, Jiaji 
  Zhou, Wen Sun, Arun Venkatraman, and Gokul Swamy
  were wonderful labmates and I will always cherish the lab lunches
  with them which usually had heated discussions on topics ranging
  from research to the latest gossip. Special shout out to Allie for
  making my defense super fun, and helping me with reviewing all my
  papers and presentations. 

  % Lairlab - Allie, Wen, Gokul, Echo, Jiaji, Arun

  % RI friends - Achal, Xuning, Alex, Adithya, Jerry, Puneet, Vishal, Senthil,
  % Ankit, Rosario, Rogerio, Arjav, Eric, Tabitha, Saumya, Dominik, Cornelia,
  % Yifan, Debi, Nick, Abhijeet, Sanjiban, Ratnesh
  The robotics institute at CMU has a wonderful student community. As
  a part of RoboOrg, and as a RI Masters and PhD student, I had the good
  fortune to know some of the most amazing and talented
  individuals. These friendships lasted over 6 years and will last for
  a lot more years to come. I would like to thank
  Puneet Puri, Vishal Dugar, Jerry Hsiung, Senthil Purushwalkam, Ankit Bhatia,
  Adithya Murali, Alex Spitzer, Xuning Yang, Achal Dave, Debidatta
  Dwibedi, Rogerio
  Bonatti, Zhi Tan, Rosario Scalise, Eric Huang, Arjav Desai, Tabitha Lee,
  Yifan Hou, Brian Okorn, Tanmay Shankar, Roberto Shu, Sibi
  Venkatesan, Nick
  Gisolfi, Abhijeet Tallavajhula, Ratnesh Madaan, Sankalp Arora, and
  Sanjiban Choudhury. A special shout-out to NSH 2204 for being
  legendary, you know you are. I would like to give special thanks to
  Sanjiban who was my first mentor at CMU, and remained a friend
  I always reach out to for guidance on work, life, and beyond.

  % RI admin - Suzanne, Jean
  Suzanne Muth and Jean Harpley made my life at RI feel seamless. Suzanne
  worked hard behind the scenes to ensure that RI PhDs could focus
  solely on research. Jean Harpley is one of the most amazing people
  in RI. She helped RoboOrg and the student community in RI at every
  step, even though it was not a part of her job description. For that
  and many more, I would like to thank both of them.

  % CMU friends - Abilash, Ben, Shefali, Nikita, Brandon, Cara, Ananya
  RI was too small to contain all the wonderful people I know, and I
  met people from other departments who played a big part in my
  journey. Abilash Subbaraman was always down to climb with me and do
  the craziest, and often the most dangerous, adventure sports with
  me. Ben Eysenbach helped me understand how my research fits with the
  broader RL literature. Shefali Umrania and Ananya Uppal were close
  friends who made my grad school life extremely fun. The Explorers
  club of Pittsburgh introduced me to several inspiring and fun
  people with whom I went backpacking, climbing, and biking
  with. All work and no play could have made this journey dull, but
  these folks ensured it was anything but dull.

  % Family - Mom, Dad, Venkat, Payal
  I am the person I am today because of my family. My dad instilled
  the value of working hard in me, and pushed me to always have bigger
  ambitions. My brother is my constant partner-in-crime and was a huge
  inspiration in my career choices. He has always prioritized my
  choices over his own, and baked a fear of mediocrity in me, that
  has always pushed me to go further. My mom is the reason for all my
  successes, and her countless sacrifices gave me the opportunity to
  focus on my education while she took care of far more important
  issues. I hope I say it enough, but if I don't here it goes - I am
  extremely grateful to be your son, brother, and a part of this
  family. 

  % Love - Pragna
  I said PhD is all about the journey. This journey started with me
  meeting my partner Pragna Mannam, and she enriched it far beyond
  what it would have been without her. She always believed in me, when
  I did not. She brought me to the surface, when I sank to the
  depths. She supported me unconditionally, when I needed it. She is
  my rock, and the shining star in my sky. At the end of this journey,
  I am at a loss of words to thank her, and can only say ``Cutie, we
  did it!''
\end{acknowledgments}
\restoregeometry

\pagestyle{plain}

\tableofcontents
\addtocontents{toc}{\vspace*{-2cm}}
\listoffigures
\addtocontents{lof}{\vspace*{-2cm}}
\listoftables
\listofalgorithms

\mainmatter


\chapter{Introduction}
\label{cha:introduction}

\epigraph{\textit{Remember that all models are wrong; the practical
    question is how wrong do they have to be to not be
    useful.}}{George Box (1987)}

\section{Motivation}
\label{sec:motivation}

Robotic planning algorithms have been widely successful in computing
feasible and optimal plans, or sequence of decisions, for tasks
involving robots operating in known environments or under known
conditions~\cite{DBLP:books/cu/L2006}. A large part of this success
can be attributed to principled algorithms that effectively
``search'' the space of all plans by exploiting the known
structure in the form of dynamical models to quickly compute the
solution~\cite{choset2005principles}. For example in the field of robot motion planning, there
have been various developments in designing planning algorithms that
exploit forward models to effectively discretize the state space into
a graph and compute a feasible plan using graph search
techniques~\cite{DBLP:books/daglib/0068760}. This enables planning
algorithms to guarantee task 
completeness, which is a requirement on the solution plan to complete
the task, and be efficient in the amount of computational resources
needed to find the solution~\cite{DBLP:journals/arobots/Hauser12}.

However for robots to operate in unstructured environments such as
homes, offices and disaster sites, planning algorithms have to
reason about how to deal with the lack of complete knowledge of the
environment while ensuring task completeness~\cite{DBLP:journals/ai/KaelblingLC98}. To retain their
effectiveness, these planning algorithms will have to utilize partial
knowledge of the environment and the task in the form of simplified
and \textit{inaccurate} dynamical models~\cite{abbeel2006using}.
Naively using these inaccurate dynamical models for planning
can result in highly suboptimal plans and in some cases, plans that do
not complete the task during execution~\cite{kolter2010learning}. An example of such behavior is
shown in Figure~\ref{fig:intro-example}. In this example, a robotic arm is
performing a pick-and-place task while avoiding collision with an
obstacle. In the first scenario (the first three figures from the left
in Figure~\ref{fig:intro-example},) the arm is interacting with a
light object whose mass is accurately captured by the dynamical model
used by the planner. This results in a computed trajectory for the
arm that grasps the object, lifts it above the obstacle and takes it
to the goal location. While this scenario has highlighted the
effectiveness of the planning algorithm to complete the task when
given access to an accurate dynamical model, consider the second
scenario (the last figure on the right in
Figure~\ref{fig:intro-example}) where the arm is interacting with a
heavy object which is modeled as a light object by the dynamical
model. Since the model is same as before, the planner computes the
same trajectory which lifts the object above the obstacle. However,
while executing the trajectory the arm cannot lift the heavy object and
cannot command the joint torques required because they are beyond the
arm's capabilities. Thus, the computed plan is not successful in
completing the task.
\begin{figure*}[t]
  \centering
  \begin{subfigure}{0.24\linewidth}
    \includegraphics[width=\linewidth]{figures/cmax/pr2_pick_place_light_1_annotated.jpeg}
  \end{subfigure}
  \begin{subfigure}{0.24\linewidth}
    \includegraphics[width=\linewidth]{figures/cmax/pr2_pick_place_light_2_annotated.jpeg}
  \end{subfigure}
  \begin{subfigure}{0.24\linewidth}
    \includegraphics[width=\linewidth]{figures/cmax/pr2_pick_place_light_3_annotated.jpeg}
  \end{subfigure}
  \begin{subfigure}{0.24\linewidth}
    \includegraphics[width=\linewidth]{figures/cmax/pr2_pick_place_heavy_1_annotated.jpeg}
  \end{subfigure}
  \caption{A robotic arm picking an object from its start location and
  placing it at a goal location while avoiding collision with the
  intermediate obstacle during motion. The first three (from left)
  figures show an execution with a light object (wooden block) and a
  plan (blue trajectory) computed
  using an accurate dynamical model which captures the weight of the
  object correctly. The last figure (rightmost) shows an instance of
  the same task but with a heavy object (black dumbbell) and same
  dynamical model as before which models the object as light. This
  results in the planner computing the same plan as before, which the
  robot cannot execute as lifting the heavy object requires joint
  torques that are beyond the robot's capabilities. Thus, the plan is
  not task complete.}
  \label{fig:intro-example}
\end{figure*}
The above example highlights the ineffectiveness of naively using
these inaccurate models for planning. This ineffectiveness can be
tackled broadly in two directions: updating either the dynamical
model or the behavior of the planner, using the accumulated
experience during execution.

\subsection{Updating the Dynamical Model}
\label{sec:updat-dynam-model-1}

The former direction of using online
experience to update existing dynamical models or learning new
dynamical models from scratch has been explored in the Reinforcement
Learning (RL) framework. This framework enables autonomous agents,
such as robots, to learn how to operate in an unknown environment by
interacting with it and compute an optimal plan that minimizes total
cost~\cite{sutton1998introduction}. With partial or no prior knowledge
about the environment, the agent needs to explore to discover low cost
actions or regions where dynamics are inaccurately modeled. The
exploration strategies leveraged by these agents require a large
amount of interactions with the environment before we can compute
plans that guarantee task completeness~\cite{kakade2003sample}. This is a major reason why RL,
despite being a very general framework, has mostly seen success in
domains where we can afford to collect large amounts of interactions
with little effort: video games and
simulations~\cite{DBLP:journals/nature/SilverSSAHGHBLB17,
  DBLP:journals/corr/abs-1912-06680, DBLP:conf/aaai/HesselMHSODHPAS18}.

Most methods in the RL framework can be categorized as either model-based
or model-free (Figure~\ref{fig:dyna}). As the name suggests, model-based methods rely on
using a model as input to a planning procedure to compute the solution
plan for a given task. These methods use the experience gained online
during execution to update the dynamics of the model and replan to
compute a new solution plan~\cite{DBLP:journals/sigart/Sutton91}. In
contrast, model-free methods directly 
use the accumulated experience to compute an updated solution plan
without ever using a dynamical model. These methods utilize the
experience to estimate value functions, which are essentially
cost-to-go estimates, and compute a plan using the estimated
values~\cite{DBLP:journals/ml/WatkinsD92}. Both methods have advantages and disadvantages. Model-based
methods relatively require fewer amounts of experience to compute a
plan of the same quality as the plan computed by a model-free
method~\cite{DBLP:conf/colt/SunJKA019}. On the other hand, model-free methods are not affected by the biases
inherent in the design of the model~\cite{deisenroth2010reducing}.
This thesis will primarily focus on model-based methods as they allow us
to exploit existing domain knowledge in the form of inaccurate models. However,
we also explore integrating model-free methods with model-based planning to
combine the advantages of both approaches.

A primary disadvatange of methods that use the accumulated experience from
execution to update the dynamics of the model is that in tasks with complex
dynamics, learning the true dynamics can require a exhorbitantly large number of
executions~\cite{DBLP:journals/corr/abs-1907-02057,
  DBLP:conf/icra/NagabandiKFL18, DBLP:journals/corr/abs-1911-08265}.
Furthermore, there might be no model in the
model class considered
that can accurately represent the true dynamics~\cite{DBLP:conf/icra/JosephGRHR13}. In such cases, prior works have
shown that finding the model with the lowest prediction error need not guarantee
task success when subsequently used for planning, and could have worse
performance when compared to a model with higher prediction
error~\cite{Farahmand2018, grimm2020value}. Thus, it is non-trivial to
guarantee task completeness for these approaches. Intuitively, there is an
inherent mismatch of objectives where the model is updated to improve its
prediction accuracy, while the planner uses the model to find a plan that completes the task~\cite{DBLP:conf/l4dc/LambertAYC20}.
This motivates the need to judiciously update the dynamical model with the goal
of optimizing the planning process, and this thesis takes some preliminary steps
in this direction by presenting a task-aware model learning approach that directly
optimizes task success.

\begin{figure}[t]
  \centering
  \includegraphics[width=0.5\linewidth]{figures/intro/dyna.pdf}
  \caption{Operation of Model-based (blue) and Model-Free RL (red) methods
    while executing in unknown environments and collecting
    experience to complete a task. Figure inspired from Dyna~\cite{DBLP:journals/sigart/Sutton91}}
  \label{fig:dyna}
\end{figure}

\subsection{Updating the Behavior of the Planner}
\label{sec:updat-behav-plann}


In contrast, the latter direction of updating the behavior of the
planner using online experience has not been explored as extensively
in past literature. Interestingly, this direction has been in use by
the practitioner for quite some time. As motivated earlier, in most
robotic tasks we seldom have access to accurate dynamical models and
the models we use for planning are often inaccurate. Robotics
engineers and practitioners have been dealing with these inaccuracies
by modifying how the planner uses the inaccurate model rather than
updating the model to improve its accuracy. As an example, consider
the task of planning footsteps of a mobile quadruped robot over
partially unknown terrain as shown in Figure~\ref{fig:zucker} taken
from \cite{DBLP:journals/ijrr/ZuckerRSCBAK11}. The
unknown part of the terrain is annotated in the figure (red oval.) To
ensure that the planner does not compute footstep trajectory that goes
through this region, a simple
hack that the practitioner does is to inflate the cost of any
state-action pair that takes the robot into this region. This results
in biasing the planner away from this region thereby updating its
behavior. There are several other works that deal with inaccurate modeling by simply updating the
behavior of the planner~\cite{DBLP:journals/ral/McConachiePMB20,
  DBLP:journals/scirobotics/MitranoMB21, DBLP:journals/ral/PowerB21,
  DBLP:conf/icra/LeeFTRGRF20}.

An observant reader will notice that these approaches require the models used
for planning to not be completely inaccurate everywhere. As a simple example, if
we use a model that predicts that the robot will crash for any action that it
can possibly take, then simply updating the behavior of the planner will not
result in any improvement as the model used is inherently bad.
While these
approaches have been explored in practice, there is very little prior work that has studied this
direction from a theoretical point of view aiming to understand the
assumptions required to guarantee task completeness, and a systematic
study to analyze its empirical performance in practice. This thesis
aims to fill this gap and develop a better understanding when, where,
and how these methods work well in practice.

\section{Thesis Goal and Contributions}
\label{sec:thes-goal-contr}

While most existing works have presented and studied approaches that
use the experience from executions to update the dynamics of the
inaccurate model, one can argue that this is wasteful if we are interested in
completing the task and not in modeling the dynamics
accurately. Furthermore, robots operating in the real world have
operating constraints that require them to quickly adapt to
new scenarios and not spend hours acquiring experience to learn true
dynamics. Finally in tasks where the dynamics are complicated, it could be
computationally infeasible to have a representation in the model class that can
capture the true dynamics. In light of these challenges and insights, the goal
of this thesis is to justify the following statement:

\begin{quote}
\textit{By updating the
behavior of the planner and not the dynamics of the model, we can
leverage simplified and potentially inaccurate models and
significantly reduce the amount of real-world experience needed to
provably guarantee that the robot completes the task.}
\end{quote}

\begin{figure}[t]
  \centering
  \includegraphics[width=0.5\linewidth]{figures/intro/zucker}
  \caption{A practitioner's approach to dealing with inaccuracies in
    dynamical models used for planning. In this example, the robot is
    planning a footstep trajectory along the partially unknown terrain
  to reach the other side. The planner has access to a model of the
  terrain which is inaccurate in the regions marked by red oval. To
  ensure that the planner does not compute any trajectory going
  through the red oval region, practitioners typically inflate the
  cost of any action executed within the region or any action that
  takes the robot into this region. This results in biasing the
  planner away from this region thereby updating its behavior. Figure
  taken from \cite{DBLP:journals/ijrr/ZuckerRSCBAK11} and the red oval
  region depicted is an example used for emphasis.}
\label{fig:zucker}
\end{figure}

% We support this argument by presenting two methods that update the
% behavior of the planner and do not require any updates to the dynamics
% of the inaccurate model used for planning.  Both methods come with
% provable guarantees on completing the task under some assumptions on the
% accuracy of the model. In addition to these methods, we also emphasize
% the importance of using model-based methods
% by analyzing the sample complexity (or the amount of experience
% needed) of exploration techniques used in model-free RL methods. Our
% analysis shows that undirected global exploration techniques popularly
% used in model-free RL methods can result in large sample complexity
% requirements that cannot be realized in practice on a robot. Furthermore to understand the
% effectiveness of using a potentially inaccurate model, we consider the linear
% quadratic control problem with unknown transition dynamics and show that the
% worst case performance that can be achieved is bounded even with significant
% modeling errors.

We support this argument through the primary contributions of this thesis which
are detailed in the following sections.

%The primary contributions of this thesis can be detailed as follows.

\subsection{Sample Complexity of Exploration in Model-Free Policy
  Search}
\label{sec:sample-compl-expl}
We analyze the sample complexity of exploration techniques in
  model-free RL methods. This analysis is presented by viewing model-free policy
  search methods through the lens of derivative-free optimization (DFO)
  and computing worst case upper bounds on the number of samples
  required to compute a $\epsilon$-suboptimal policy. We present a DFO
  point of view for methods that involve either exploration in action
  space or exploration in policy space, and present trade-offs between
  both styles of exploration in terms of the dimensionality of the
  policy parameter space, and the horizon length of the task. This
  analysis is presented in Chapter~\ref{CHA:ARS} of the
  thesis and is also presented in our paper~\cite{aistats19}. In addition
  to contrasting exploration in policy space vs action space, this
  work also emphasizes the large sample complexity required by
  model-free methods, which cannot be realized in practice on a robot.
  
\subsection{Planning and Execution using Inaccurate Models}
\label{sec:plann-exec-using}
  We present the first systematic effort to understand methods
  that use online experience from executions to update the behavior of
  the planner and not update the dynamics of the model. These
  methods can make progress towards completing the task despite using
  a potentially inaccurate model. One can construct cases where if the
  model is highly inaccurate (e.g. a model that predicts a humanoid
  falling down for any action and failing to complete the task of
  moving forward,) then such a method cannot be expected to finish the
  task. Hence, we study the assumptions required on the accuracy of
  the model used for planning that ensures task completeness without
  requiring any updates to the dynamics of the model. Furthermore, we
  frame our problem in the purely online setting where the experience
  gathered by the robot is along a single trajectory without any
  access to resets. We believe that this setting is realistic and has
  challenges that these methods are uniquely positioned to tackle.

  We propose \cmax{}, an approach that guarantees that the robot
  completes the task using the inaccurate model without any resets and
  without requiring any updates to the dynamics of the model. This is
  achieved by biasing the planner away from transitions whose dynamics
  are discovered to be inaccurately modeled during online
  execution. On the theoretical side, we establish provable guarantees
  on task completeness under assumptions on the accuracy of the model
  used for planning. Empirically, we show that \cmax{} outperforms
  state-of-the-art model-free and model-based RL methods in terms of
  the number of executions taken to complete the task. Crucially,
  \cmax{} exhibits goal-driven behavior which enables it to focus on
  completing the task as quickly as possible and not waste executions  
  learning the true dynamics. This method is explained in detail in
  Chapter~\ref{CHA:CMAX} and is also presented in our
  paper~\cite{cmax}.

\subsection{Leveraging Experience in Planning and Execution using
  Inaccurate Models}
\label{sec:lever-exper-plann}
  
While a robot using \cmax{} is provably guaranteed to complete
  the task, it requires strong assumptions on the accuracy of the
  model that are often not realized in practice and hard to verify
  prior to execution. Furthermore for repetitive tasks, \cmax{} fails
  to improve the quality of the solution over repetitions of the same
  task as it does not leverage previously discovered inaccurately
  modeled transitions. This is remedied by our second approach
  \cmaxpp{} that leverages experience from past executions to improve
  the quality of solution over repetitions of the same
  task. Crucially unlike \cmax{}, \cmaxpp{} can compute solution
  plans that contain previously discovered inaccurately modeled
  transitions. \cmaxpp{} achieves this by integrating model-free value
  learning using acquired experience with model-based planning using
  the inaccurate model. As a consequence of this in addition to
  completeness, \cmaxpp{} also guarantees asymptotic convergence to
  the optimal path cost as the number of repetitions increases. These
  guarantees of \cmaxpp{} are established under assumptions on the
  accuracy of the model that are much more relaxed compared to the
  assumptions required by \cmax{}. Importantly, like \cmax{},
  \cmaxpp{} never updates the dynamics of the model. This method is
  explained in detail in Chapter~\ref{CHA:CMAXPP} and is also
  presented in our paper~\cite{cmaxpp}.

\subsection{On the Effectiveness of Using Inaccurate Models}
\label{sec:effect-using-inacc}

In Chapters~\ref{CHA:CMAX} and~\ref{CHA:CMAXPP}, the focus is on proving task
completeness guarantees for methods that update the behavior of the planner.
However, there are no provable guarantees on the performance of the plan as a
function of the modeling error. This is especially useful in understanding how
accurate a model needs to be, in order to converge to a plan that has bounded
worst case performance. To derive such a guarantee, we study the control of
linearized systems with quadratic costs which is easier to analyze and provides insights
into the performance we can expect from using an inaccurate model. In this
setting, we analyze the performance of \textit{iterative learning control} (ILC)
approaches that use online experience from executions to update the behavior of
the planner and do not update the model, similar to \cmax{} and \cmaxpp{}. We
present upper bounds in terms of modeling error on the sub-optimality gap between the cost incurred by the
controller that ILC converges to, and the cost incurred by using the optimal
linear quadratic controller. Our analysis shows that the sub-optimality gap
bound for ILC has a nice quadratic dependency on the modeling error. Furthermore, our
analysis also highlights the
pitfalls of methods that naively use inaccurate models without updating the
behavior of the planner. For these methods, the sub-optimality gap bound has a
dependence on quadratic and higher-order terms in modeling error that can make it
significantly worse than ILC when modeling error is significant. This analysis
is explained in detail in Chapter~\ref{CHA:ILC} and is also presented in our
paper~\cite{ilc}.

\subsection{Task-Aware Online Model Search with Misspecified Model Classes}
\label{sec:task-aware-model}

Our final contribution departs from the algorithms developed so far, and studies
the alternative of using experience from executions to update the dynamics of
the model. More precisely, we study the problem of planning in an environment
with unknown transition dynamics when given access to a misspecified model
class. A model class is misspecified if no model in the class can capture the
true dynamics of the environment, which is usually the case in real-world
domains where we either have limited domain knowledge or we would like to use a
small model for computational efficiency. In such a setting, finding a model
that optimizes prediction error, as done in most of the existing work, can
lead to poor task performance when the model is used for planning. We develop a
model learning method \taml{} that is task-aware, i.e. optimizes completing the task
when used for planning, rather than prediction error. We achieve this
by performing derivative-free optimization in the space of model
parameters to explicitly select a model that 
results in a policy, upon planning, that achieves the best task performance
during execution. To measure the performance of any policy in the
environment without executing it, we rely on a monte-carlo evaluation procedure
that uses the experience accumulated so far in
state-action regions where we have good coverage, and fall back on an
optimistic model everywhere else. Theoretically, we can show that given
unlimited computation, \taml{} is guaranteed to reach the goal in
small state-action spaces as long as there is at least one good
performing model in the model class. Empirically, we show that \taml{} performs
significantly better than traditional model learning methods that optimize
prediction error in a simple mountain car domain. This work is explained in
detail in Chapter~\ref{CHA:TAML}.

\section{Bibliographical Remarks}
\label{sec:bibl-remarks}

This thesis only contains works for which this author is the primary
contributor.

Chapter~\ref{CHA:ARS} is based on joint work with Wen Sun and Drew Bagnell
that appeared in~\cite{aistats19}.
Chapter~\ref{CHA:CMAX} is based on joint work with Yash Oza, Drew Bagnell, and Max
Likhachev that appeared in~\cite{cmax}.
Chapter~\ref{CHA:CMAXPP} is based on joint work with Drew Bagnell and Max
Likhachev that appeared in~\cite{cmaxpp}.
Chapter~\ref{CHA:ILC} is based on joint work with Wen Sun, Max Likhachev,
and Drew Bagnell that appeared in~\cite{ilc}.
Chapter~\ref{CHA:TAML} is based on ongoing work with Sanjiban Choudhury, Drew
Bagnell, and Max Likhachev.

\section{Open Source Software}
\label{sec:open-source-software}

The author is a huge proponent of open sourcing research
code. Previously open sourced code has hugely benefitted the work in
this thesis, and the author would like to give back to the community
by open sourcing all the code for this thesis. The links to code are
listed below:
\begin{enumerate}
\item Chapter~\ref{CHA:ARS} code can be found at
  \url{https://github.com/LAIRLAB/contrasting_exploration_rl}
\item Chapter~\ref{CHA:CMAX} code can be found at
  \url{https://github.com/vvanirudh/CMAX}
\item Chapter~\ref{CHA:CMAXPP} code can be found at
  \url{https://github.com/vvanirudh/CMAXPP}
\item Chapter~\ref{CHA:ILC} code can be found at
  \url{https://github.com/vvanirudh/ILC.jl}
\item Chapter~\ref{CHA:TAML} code can be found at
  \url{https://github.com/vvanirudh/TOMS.jl}
\end{enumerate}

\section{Excluded Research}
\label{sec:excluded-research}

The author has excluded a significant portion of his doctoral work for the
purpose of keeping this thesis succinct. The excluded works are listed below:
\begin{enumerate}
  \item TRON: A Fast Solver for Trajectory Optimization with Non-Smooth Cost
        Functions, that appeared in~\cite{tron}.
  \item Planning, Learning and Reasoning Framework for Robot Truck Unloading,
        that appeared in~\cite{truck}.
  \item Provably Efficient Imitation Learning from Observation Alone, that
        appeared in \cite{fail}.
  \item Improved Soft Duplicate Detection in Search-Based Motion Planning, that
        appeared in~\cite{duplicate}.
  \item Task-informed Fidelity Management for Speeding up Robotics Simulation,
        that appeared in~\cite{fidelity}.
\end{enumerate}


%%% Local Variables:
%%% mode: latex
%%% TeX-master: "../main"
%%% End:

\chapter{Background}
\label{cha:background}

\epigraph{\textit{If I have seen further it is by standing on the
    shoulders of Giants.}}{Isaac Newton (1676)}

In this chapter, we provide background knowledge with the aim of
introducing classical techniques
that we will use throughout this thesis. Each chapter of this thesis
is self-contained and has detailed definitions that are more tailored
to the specific problem tackled in each chapter.

\section{Fundamentals of Markov Decision Processes}
\label{sec:fund-mark-decis}

In this thesis, we will primarily deal with finite horizon problems
where the objective is for an agent to minimize cumulative cost incurred over a
horizon of finite length. This is typically formulated as a finite horizon Markov
Decision Process (MDP)~\cite{bellman} that is represented as
$(\statespace, \actionspace, P, c, H)$ where $\statespace$ is the set
of states that the agent can be in, $\actionspace$ is the set of
actions that the agent can execute, $P$ is the transition dynamics
such that for any $s_t \in \statespace$, $s_{t+1} \in \statespace$,
$a_t \in \actionspace$, $P(s_{t+1}|s_t, a_t)$ is the probability of
transitioning to state $s_{t+1}$ from state $s_t$ by taking action
$a_t$, $c$ is the cost function such that for any transition $(s_t,
a_t)$, $c(s_t, a_t)$ is the cost incurred for that transition, and $H
\in \mathbb{N}^+$ is the length of the horizon. Typically, this
formulation also has a discounting factor $\gamma$ and a initial state
distribution $\rho$. In this thesis, we consider the non-discounted
setting where $\gamma = 1$ and a fixed initial state $s_0$ that is
known (thus, $\rho$ is a delta distribution on $s_0$.)

A deterministic policy $\pi: \statespace \rightarrow \actionspace$
maps from a state to an action. Given $\pi$ and a time step $t$, we can define the 
cost-to-go or value estimate $V_\pi^t(s)$ as follows:
\begin{equation}
  \label{eq:3}
  V_\pi^t(s) = \mathbb{E}\left[ \sum_{i=t}^H c(s_i, a_i) | a_i =
    \pi(s_i), s_{i+1} \sim P_{s_i, a_i}, s_t = s \right]
\end{equation}
where $P_{s_i, a_i} = P(\cdot|s_i, a_i)$ is a distribution over the
next state.

Similarly, we can also define the state-action value estimate
$Q_\pi^t(s, a)$ as follows:
\begin{equation}
  \label{eq:4}
  Q_\pi^t(s, a) = c(s, a) + \mathbb{E}_{s' \sim P_{s,a}}[V_\pi^{t+1}(s')]
\end{equation}

The objective function $J(\pi)$ is defined as
\begin{equation}
  \label{eq:6}
  J(\pi) = V_\pi^0(s_0)
\end{equation}
and the goal is to find a policy from a given policy set $\Pi$ that
minimizes the above objective function.

If the cost function and the transition dynamics is known, then one
can compute the optimal policy using Dynamic Programming (DP.) Denote
the optimal policy using $\pi^*$. Define $Q_{\pi^*}^{H-1}(s, a) = c(s,
a)$, we perform DP as follows: Starting from $t = H-2$ until $t = 1$
we iteratively do
\begin{align}
%  \label{eq:7}
  V_{\pi^*}^t(s) &= \max_{a \in \actionspace} Q_{\pi^*}^t(s, a) \\
  Q_{\pi^*}^{t-1}(s, a) &= c(s, a) + \mathbb{E}_{s' \sim P_{s, a}}[V_{\pi^*}^t(s')]
\end{align}
Given $Q_{\pi^*}^t$ we can compute the optimal action at time step $t$
and state $s_t$ 
as $\min_{a \in \actionspace} Q_{\pi^*}^t(s_t, a)$. The above
iterative process is called as Value Iteration. This iterative
procedures is derived by observing that the value function of the
optimal policy satisfies the following fixed point equation
\begin{equation}
  \label{eq:8}
  V_{\pi^*}^t(s) = \min_{a \in \actionspace} \left( c(s, a) + \mathbb{E}_{s'
  \sim P_{s,a}}[V_{\pi^*}^{t+1}(s')]\right)
\end{equation}
The above equation is known as the Bellman optimality condition.

\section{Deterministic Shortest Path Problem}
\label{sec:determ-short-path}

The shortest path problem is to find among all paths that start at a
given state and end at a goal state, a path has the minimum cost; this
is also called a shortest path~\cite{bertsekas1995neuro}. This can be instantiated as a markov
decision process represented using the tuple $(\statespace,
\actionspace, \goalspace, f, c)$ where $\goalspace \subseteq
\statespace$ is the set of goal states, and $f:\statespace \times
\actionspace \rightarrow \statespace$ is a deterministic
dynamics function that determines the successor state $s_{t+1}$ of a
transition $(s_t, a_t)$ as $f(s_t, a_t)$. The goal states in
$\goalspace$ are cost-free termination states, i.e. $f(g, a) = g$, $c(g,
a) = 0$ for
all $g \in \goalspace$ and any action $a \in \actionspace$. We also
assume that the cost of any transition starting from a non-goal state
is positive, i.e. $c(s, a) > 0$ for all $s \in \statespace \setminus
\goalspace$ and $a \in \actionspace$.

We are interested in problems where reaching the termination state is
inevitable, at least under an optimal policy. Thus, the essence of the
problem is how to reach a goal state with minimum cost. In the
shortest path problem setting, we use $V(s)$ to denote the cost-to-go
(or value) estimate of any state $s \in \statespace$ and $V^*(s)$ to
denote the optimal cost-to-go (or value.) From the Bellman optimality
condition we know
\begin{equation}
  \label{eq:9}
  V^*(s) = \min_{a \in \actionspace}\left( c(s, a) + V^*(f(s, a)) \right)
\end{equation}
A value estimate $V$ is called admissible if underestimates the
optimal value at all states, i.e. $V(s) \leq V^*(s)$ for all $s \in
\statespace$. Furthermore, $V$ is called consistent if it satisfies
the condition that for any state-action pair $(s, a)$, $s \notin
\goalspace$, $V(s) \leq c(s, a) + V(f(s, a))$ and $V(g) = 0$ for all
$g \in \goalspace$.
A typical assumption made in all shortest path problems is that there
exists at least one path from each state $s \in \statespace$ to one of
the goal states in $\goalspace$. This ensures that the optimal value
for any state is finite.

\section{Real-time Heuristic Search}
\label{sec:real-time-heuristic-1}

A traditional way to solve the shortest path problem is to search the
graph constructed using a mental model of the world, and then
subsequently execute the resulting plan (or follow the computed path.)
Thus, planning and execution are completely separated. An alternative
way of solving this problem is to search online by interleaving
planning and execution which results in several advantages with the
major advantage being drastic reductions in planning time. This is
achieved by performing search locally until a fixed horizon (or until
a fixed number of states are expanded,) and then execute the best
action for the current state. After the execution, planning is
performed once again to find the next best action. This can decrease
the time used for planning as we are not planning all the way to the
goal. Another significant advantage is when the mental model of the
world is inaccurate, these methods enable the agent to update its
model and ensure future replanning results in more optimal paths.

Since the future consequences of executed actions are unknown,
interleaving planning and execution can result in slight overhead in
terms of the number of actions executed but this is often a much
smaller overhead compared to the reduced planning time. Real-time
search methods are methods that interleave planning and execution by
searching forward from the current state of the agent. Most
importantly, real-time heuristic search methods can satisfy hard
real-time requirements in large state spaces since the sizes of their
local search spaces are independent of the sizes
of the state spaces and can thus remain small.

\subsection{LRTA*}
\label{sec:lrta}

In this thesis, we will focus on Learning real-time A* (LRTA*) real-time search methods
that are real-time search methods that associate information with the
states to prevent cycling. These methods are promising for
interleaving planning and execution as they are efficient
domain-independent search methods that allow fine-grained control over
how much planning is allowed between executions, use heuristic
knowledge to guide planning, and improve their performance over time
as they solve similar planning tasks. LRTA* operates on deterministic
domains only.

\begin{algorithm}[t]
  \caption{LRTA* with Lookahead $1$~\cite{DBLP:journals/ai/Korf90}}
  \begin{algorithmic}[1]
    \State $s \leftarrow s_0$
    \While{$s \notin \goalspace$}
    \State Compute action $a = \argmin_{a \in \actionspace} \left( c(s, a) +
      V(f(s, a)) \right)$
    \State Update $V(s) \leftarrow \min\left( V(s), c(s, a) + V(f(s,
      a)) \right)$
    \State Execute action $a$ and update $s \leftarrow f(s, a)$
    \EndWhile
  \end{algorithmic}
  \label{alg:lrta}
\end{algorithm}

Algorithm~\ref{alg:lrta} presents the LRTA* algorithm for a lookahead
or search horizon of $1$. At each time step, the algorithm looks one
action execution ahead and always greedily chooses the action that
leads to a successor state with the minimum sum of cost of
transitioning into the successor state and the value estimate of the
successor state. Furthermore unlike classical real-time search
methods, LRTA* also updates the value estimates of the current state
to reflect the updated estimate of the best path to the goal so that
future replanning is more efficient. The planning time of LRTA*
between executions is linear in the number of actions. If the size of
action space is independent of the size of state space, then the
planning time is independent of the size of state space which is a
major improvement over offline planning methods whose computational
complexity is at most the size of the state space.

LRTA* can be viewed as a form of asynchronous incremental dynamic
programming method~\cite{DBLP:journals/ai/BartoBS95}. It can be shown
that LRTA* is guaranteed to reach a goal state in a finite number of
executions and if we reset to the start state after reaching the goal
state, then the value estimates eventually converge to the optimal
value function~\cite{DBLP:journals/ai/Korf90}. These guarantees hold
under the assumption that the initial value estimates that we start
with are admissible and consistent. These assumptions are very similar
to the traditional definitions of admissible and consistent heuristic
values for A* search. Note that zero-initialized value estimates are
both admissible and consistent.

\subsection{RTAA*}
\label{sec:rtaa}

Real-time Adaptive A* (RTAA*) proposed
in~\cite{DBLP:conf/atal/KoenigL06} is similar to LRTA*
(Algorithm~\ref{alg:rtaa}). They only 
differ in the way they update the
value estimates at each time step. To understand this better, let us
look at how LRTA* updates value estimates. LRTA* replaces the value
estimate of each expanded state with the sum of costs of from the
state to a generated but unexpanded state $s$ (leaf node in the search
tree) and the value estimate of state $s$, minimized over all
generated but unexpanded states (all leaf nodes of the search tree.)
If we denote $V$ as the value estimates after all the value
updates then the LRTA* updates satisfy the following system of
equations for all expanded states $s$:
\begin{equation}
  \label{eq:10}
  V(s) = \min_{a \in \actionspace}\left( c(s, a) + V(f(s, a)) \right)
\end{equation}

\begin{algorithm}[t]
  \caption{RTAA* with lookahead $K \geq 1$~\cite{DBLP:conf/atal/KoenigL06}}
  \begin{algorithmic}[1]
    \State $s \leftarrow s_0$
    \While{$s \notin \goalspace$}
    \State Construct a search tree at $s$ until $K$ expansions
    \State Estimate $\bar{s}$ as the leaf node with the least $g + V$
    estimate among all leaf nodes
    \For{all expanded states $s'$}
    \State Update $V(s') \leftarrow g(\bar{s}) + V(\bar{s}) - g(s')$
    \EndFor
    \State Compute action $a$ as the first action on the path from
    state $s$ to state $\bar{s}$ in the search tree
    \State Execute action $a$ and update $s \leftarrow f(s, a)$
    \EndWhile
  \end{algorithmic}
  \label{alg:rtaa}
\end{algorithm}

On the other hand, RTAA* constructs a search tree very similar to
LRTA* (the number of states expanded is equal to the lookahead) but
updates the value estimates for all expanded states $s$ as follows:
\begin{equation}
  \label{eq:11}
  V(s) = g(\bar{s}) + V(\bar{s}) - g(s)
\end{equation}
where $g(s)$ encodes the cost-to-come from the root of the search tree
to the state $s$ (i.e. sum of costs of all transitions on the path
from root to state $s$,) and $\bar{s}$ is the state corresponding to
the leaf node with the least sum of $g$ and $V$ among all leaf nodes
in the search tree. In other words, $\bar{s}$ is the state that was
about to be expanded just before the search was terminated. One can
show that LRTA* and RTAA* updates are exactly the same when the
lookahead is $1$. But when the lookahead is greater than $1$, these
updates differ. More specifically, LRTA* updates tend to be more
informed or reflect the optimal cost-to-go better when compared to
RTAA* updates. However, it takes LRTA* more time to update the value
estimates and is difficult to implement. This is because LRTA*
performs one search to determine the local search space and a second
search to determine how to update the value estimates since it is
unable to use the results from the first search for this
purpose. Thus, there is a trade-off between the total search time and
cost of the resulting path. In practice, for lookaheads greater than
$1$, RTAA* tends to compute solution paths that have higher costs
compared to LRTA* but the time taken for planning before each execution is
significantly less in RTAA* compared to LRTA*. This makes RTAA*
desirable in applications where planning is slow but actions can be
executed fast and there is a very strict time limit per search episode.

\section{Local Function Approximation Methods}
\label{sec:local-funct-appr-1}

The goal of function approximation methods is to capture the
underlying relationship between input and output data. A typical
approach is to use all the training data to fit a global model that
predicts the output given the input throughout the input space. The
hope is that this approximation predicts output values that are close to the
true output values of the original function. A major disadvantage of
these global function approximation methods is that in many cases,
there exists no parameter values that provide a sufficiently good
approximation. Moving to a larger function approximation class with
more parameters requires a significantly larger training data which
might not be available. Furthermore, in cases where the model needs to
be updated incrementally, the computational cost of recomputing the
global function approximation is very high and potentially infeasible
on real-time systems.

An alternative to global function approximation methods are local
function approximation methods such as Locally Weighted Learning
(LWL)~\cite{DBLP:journals/air/AtkesonMS97a,
  DBLP:journals/air/AtkesonMS97}. LWL methods are 
non-parametric and prediction is computed using local functions which
use only a subset of the training data. The basic idea of LWL is for
each query point, a local model is constructed based on
neighboring training data. Each data point is associated with a
weighting factor that captures the influence of the data point in
computing the prediction for the query point. Intuitively, the closer
the data point to the query point the higher its influence. Since the
training data is directly used during prediction and there is no
pre-processing before prediction, LWL can be a very accurate and fast
incremental function approximation method.

For ease of exposition, let us consider the following regression
model
\begin{equation}
  \label{eq:12}
  y = f(\xbold) + \epsilon
\end{equation}

where $f(\xbold)$ is the unknown function that we are seeking to
approximate, $\xbold \in \mathbb{R}^d$, $y \in \mathbb{R}$ and
$\epsilon$ is zero mean noise. Given a dataset
$\buffer = \{(\xbold_i, y_i)\}_{i=1}^N$ and a query point $\xbold_q$ we can define
the following cost function,
\begin{equation}
  \label{eq:13}
  J(\beta_q) = \frac{1}{N}\sum_{i=1}^N w_i(\xbold_q)(y_i - \betabold_q^T\xbold_i)^2
\end{equation}
where $w_i$ are weights that capture the influence of the $i$-th data
point $(\xbold_i, y_i)$ on the prediction for query point $\xbold_q$, and
$\betabold_q$ is the coefficients for our linear model that is used for
prediction. The goal is to find $\betabold_q$ that minimizes the above
cost function and predict $\hat{y}_q = \betabold_q^T\xbold_q$ (Assume
that $\xbold_i, \xbold_q$ vectors have a $1$ added to account for the
offset term.) The weights $w_i(\xbold_q)$ are computed typically using
a distance metric $d(\xbold_i, \xbold_q)$ that captures relevance of
training points to the query point, and a kernel function $K(d)$ which
computes the weight given a distance value.

\subsection{K-Nearest Neighbor Regression}
\label{sec:k-nearest-neighbor}

A very simple LWL method is K-Nearest Neighbor (KNN) regression which
given a query point $\xbold_q$ finds the $K$ nearest neighbors in the
training data $\buffer$ using a distance metric $d(\xbold_q,
\xbold_i)$. There are several variants of this method, one of which
uniformly weights all the $K$ nearest neighbor's outputs to obtain the
prediction for the query point, i.e.
\begin{equation}
  \label{eq:14}
  \hat{y}_q = \frac{1}{K} \sum_{\xbold_i \in \buffer_K(\xbold_q)} y_i
\end{equation}
where $\buffer_K(\xbold_q)$ represents the set of size $K$ consisting
of the $K$ nearest neighbors of the query point $\xbold_q$ in
$\buffer$. Another variant, which often works well in practice, is to
weigh each neighbor by the inverse of their distance to the query
point. Intuitively, closer neighbors have a greater influence than
farther neighbors. Thus, we have
\begin{equation}
  \label{eq:15}
  \hat{y}_q = \frac{\sum_{\xbold_i \in \buffer_K(\xbold_q)} w_iy_i}{\sum_{\xbold_i \in \buffer_K(\xbold_q)} w_i}
\end{equation}
where the weight $w_i = \frac{1}{d(\xbold_q, \xbold_i)}$ is the
inverse of distance to the query point. KNN regression requires
storing all the training data in memory in the form of K-d trees which
allow very fast computation of $K$ nearest neighbors.

\subsection{Locally Weighted Regression}
\label{sec:locally-weight-regr}

Locally weighted regression (LWR) is a locally weighted learning
method that maintains all the training data in memory and quickly
computes the prediction for any given query point. LWR is presented in
Algorithm~\ref{alg:lwr} which is executed once for each query point
$x_q$. There are also simple extensions to the batch setting where we
want to obtain predictions for a batch of query points. The only
hyperparameter is the matrix $D$ which is usually set to a scaled
identity matrix $D = h\mathbb{I}$ where $h$ is a scalar hyperparameter
that is chosen using cross validation. LWR typically has a very high
approximation accuracy due to its local nature and has only few
hyperparameters. The disadvantage is that Algorithm~\ref{alg:lwr} has
a computational complexity of $\mathcal{O}(N^2)$ where $N$ is the size
of training data, which can be very expensive for large
datasets. Furthermore, LWR requires you to store all the training data
in memory which might be infeasible for extremely large datasets.

\begin{algorithm}[t]
  \caption{Locally Weighted Regression}
  \begin{algorithmic}[1]
    \State \textbf{Input:} Training data $\buffer = \{(\xbold_i,
    y_i)\}_{i=1}^N$, Query point $\xbold_q$
    \State Construct matrix $X$ with rows corresponding to $\hat{\xbold}_i$
    where $\hat{\xbold}_i = [\xbold_i^T 1]^T$
    \State Construct vector $\ybold$ with each element corresponding
    to $y_i$
    \State Compute diagonal weight matrix $W$ where the $i$-th
    diagonal element is given by $\exp\left( -\frac{1}{2} (\xbold_i -
      \xbold_q)^TD(\xbold_i - \xbold_q) \right)$
    \State Compute $\beta_q = (X^TWX)^{-1}X^TW\ybold$
    \State Compute prediction $\hat{y}_q = \beta_q^T\hat{\xbold_q}$
    where $\hat{\xbold_q} = [\xbold_q^T 1]^T$
  \end{algorithmic}
  \label{alg:lwr}
\end{algorithm}


%%% Local Variables:
%%% mode: latex
%%% TeX-master: "../main"
%%% End:

%\input{chapters/related-work}
\input{chapters/aistats19}
\input{chapters/cmax}
\input{chapters/cmaxpp}
\chapter{On the Effectiveness of Using an Inaccurate Model}
\label{CHA:ILC}

\epigraph{\textit{Internal models play an important role in generating
    command corrections from performance errors. As an internal model
    is made more accurate, learning efficiency and initial performance
    are improved.}}{Chae H. An, Christopher G. Atkeson, John
  M. Hollerbach (1988)}

The approaches, presented in Chapters~\ref{CHA:CMAX}
and~\ref{CHA:CMAXPP}, are
in the spirit of this thesis by using online experience to update the
behavior of the planner rather than updating the accuracy of the
model. To understand the usefulness of such approaches from a
theoretical perspective, we will take a closer look at a similar
approach that is popularly known as \textit{iterative learning
  control} in the literature. To derive concrete bounds on the
performance of the controller (or plan/policy) as a function of the
modeling errors, we consider the simplified setting of continuous
linearized systems with quadratic costs (LQR.) Using LQR as our
testbed allows easier analysis and provides insights into the
performance one can expect from feedforward adjustment to controls
using an approximate model, rather than updating the model itself. This chapter
is adapted from our original paper~\cite{ilc}.

\section{Introduction}
\label{sec:intro}

Iterative learning control (\ILC{}) has seen widespread adoption in a range of control
applications where the dynamics of the system are subject to unknown
disturbances or in instances where model parameters are
misspecified~\cite{DBLP:journals/jfr/MooreDB92}. While traditional feedback-based control methods
have been successful at tackling non-repetitive noise, \ILC{} has shown itself to
be effective at adjusting to repetitive disturbance through feedforward control
adjustment~\cite{arimoto84}. This was shown empirically in several robotic
applications such as manipulation~\cite{kuc91}, and quadcopter trajectory
tracking~\cite{schoellig12a, schoellig12b} among others.
% However there has been
% little prior theoretical work~\cite{an1988model} aimed at understanding why ILC is effective and
% robust in the presence of large modeling errors. \jab{This sentence is
% weird given the CGA reference.}
Prior work~\cite{an1988model} uses fixed point theory to analyze the
conditions for convergence of \ILC{} but does not present performance
bounds at convergence. Very recent
work~\cite{DBLP:conf/icml/AgarwalHMS21} presented a \ILC{} algorithm
that is robust to model mismatch and uncertainty. However, they
analyze the algorithm using planning regret, which measures regret
with respect to the best open loop plan in hindsight, and do not
study how the performance depends on modeling error.
Our work contributes to understanding the effectiveness of \ILC{}
%\jab{Don't think it fills, but contributes to our understanding}
by studying its worst case performance, as a function of modeling
error, in the linear quadratic
regulator (LQR) setting with unknown transition dynamics
and access to
an approximate model of the dynamics.

A simple approach to the LQR problem with an approximate
model of the dynamics is
% certainty equivalent (CE) control~\cite{astrom13}. It is
% based on the certainty equivalence principle which allows the synthesis of a
% controller by treating the approximate model as the true
% dynamics. \jab{This is not what I would call CE. I would call CE
% ignoring uncertainty.}
to do optimal control using the misspecified model (\MM{}.) The
resulting controller is similar to the certainty equivalent controller
obtained by performing optimal control on estimated parameters of the
regulator and ignoring the uncertainty of the estimates in adaptive
control~\cite{astrom13}.
Despite the
simplicity of \MM{}, it is challenging to quantify its suboptimality, with
respect to the optimal LQR controller, as a result of the
modeling errors in the approximate model.
% Very recent work from~\cite{mania19}
% has presented worst case cost suboptimality bounds for \MM{} in the
% infinite horizon LQR setting
% using fixed point discrete Ricatti perturbation bounds~\cite{konstantinov93}.
% In this work, we consider the more challenging finite horizon LQR
% setting for which the solution is not a fixed point.

%\jab{I think CE is the wrong phrase here generally.}

Our first contribution is proving worst case cost suboptimality bounds for
\MM{} in the finite horizon LQR setting in terms of the modeling error. This
requires us to depart from the fixed point analysis used in prior
work~\cite{mania19, konstantinov93}, as the
solution to the discrete Ricatti equation in the finite horizon is not a fixed
point. A key part of our analysis is establishing perturbation bounds by
carefully tracking the effect of modeling error through the horizon of the
control task. This allows us to quantify the worst case suboptimality gap of \MM{} in the
finite horizon LQR setting.

The second contribution is to utilize the same proof techniques as we used for
\MM{} to analyze the
suboptimality gap of \ILC{}. This allows us to explicitly compare the
worst case performance of
\ILC{} and \MM{} for LQR problems, and understand why \ILC{} works well in the regime of
large modeling errors when \MM{} often performs poorly.
Our analysis highlights that the suboptimality gap for \ILC{} is lower than that
for \MM{} by higher order terms that can become significant
when modeling errors are high. We also show that \ILC{} is capable of keeping the
system stable and cost from blowing up even in the presence of large modeling
errors, which \MM{} is incapable of. By interpreting the worst case
bounds, we identify several linear systems
with key characteristics that enable \ILC{} to be robust to large model
misspecifications, whereas \MM{} is unable to deal with model errors and
results in poor solutions.

The final contribution of this work is to present simple empirical experiments
involving optimal control tasks with linear and nonlinear dynamical systems that
back the theoretical findings from our analysis. The experiment results
reinforce our finding that in the regime of large modeling errors, \ILC{} performs
better than \MM{} and synthesizes control inputs that result in smaller suboptimality
gaps.

\section{Problem Setup}
\label{sec:problem-setup}

We consider the finite horizon linear quadratic regulator (LQR) setting with a
horizon $H$ and a fixed initial state $x_{0} \in \reals^{n}$. The dynamics of
the system are described by unknown matrices $A_{t} \in \reals^{n\times n}$ and
$B_{t} \in \reals^{n\times d}$ for $t=0, \cdots, H-1$ as follows:
%\begin{equation}
%  \label{eq:dynamics}
$x_{t+1} = A_{t}x_{t} + B_{t}u_{t}$
%\end{equation}
where $u_{t} \in \reals^{d}$ is the control input at time step $t$. Any sequence
of control inputs $(u_{0}, \cdots, u_{H-1})$ results in a state trajectory
$(x_{0}, \cdots, x_{H})$. The cost function is defined using matrices
$Q \in \reals^{n\times n}$, $Q_{f} \in \reals^{n\times n}$ and $R \in \reals^{d \times d}$ as follows:
\begin{equation}
  \label{eq:7}
  V_{0}(x_{0}) = \sum_{t=0}^{H-1} x_{t}^{T}Qx_{t} + u_{t}^{T}Ru_{t} + x_{H}^{T}Q_{f}x_{H}
\end{equation}

From optimal control literature~\cite{anderson07}, we know that the above cost is minimized by
a linear time-varying state-feedback controller $\KOPT = (\KOPT_{0}, \cdots, \KOPT_{H-1})$
with control inputs $u_{t} = \KOPT_{t}x_{t}$ satisfying:
\begin{align*}
  \KOPT_{t} &= -(R + B_{t}^{T}\POPT_{t+1}B_{t})^{-1}B_{t}^{T}\POPT_{t+1}A_{t} \\
  \POPT_{t} &= Q + A_{t}^{T}\POPT_{t+1}(I + B_{t}R^{-1}B_{t}^{T}\POPT_{t+1})^{-1}A_{t}
\end{align*}
where we initialize $\POPT_{H} = Q_{f}$ and the matrices $\POPT_{t}$ define the optimal
cost-to-go incurred using the optimal controller $\KOPT$ from time step $t$ as
$\VOPT_{t}(x_{t}) = x_{t}^{T}\POPT_{t}x_{t}$. For any controller $K$, we will
use the notation $M_{t}(K)$ to denote the matrix $A_{t} + B_{t}K_{t}$, and the
notation $L_{t}(K)$ to denote the product $\prod_{i=0}^{t}M_{i}(K)$. This is
useful for conciseness as we can observe that the state trajectory obtained
using $K$ can be expressed as $x_{t} = M_{t-1}(K)x_{t-1} = L_{t-1}(K)x_{0}$.

We are given access to an approximate model of the dynamics of the system
specified by matrices $\Ahat_{t} \in \reals^{n\times n}$ and
$\Bhat_{t} \in \reals^{n \times d}$ for $t = 0, \cdots, H-1$ such that there
exists some $\epsA, \epsB \geq 0$ (also referred to as the modeling error) satisfying
$||A_{t} - \Ahat_{t}|| \leq \epsA$ and $||B_{t} - \Bhat_{t}|| \leq \epsB$.
For the purposes of this paper, we use the notation $||\cdot||$ to refer to the
matrix norm induced by the L2 vector norm.
In this paper, we consider two control strategies: optimal control
using the misspecified model
(\MM{}) and iterative learning control (\ILC{}.)

\subsection{Optimal Control using Misspecified Model}
\label{sec:cert-equiv-contr}

Optimal control using misspecified model uses the
approximate model to synthesize a time-varying linear controller
$\KCE = (\KCE_{0}, \cdots, \KCE_{H-1})$ satisfying:
\begin{align*}
  \KCE_{t} &= -(R + \Bhat_{t}^{T}\PCE_{t+1}\Bhat_{t})^{-1}\Bhat_{t}^{T}\PCE_{t+1}\Ahat_{t} \\
  \PCE_{t} &= Q + \Ahat_{t}^{T}\PCE_{t+1}(I + \Bhat_{t} R^{-1}\Bhat_{t}^{T}\PCE_{t+1})^{-1}\Ahat_{t}
\end{align*}
where we initialize $\PCE_{H} = Q_{f}$ and the control inputs are defined as
$u_{t}^{\mathsf{CE}} = \KCE_{t}x_{t}$. One can observe that the controller
$\KCE$ results in suboptimal cost when executed in the system as it is
optimizing the cost under approximate
dynamics rather than the true dynamics of the system. Thus, the suboptimality
gap $\VCE_{0}(x_{0}) - \VOPT_{0}(x_{0})$ depends on the approximate dynamics
$\Ahat_{t}, \Bhat_{t}$, and how well they approximate the true dynamics.

\subsection{Iterative Learning Control}
\label{sec:iter-learn-contr-1}
%\jab{Might be worth tightening some the text up earlier so you can
%spare a line or two on the philosophy of ILC. I.e. why do it?}

Iterative learning control~\cite{arimoto84, DBLP:journals/jfr/MooreDB92} is a framework
that is used to efficiently calculate the feedforward input signal
adjustment by using information from previous trials to improve the
performance in a small number of iterations. An example of an \ILC{}
algorithm is shown in Algorithm~\ref{alg:ilc}.
\ILC{} assumes a rollout access to
the system, i.e. we are
allowed to conduct full rollouts of horizon $H$ in the system to evaluate the
cost and obtain the trajectory under true dynamics
(Line~\ref{line:rollout}). Note that this access is
only restricted to rollouts, and the true dynamics $A_{t}, B_{t}$ are unknown.
\ILC{} can be understood as an iterative shooting method where we
synthesize control inputs by 
always evaluating in the true system while computing updates to the controls
using the approximate model~\cite{abbeel2006using, DBLP:conf/icml/AgarwalHMS21}. In
Algorithm~\ref{alg:ilc}, this is achieved by linearizing the
dynamics and quadraticizing the cost around the observed
trajectory (Line~\ref{line:lqr}) resulting in an LQR problem with the objective:
\begin{equation}
  \label{eq:5}
  J(\Delta x, \Delta u) = \sum_{t=0}^{H-1} (2x_t + \Delta x_t)^TQ\Delta x_t +
    (2u_t + \Delta u_t)^TR\Delta u_t + (2x_H + \Delta x_H)^TQ_f\Delta x_H 
  \end{equation}
  where $x_{0:H}$ is the observed trajectory on the true system when
  executing controls $u_{0:H-1}$, and for any $t=0, \cdots, H-1$ we
  have $\Ahat_t\Delta x_t + \Bhat_t\Delta u_t = \Delta x_{t+1}$.
  \begin{algorithm}[t]
    \small
  \caption{\ILC{} Algorithm for Linear Dynamical System with
    Approximate Model}
  \label{alg:ilc}
  \begin{algorithmic}[1]
    \State {\bfseries Input:} Approximate model $\Ahat_t,
    \Bhat_t$, Initial state $x_0$, Step size $\alpha$, cost matrix
    $Q, R, Q_f$
    \State Initialize a control sequence $u_{0:H-1}$ using approximate
    model
    \While {not converged}
    \State Rollout $u_{0:H-1}$ on the true system to get trajectory
    $x_{0:H}$\label{line:rollout}
    \State Compute LQR solution $\arg\min_{\Delta x, \Delta u} J(\Delta x, \Delta u)$ subject to $\Ahat_t\Delta x_t + \Bhat_t\Delta u_t = \Delta x_{t+1}$ \label{line:lqr}
    \State Update $u_{0:H-1} = u_{0:H-1} + \alpha \Delta u_{0:H-1}$
    \EndWhile
  \end{algorithmic}
\end{algorithm}

At convergence in Algorithm~\ref{alg:ilc}, we have $\Delta u = 0$,
i.e. the LQR problem in line~\ref{line:lqr} returns the solution where
$\Delta u = 0$. The solution to the LQR problem can be derived in
closed form using dynamic programming, and for any $t \in \{0, \cdots,
H-1\}$ is given by
\begin{align*}
  \Delta u_t = -(R + \Bhat_t^T\PILC_{t+1}\Bhat_t)^{-1}(Ru_t + \Bhat_t^T\PILC_{t+1}x_{t+1})
\end{align*}
where $\PILC_{t+1}$ captures the cost-to-go from time step $t+1$ with
$\PILC_H = Q_f$. To obtain $\Delta u_t = 0$ for any $t \in \{0, \cdots, H-1\}$,
it is necessary for the following condition to hold,
% At convergence, we have $\Delta u = 0$ which is achieved when the
% gradient $\nabla_u (\sum_{t=0}^{H-1}x_t^TQx_t + u_t^TRu_t +
% x_H^TQ_fx_H)$ is zero.
%\todo[inline]{Add notes here}
% To characterize the control inputs \ILC{} converges to, let 
% us consider the gradient of the cost-to-go at time step $t$,
% $\nabla_{u}(x_{t}^{T}Qx_{t} + u_{t}^{T}Ru_{t} +
% x_{t+1}^{T}\PILC_{t+1}x_{t+1})$ where $\PILC_{t+1}$ captures the
% cost-to-go from time step $t+1$,
% and set it to zero:
\begin{align*}
  &Ru_{t} + \Bhat^{T}_{t}\PILC_{t+1}x_{t+1} = 0 \\
  &Ru_{t} + \Bhat^{T}_{t}\PILC_{t+1}(A_{t}x_{t} + B_{t}u_{t}) = 0 \\
  &u_{t} = -(R + \Bhat_{t}^{T}\PILC_{t+1}B_{t})^{-1}\Bhat_{t}^{T}\PILC_{t+1}A_{t}x_{t}
\end{align*}
where we use the rollout trajectory to obtain
$x_{t+1} = A_{t}x_{t} + B_{t}u_{t}$. It is important to note that converging to
these control inputs require carefully chosing appropriate step sizes $\alpha$ at each
iteration in the \ILC{} Algorithm~\ref{alg:ilc}. Thus, we can see that
the control inputs $u_{0:H-1}$
\ILC{} converges to can be described using a time-varying state-feedback linear
controller $\KILC$ defined as:
\begin{align*}
  \KILC_{t} &= -(R + \Bhat_{t}^{T}\PILC_{t+1}B_{t})^{-1}\Bhat_{t}^{T}\PILC_{t+1}A_{t} \\
  \PILC_{t} &=  Q + \Ahat_{t}^{T}\PILC_{t+1}(I + B_{t}R^{-1}\Bhat_{t}^{T}\PILC_{t+1})^{-1}A_{t}
\end{align*}
where we initialize $\PILC_{H} = Q_{f}$ and the control inputs are defined as
$u^{\mathsf{ILC}}_{t} = \KILC_{t}x_{t}$. We can observe that the \ILC{} converges
to control inputs that are different from the ones computed by the optimal
controller $\KOPT$, and hence achieves suboptimal cost. In the next few sections, we
will analyze the suboptimality bounds for both \MM{} and \ILC{}, and show how \ILC{}
converges to control sequence that achieves lower costs and is more robust to
high modeling errors when compared to \MM{}.

\subsection{Assumptions}
\label{sec:assumptions}
In this section, we will present all the assumptions used in our analysis.
Our first assumption is on the cost matrices $Q, Q_f$ and $R$, also used in~\cite{mania19}:
\begin{assumption}
  We assume that $Q, Q_f$, and $R$ are positive-definite matrices. Note that simply
  scaling all of $Q, Q_f$, and $R$ does not change the optimal controller $\KOPT$, so we can
  assume that the smallest singular value of $R$, $\ubar{\sigma}(R) \geq 1$.
  \label{assumption:singularvalue}
\end{assumption}
The above assumption allows us to ignore terms relating to singular
values of $R$ in the analysis, keeping it concise. The
next assumption states that the true system is stable under the
optimal controller $\KOPT$. Similar notions of stability have been considered in~\cite{cohen18}:
\begin{assumption}
  We assume that the optimal controller $\KOPT$ satisfies
  $||A_{t} + B_{t}\KOPT_{t}|| \leq 1 - \delta$ for some
  $0 < \delta \leq 1$ and all $t=0, \cdots, H-1$.
  \label{assumption:stability}
\end{assumption}
Observe that the above assumption implies that $||M_{t}(\KOPT)|| \leq 1 - \delta$
and $||L_{t}(\KOPT)|| \leq (1 - \delta)^{t+1} \leq e^{{-\delta(t+1)}}$.
Finally, we make a crucial assumption about the model that
is required for our \ILC{} analysis:
\begin{assumption}
  We assume that the matrix $B_{t}R^{-1}\Bhat_{t}^{T}$ has eigenvalues that have
  non-negative real parts for all $t=0, \cdots, H-1$. A sufficient condition for
  this to hold is that the
  modeling error satisfy
  $\epsB \leq \frac{\ubar{\sigma}(B_{t}^{T}RB_{t})}{||B_{t}^{T}R||}$
  for all $t=0, \cdots, H-1$.
  \label{assumption:psd}
\end{assumption}
The above assumption ensures that $x^{T}B_{t}R^{-1}\Bhat_{t}^{T}x \geq 0$ for any vector
$x \in \reals^{n}$ and all time steps $t$. Intuitively, if this is not true then \ILC{} is not
guaranteed to converge to a local minima. A more detailed
explanation is given in Appendix~\ref{sec:assumpt-refass}.

\section{Main Results}
\label{sec:main-results}

In this section, we will present the main results concerning the worst
case performance bounds of \MM{} and \ILC{} in the LQR setting with an
approximate model as described in Section~\ref{sec:problem-setup}. Our
first theorem
bounds the cost suboptimality of any time-varying linear 
controller $\Khat$ in terms of the norm differences $\|\KOPT_t -
\Khat_t\|$:
\begin{restatable}{theorem}{costTheorem}
  \label{theorem:cost}
  Suppose $d \leq n$. Denote
  $\Gamma = 1 + \max_t\{||A_t||, ||B_t||, ||\POPT_t||, ||\KOPT_t||\}$. Then under
  Assumption~\ref{assumption:stability} and if $||\KOPT_{t} -
  \Khat_{t}|| \le \frac{\delta}{2||B_{i}||}$ for all $t=0, \cdots, H-1$, we have
  \begin{equation}
    \label{eq:cost}
    \Vhat_0(x_0) - \VOPT_0(x_0) \leq d\Gamma^{3}\|x_{0}\|^{2} \sum_{t=0}^{H-1} e^{-\delta t}\|\KOPT_{t} - \Khat_{t}\|^{2}
  \end{equation}
\end{restatable}
The proof for the above theorem is given in
Appendix~\ref{sec:general-results-1}.

This theorem is central to our analysis as it states that as long as we can keep
the norm differences $||\Khat_{t} - \KOPT_{t}||$ small, then the cost
suboptimality scales 
with the norm difference squared at each time step and goes exponentially down
with time step. % TODO: Could add some more comments on interpreting
                % the above result
We will now present results on how we can
bound these norm differences for both \MM{} and \ILC{}.

\paragraph{Results for Optimal Control with Misspecified Model}
\label{sec:optimal-control-with}

Our next
lemma bounds the difference $\|\KCE_t - \KOPT_t\|$ in terms of
$\|\PCE_{t+1} - \POPT_{t+1}\|$ and modeling errors $\epsA, \epsB$:
\begin{restatable}{lemma}{ceLemma}
  \label{lemma:ce}
  If $||A_{t} - \Ahat_{t}|| \leq \epsA$ and
  $||B_{t} - \Bhat_{t}|| \leq \epsB$ for $t=0, \cdots, H-1$, and we have
  $||\POPT_{t+1} - \PCE_{t+1}|| \leq \fCE_{t+1}(\epsA, \epsB)$ for some function
  $\fCE_{t+1}$. Then we have under
  Assumption~\ref{assumption:singularvalue} for all $t=0, \cdots, H-1$,
  \begin{align}
    ||\KOPT_{t} - \KCE_{t}|| \leq 14\Gamma^3\epsilon_t
    \label{eq:ce-k-diff}
  \end{align}
  where
  $\Gamma = 1 + \max_{t}\{||A_{t}||, ||B_{t}||, ||\POPT_{t}||, ||\KOPT_{t}||\}$
  and $\epsilon_{t} = \max\{\epsA, \epsB, \fCE_{t+1}(\epsA, \epsB)\}$.
\end{restatable}
The proof for the above lemma is given in Appendix~\ref{sec:cert-equiv-contr-2}.

This result is very promising but there is a big piece still missing: how do we
bound $\fCE_{t+1}(\epsA, \epsB)$. To do this, we need to establish perturbation
bounds for the discrete ricatti equation in the finite
horizon setting. Prior work~\cite{konstantinov93, mania19} has only established such
bounds in the infinite horizon setting using fixed point analysis. Our treatment
is significantly different as the finite horizon solution is not a fixed point.
Our final perturbation bounds are presented in the theorem below:
\begin{restatable}{theorem}{theoremCE}
  \label{theorem:ce}
  If the cost-to-go matrices for the optimal controller and \MM{}
  controller are specified by $\{\POPT_t\}$ and $\{\PCE_t\}$ such that
  $\POPT_H = \PCE_H = Q_f$ then,
  \begin{align}
    \label{eq:14}
    ||\POPT_{t} - \PCE_{t}|| &\leq  \|A_{t}\|^{2}\|\POPT_{t+1}\|^{2}(2\|B_{t}\|\|R^{-1}\|\epsB + \|R^{-1}\|\epsB^{2}) \nonumber\\
    &+ 2\|A_{t}\|\|\POPT_{t+1}\|\epsA + \|\POPT_{t+1}\|\epsA^{2} \nonumber\\
                        &+ c_{\POPT_{t+1}}(\|A_{t}\|+ \epsA)^{2}||\POPT_{t+1} - \PCE_{{t+1}}||
  \end{align}
  for $t=0, \cdots, H-1$ where $c_{\POPT_{t+1}} \in \reals^+$  is a constant that is
  dependent only on $\POPT_{t+1}$ if $\epsA, \epsB$ are small enough such
  that $\|\POPT_{t+1} - \PCE_{t+1}\| \leq \|\POPT_{t+1}\|^{-1}$. Furthermore, the upper
  bound~\eqref{eq:14} is tight up to constants that only depend on the
  true dynamics $A_t, B_t$, cost matrix $R$, and $\POPT_{t+1}$.
\end{restatable}
The proof for the above theorem is given in Appendix~\ref{sec:cert-equiv-contr-2}.

The above theorem gives us an upper bound for $\fCE_{t}$ for $t=0,
\cdots, H-1$ in
Lemma~\ref{lemma:ce} with $\fCE_H = 0$. The resulting upper bound on $\|\KOPT_t -
\KCE_t\|$ from Lemma~\ref{lemma:ce} combined with
Theorem~\ref{theorem:cost} gives us the cost 
suboptimality bound for \MM{}. Notice that the
bound on $\fCE_t$ grows quickly as $t$ decreases making $\fCE_{t+1}$
in Lemma~\ref{lemma:ce} the dominant error term that affects the cost
suboptimality of \MM{}.

\paragraph{Results for Iterative Learning Control}
\label{sec:iter-learn-contr-3}

Our final set of results establish similar worst case cost
suboptimality bounds for \ILC{} by first establishing a bound on the
difference $\|\KILC_t - \KOPT_t\|$ in terms of $\|\PILC_{t+1} -
\POPT_{t+1}\|$ and modeling error $\epsA, \epsB$:
\begin{restatable}{lemma}{ilcLemma}
  \label{lemma:ilc}
  If $||A_{t} - \Ahat_{t}|| \leq \epsA$ and
  $||B_{t} - \Bhat_{t}|| \leq \epsB$ for $t=0, \cdots, H-1$, and we have
  $||P_{t+1} - \PILC_{t+1}|| \leq \fILC_{t+1}(\epsA, \epsB)$ for some function
  $\fILC_{t+1}$. Then we have under
  Assumption~\ref{assumption:singularvalue} for all $t=0, \cdots, H-1$,
  \begin{align}
    ||\KOPT_{t} - \KILC_{t}|| \leq 6\Gamma^3\epsilon_t
    \label{eq:ilc-k-diff}
  \end{align}
  where
  $\Gamma = 1 + \max_{t}\{||A_{t}||, ||B_{t}||, ||\POPT_{t}||, ||\KOPT_{t}||\}$
  and $\epsilon_{t} = \max\{\epsA, \epsB, \fILC_{t+1}(\epsA,
  \epsB)\}$. 
\end{restatable}
The proof for the above lemma is given in Appendix~\ref{sec:iter-learn-contr}.

Similar to \MM{}, we need to bound the crucial term
$\fILC_{t+1}(\epsA, \epsB)$ to bound the norm difference $\|\KOPT_t -
\KILC_t\|$ using Lemma~\ref{lemma:ilc}. We will present perturbation
bounds for the \ILC{}
recursion equation given in 
Section~\ref{sec:iter-learn-contr-1} in the finite horizon setting
below:
\begin{restatable}{theorem}{ilcTheorem}
  \label{theorem:ilc}
  If the cost-to-go matrices for the optimal controller and iterative learning
  control are specified by $\{\POPT_{t}\}$ and $\{\PILC_{t}\}$ such
  that $\POPT_{H} = \PILC_{H} = Q_f$ then we have under
  Assumption~\ref{assumption:psd},
  \begin{align}
    \label{eq:15}
    ||\POPT_{t} - \PILC_{t}|| &\leq  \|A_{t}\|^{2}\|\POPT_{t+1}\|^{2}\|B_{t}\|\|R^{-1}\|\epsB
    + \|A_{t}\|\|\POPT_{t+1}\|\epsA \nonumber\\
                        &+ c_{\POPT_{t+1}}||A_{t}||(\|A_{t}\|+ \epsA)||\POPT_{t+1} - \PILC_{{t+1}}||
  \end{align}
  for $t=0, \cdots, H-1$ where $c_{\POPT_{t+1}} \in \reals^+$ is a
  constant that is dependent only on 
  $\POPT_{t+1}$ if $\epsA, \epsB$ are small enough that $\|\POPT_{t+1}
  - \PILC_{t+1}\| \leq \|\POPT_{t+1}\|^{-1}$. Furthermore, the upper bound~\eqref{eq:15} is tight
  upto constants that depend only on the true dynamics $A_t, B_t$,
  cost matrix $R$, and $\POPT_{t+1}$. 
\end{restatable}
The proof for the above theorem is given in Appendix~\ref{sec:iter-learn-contr}.

The above theorem gives us a bound on $\fILC_t$ for $t=0, \cdots, H-1$
in Lemma~\ref{lemma:ilc} with $\fILC_H = 0$. The resulting upper bound
on $\|\KOPT_t - \KILC_t\|$ from Lemma~\ref{lemma:ilc} combined with
Theorem~\ref{theorem:cost} gives us the cost suboptimality bound for
iterative learning control. Similar to \MM{}, the dominant error term in
Lemma~\ref{lemma:ilc} turns out to be $\fILC_{t+1}$ especially for
smaller $t$ as the upper bound~\eqref{eq:15} grows quickly as $t$ decreases.

% \section{General Results}
% \label{sec:general-results}

% In this section, we will present general results that bound the cost suboptimality of
% any time-varying controller $\Khat$ in terms of the norm differences
% $||\KOPT_{t} - \Khat_{t}||$. Our first lemma makes use of
% Assumption~\ref{assumption:stability} to show that if the norm differences
% $||\KOPT_{t} - \Khat_{t}||$ are small, then the true system can be stable under
% $\Khat$:
% \begin{restatable}{lemma}{stabilityLemma}
%   \label{lemma:stability}
%   If Assumption~\ref{assumption:stability} holds and if $\Khat$ satisfies
%   $||\KOPT_{i} - \Khat_{i}|| \le \frac{\delta}{2||B_{i}||}$ for all
%   $i \in \{0, \cdots, H-1\}$, then we have
%   \begin{equation}
%     \label{eq:1}
%     ||L_{t}(\Khat)|| \leq \left(1 - \frac{\delta}{2}\right)^{t+1} \leq e^{-\frac{\delta}{2}(t+1)}
%   \end{equation}
% \end{restatable}
% \begin{proof}
%   Proof is given in Appendix~\ref{sec:general-results-1}.
% \end{proof}

% The next lemma is very similar to the performance difference lemma that was
% first proposed in~\cite{kakade2002approximately}. We borrow the version presented
% in~\cite{fazel18} and extend it to the finite horizon setting below:
% \begin{lemma}
%   \label{lemma:performance-difference}
%   Let $\xhat_0,
%   \uhat_0, \cdots, \xhat_H, \uhat_H$ be the trajectory generated by
%   controller
%   $\Khat$ using the true dynamics such that $\xhat_0 = x_0$, $\uhat_t = \Khat_t\xhat_t$ for
%   $t=0, \cdots, H-1$. Then
%   we have:
%   \begin{equation}
%     \label{eq:2}
%     \Vhat_0(x_0) - \VOPT_0(x_0) = \sum_{t=0}^{H-1} \AOPT_t(\xhat_t, \uhat_t) - \VOPT_H(\xhat_H)
%   \end{equation}
%   where $\Vhat_t$ is the cost-to-go using controller $\Khat$ from time
%   step $t$, $\VOPT_t$ is the cost-to-go using the optimal controller $\KOPT$ from time
%   step $t$, and $\AOPT_t(x, u) = \QOPT_t(x, u) - \VOPT_t(x)$ is the advantage of
%   the controller $\KOPT$ at time step $t$. Furthermore, we have that for any $x$
%   \begin{equation*}
%     %\label{eq:3}
%     \AOPT_t(x, \Khat_tx) = x^T(\Khat_t - \KOPT_t)^T(R + B_t^T\POPT_{t+1}B_t)
%     (\Khat_t - \KOPT_t)x
%   \end{equation*}
% \end{lemma}
% \begin{proof}
%   For proof, we refer the readers to~\cite{fazel18}.
% \end{proof}

% We use the performance difference lemma, as stated above, in the finite horizon
% LQR setup and make use of Lemma~\ref{lemma:stability} to establish the
% suboptimality bound in terms of the norm differences $||\KOPT_{t} - \Khat_{t}||$:
% \costTheorem*
% \begin{proof}
%   Proof is given in Appendix~\ref{sec:general-results-1}.
% \end{proof}
% This theorem is central to our analysis as it states that as long as we can keep
% the norm differences $||\Khat_{t} - \KOPT_{t}||$ small, according to
% Lemma~\ref{lemma:stability}, then the cost suboptimality scales
% with the norm difference squared at each time step and goes exponentially down
% with time step. We will now present results on how we can
% bound these norm differences for both \MM{} and \ILC{}.

% \section{Optimal Control with Misspecified Model Results}
% \label{sec:ce}
% In this section, we will present results pertaining to \MM{}, and bound the norm difference $||\KOPT_{t} - \KCE_{t}||$.
% Our first lemma uses some basic results on quadratic cost minimizers
% from~\cite{mania19} and makes use of Assumption~\ref{assumption:singularvalue}
% to bound $\|\KOPT_t - \KCE_t\|$ in terms of $\epsA, \epsB$ and the
% bound on $\|\POPT_{t+1} - \PCE_{t+1}\|$:
% \ceLemma*
% \begin{proof}
%   Proof is given in Appendix~\ref{sec:cert-equiv-contr-2}.
% \end{proof}

% This result is very promising but there is a big piece still missing: how do we
% bound $\fCE_{t+1}(\epsA, \epsB)$. To do this, we need to establish perturbation
% bounds for the discrete ricatti equation in the finite
% horizon setting. Prior work~\cite{konstantinov93, mania19} has only established such
% bounds in the infinite horizon setting using fixed point analysis. Our treatment
% is significantly different as the finite horizon solution is not a fixed point.
% Our final perturbation bounds are presented in the theorem below:
% \theoremCE*
% \begin{proof}
%   Proof for upper bounds~\eqref{eq:4} and~\eqref{eq:14} is given in
%   Appendix~\ref{sec:cert-equiv-contr-2}. The worst case upper bound is achieved 
%   when $\Bhat = 0$, i.e. when the model assumes that the system is not
%   controllable. An example of a scalar system that matches the upper
%   bound upto constants is given in Appendix~\ref{sec:cert-equiv-contr-1}.
% \end{proof}
% The above theorem gives us an upper bound for $\fCE_{t}$ for $t=0,
% \cdots, H-1$ in
% Lemma~\ref{lemma:ce} with $\fCE_H = 0$. The resulting upper bound on $\|\KOPT_t -
% \KCE_t\|$ from Lemma~\ref{lemma:ce} combined with
% Theorem~\ref{theorem:cost} gives us the cost 
% suboptimality bound for \MM{}. Notice that the
% bound on $\fCE_t$ grows quickly as $t$ decreases making $\fCE_{t+1}$
% in Lemma~\ref{lemma:ce} the dominant error term that affects the cost
% suboptimality of \MM{}.

% \section{Iterative Learning Control Results}
% \label{sec:ilc}

% In this section, we present results pertaining to iterative learning control and
% upper bound the norm differences $||\KOPT_{t} - \KILC_{t}||$. Our first lemma is
% similar to Lemma~\ref{lemma:ce}:
% \ilcLemma*
% \begin{proof}
%   Proof is given in Appendix~\ref{sec:iter-learn-contr}.
% \end{proof}
% Similar to Section~\ref{sec:ce}, we need to bound the crucial term
% $\fILC_{t+1}(\epsA, \epsB)$ to bound the norm difference $\|\KOPT_t -
% \KILC_t\|$ using Lemma~\ref{lemma:ilc}. We will present perturbation
% bounds for the \ILC{} recursion equation given in
% Section~\ref{sec:iter-learn-contr-1} in the finite horizon setting
% below:
% \ilcTheorem*
% \begin{proof}
%   Proof for upper bounds~\eqref{eq:6} and~\eqref{eq:15} is given in
%   Appendix~\ref{sec:iter-learn-contr}. Similar to \MM{}, the worst case
%   upper bound is achieved  
%   when $\Bhat = 0$, and an example of a scalar system that matches the upper
%   bound upto constants is given in Appendix~\ref{sec:iter-learn-contr-2}.
% \end{proof}
% The above theorem gives us a bound on $\fILC_t$ for $t=0, \cdots, H-1$
% in Lemma~\ref{lemma:ilc} with $\fILC_H = 0$. The resulting upper bound
% on $\|\KOPT_t - \KILC_t\|$ from Lemma~\ref{lemma:ilc} combined with
% Theorem~\ref{theorem:cost} gives us the cost suboptimality bound for
% iterative learning control. Similar to \MM{}, the dominant error term in
% Lemma~\ref{lemma:ilc} turns out to be $\fILC_{t+1}$ especially for
% smaller $t$ as the upper bound~\eqref{eq:15} grows quickly as $t$ decreases.

\section{Interpreting the Worst Case Bounds}
\label{sec:interpr-worst-case}

The recursive bounds
presented in~\eqref{eq:14} and~\eqref{eq:15} make it difficult to
compute a concise bound in Theorem~\ref{theorem:cost}. In this
section, we will explicitly compare the cost suboptimality
bounds for \MM{} and \ILC{} under different scenarios, where the bound can
be simplified.
%Since upper bounds for both \MM{} and \ILC{} are tight (up to constants), we can compare the worst case bounds to effectively compare performance.

\paragraph{Small Modeling Errors}
\label{sec:small-model-errors}

In the regime of small modeling errors $\epsA << 1$ and $\epsB << 1$,
we can ignore quadratic terms $\epsA^2$ and $\epsB^2$ in upper
bound for \MM{}~\eqref{eq:14} which results in an upper bound that
matches that of \ILC{}~\eqref{eq:15} upto a constant. This suggests that
when the modeling errors are small, both \ILC{} and \MM{} have almost the
same worst case performance, with \ILC{} having better performance over
\MM{} by a constant factor. Intuitively, this makes sense as the approximate model is
a very good approximation of the true dynamics, and despite using only the
model, \MM{} can synthesize a near-optimal controller.

\paragraph{Highly Damped Systems}
\label{sec:highly-stable-system}

The second scenario we consider is that of a system that is highly damped
which implies $\|A_t\| << 1$ for all $t=0, \cdots,
H-1$. In this regime, the upper bound for \ILC{}~\eqref{eq:15} goes down
to zero resulting in \ILC{} achieving near-optimal cost despite having
non-zero modeling errors $\epsA, \epsB$. The suboptimality in \ILC{}
(from Lemma~\ref{lemma:ilc}) only
arises from $\epsA, \epsB$ and not from $\fILC_{t+1}$ which is
$0$.
In contrast, the upper
bound for \MM{}~\eqref{eq:14} does not go down to zero and has terms that
depend on $\epsA^2$, which can be significant when $\epsA$ is not
small. Thus, for highly damped systems we have that the worst case
performance of \ILC{} can be significantly better than \MM{}, especially
when $\epsA$ is large. Intuitively, this can be understood by
observing that \ILC{} removes 
the effect of modeling errors by always performing rollouts using true
dynamics, 
while \MM{} errors are exacerbated by using the approximate
model for rollouts.
Interestingly, we also notice that the modeling
error $\epsB$ does not affect the cost-suboptimality in
upper bound for \MM{}~\eqref{eq:14}  when the system is highly damped.

\paragraph{Weakly Controlled Systems}
\label{sec:small-b_t}

For systems with small $\|B_t\| << 1$, i.e. where the control inputs do not
affect the dynamics of the system to a large extent, we can observe
that the upper bound for \ILC{}~\eqref{eq:15} reduces to a bound that
does not depend $\epsB$. In other words, any modeling error $\epsB$ in
estimating the $B_t$ matrices does not affect the upper
bound~\eqref{eq:15} for \ILC{}.
In constrast, the upper bound for \MM{}~\eqref{eq:14} reduces to an
expression that has terms that depend
on $\epsB^2$, which can become significant when $\epsB$ is
large. Thus, for systems with $\|B_t\| << 1$, \ILC{} is robust to any
modeling errors $\epsB$ in the $B_t$ matrices, whereas \MM{} degrades its
worst case performance with increasing $\epsB$.

\paragraph{Modeling Error only at the first time step}
\label{sec:modeling-error-only}

Consider a scenario where the model is inaccurate only at $t=0$,
i.e. $\|A_0 - \Ahat_0\|\leq\epsA$ and 
$\|B_0 - \Bhat_0\|\leq\epsB$, while $\Ahat_t = A_t$ and $\Bhat_t =
B_t$ for all $t=1, \cdots, H-1$. In this case, the upper
bounds~\eqref{eq:14} and~\eqref{eq:15} simplify greatly as $\|\POPT_t
- \PILC_t\| = \|\POPT_t - \PCE_t\| = 0$ for all $t=1, \cdots, H-1$,
and we only have upper bounds on $||\POPT_0 - \PCE_0||$ and $||\POPT_0
- \PILC_0||$ as given by Theorems~\ref{theorem:ce}
and~\ref{theorem:ilc} which when combined with
Theorem~\ref{theorem:cost} gives us the suboptimality bounds:
% \begin{align*}
%   \|\POPT_0 - \PCE_0\| &\leq
%                          \|A_{0}\|^{2}\|\POPT_{1}\|^{2}(2\|B_{0}\|\|R^{-1}\|\epsB
%                          + \|R^{-1}\|\epsB^{2}) +
%                          2\|A_{0}\|\|\POPT_{1}\|\epsA +
%                          \|\POPT_{1}\|\epsA^{2}  \\
%   \|\POPT_0 - \PILC_0\| & \leq \|A_{0}\|^{2}\|\POPT_{1}\|^{2}\|B_{0}\|\|R^{-1}\|\epsB
%     + \|A_{0}\|\|\POPT_{1}\|\epsA
% \end{align*}
% Setting $\epsilon_0$ as the terms on the right hand side of the above
% bounds in Lemmas~\ref{lemma:ce} and~\ref{lemma:ilc}, and combining
% them with Theorem~\ref{theorem:cost} gives us the suboptimality bounds
\begin{align}
  \Vhat_0^{\mathsf{MM}}(x_0) - V_0^\star(x_0) &\leq
  \mathcal{O}(1)d\Gamma^9\|x_0\|^2(\epsA + \epsA^2 + \epsB +
                                                \epsB^2)^2 \label{eq:ce-cost}\\
  \Vhat_0^{\mathsf{ILC}}(x_0) - V_0^\star(x_0) &\leq
  \mathcal{O}(1)d\Gamma^9\|x_0\|^2(\epsA + \epsB)^2\label{eq:ilc-cost}                                                
\end{align}
The above two cost suboptimality bounds highlight the differences
between \MM{} and \ILC{} in worst case performance. As described in
Section~\ref{sec:small-model-errors}, if $\epsA$ and $\epsB$ are
small, then \MM{} and \ILC{} worst case performances match up to constants as
we can ignore higher order terms.
However, in cases where modeling errors $\epsA$ and $\epsB$ are large
and higher order terms like $\epsA^2\epsB$, $\epsA^4$ etc. start
becoming significant, the worst case performance of \ILC{} tends to be
better than \MM{} as indicated by equations~\eqref{eq:ce-cost}
and~\eqref{eq:ilc-cost}. Furthermore, the conditions for stability
under synthesized control inputs, as
stated in Theorem~\ref{theorem:cost} (and in
Lemma~\ref{lemma:stability} in Appendix~\ref{sec:general-results-1},) is harder to satisfy for \MM{}
when compared to \ILC{}, especially when modeling errors are large.

\section{Empirical Results}
\label{sec:empirical-results}

In this section, we present three empirical
experiments: a linear dynamical system with an approximate model, a
nonlinear inverted pendulum system with misspecified mass, and a nonlinear planar
quadrotor system in the presence of wind. The aim of these
experiments is to show that under high modeling errors, \ILC{} is
more efficient than \MM{}, thus backing our theoretical findings.\footnote{The
  code for all experiments can be found at
\url{https://github.com/vvanirudh/ILC.jl}.}

\subsection{Linear Dynamical System with Approximate Model}
\label{sec:line-dynam-syst}
In this experiment, we use a linear dynamical system with states
$x \in \reals^{2}$ and control inputs $u \in \reals$.
The dynamics of the system
are specified by matrices:
%\begin{align*}
$A_{t} =
      \begin{bmatrix}
        1 & 1 \\
        -3 & 1
      \end{bmatrix},
  B_{t} =
          \begin{bmatrix}
            1 \\
            3
          \end{bmatrix}$.
%\end{align*}
The approximate model we use is constructed by perturbing the dynamics as
follows: $\Ahat_t = A_t + \epsilon\mathbb{I}$, $\Bhat_t = B_t +
\epsilon\begin{bmatrix}1 \\ 0\end{bmatrix}$
% \begin{align*}
%   \Ahat_{t} = A_{t} + \epsilon
%   \begin{bmatrix}
%     1 & 0\\
%     0 & 1
%   \end{bmatrix}
%   , \Bhat_{t} = B_{t} + \epsilon
%         \begin{bmatrix}
%           1 \\ 0
%         \end{bmatrix}
% \end{align*}
for any $\epsilon \geq 0$. Observe that this satisfies
$||\Ahat_{t} - A_{t}|| \leq \epsilon$ and $||\Bhat_{t} - B_{t}|| \leq \epsilon$.
We use a quadratic cost as specified in equation~\ref{eq:7} with matrices:
  $Q = Q_{f} = \mathbb{I}, R = 1$ (more details in Appendix~\ref{sec:line-dynam-syst}).
We can solve for the optimal controller $\KOPT$ in closed form using true
dynamics $A_{t}, B_{t}$ as specified in Section~\ref{sec:problem-setup}. We compare
\MM{} controller $\KCE$ 
and iterative learning controller $\KILC$ with approximate model
$\Ahat_{t}, \Bhat_{t}$ in \Cref{fig:lds} where we vary $\epsilon$ along
the X-axis (in log scale) and report the cost suboptimality gap
$V_{0}(x_{0}) - \VOPT_{0}(x_{0})$ on the Y-axis
(in log scale) where $V_{0}(x_{0})$ is the cost incurred by $\KCE$ or $\KILC$.
To ensure that
Assumption~\ref{assumption:psd} is not violated, the X-axis is capped at
$\epsilon = \frac{\ubar{\sigma}(B_{t}^{T}RB_{t})}{||B_{t}^{T}R||}$.
It is important to note that to generate the plot in \Cref{fig:lds} we
directly used the closed form solution for $\KILC$ (as described in
Section~\ref{sec:problem-setup}) and did not run a iterative learning control
algorithm. This was done to ensure that our results do not have any dependence
on how well the step size sequence was tuned for \ILC{}.
% We have also verified that
% running a simple ILC algorithm (LQR with forward pass using true dynamics and
% backward pass using model) with backtracking line search gives us the same
% results as the closed form solution $\KILC$ as long as $\epsilon \leq ||B||$.

% \begin{figure}[t]
%   \centering
%   % \subfigure{\label{fig:lds}\includegraphics[width=.32\linewidth]{figures/ilc/lds.pdf}}
%   % \subfigure{\label{fig:pendulum}\includegraphics[width=.32\linewidth]{figures/ilc/pendulum.pdf}}
%   % \subfigure{\label{fig:quadrotor}\includegraphics[width=.32\linewidth]{figures/ilc/quadrotor.pdf}}
%   \begin{subfigure}{.32\linewidth}
%     \includegraphics[width=\linewidth]{figures/ilc/lds.pdf}
%     \label{fig:lds}
%   \end{subfigure}
%   \begin{subfigure}{0.32\linewidth}
%     \includegraphics[width=\linewidth]{figures/ilc/pendulum.pdf}
%     \label{fig:pendulum}
%   \end{subfigure}
%   \begin{subfigure}{0.32\linewidth}
%     \includegraphics[width=\linewidth]{figures/ilc/quadrotor.pdf}
%     \label{fig:quadrotor}
%   \end{subfigure}
%   \caption{(a) Cost suboptimality gap with varying
%     modeling error $\epsilon$ for a
%     linear dynamical system. Note that both X-axis
%     and Y-axis are in log scale. (b) Cost suboptimality gap with
%     varying mass misspecification $\Delta m$ for a nonlinear 
%     inverted pendulum system. (c) Cost suboptimality gap for planar
%     quadrotor control with varying magnitude of wind $\eta$.}
%   \vspace{-0.3cm}
% \end{figure}
\begin{figure}[t]
  \centering
  \includegraphics[width=0.5\linewidth]{figures/ilc/lds.pdf}
  \caption{Cost suboptimality grap with varying modeling error
    $\epsilon$ for a linear dynamical system. Note that both X-axis
    and Y-axis are in log scale.}
  \label{fig:lds}
\end{figure}

We can observe that for small modeling errors $\epsilon < 10^{-1}$, \ILC{}
outperforms \MM{} by a constant factor (about $4\times 10^{2}$) as evidenced by the linear
trend in log scale. However in the regime of high modeling errors
$\epsilon > 10^{-1}$ we observe that the gap between \ILC{} and \MM{} is not a
constant factor anymore and grows very quickly as $\epsilon$ increases. This can
be explained by the fact that for high $\epsilon$, the
higher order terms in the gap between \ILC{} and \MM{} starts becoming significant
and results in poor performance for \MM{} when compared to \ILC{}. For large epsilons,
we also observe that the cost for \MM{} blows up to really big values as the system
is not stable anymore under $\KCE$ due to violation of the condition
in Theorem~\ref{theorem:cost} (and in
Lemma~~\ref{lemma:stability} in Appendix~\ref{sec:general-results-1}.) This experiment validates our claim from the
analysis that \ILC{} tends to perform better in terms of cost and is more robust
when modeling errors are high.

\subsection{Nonlinear Inverted Pendulum with Misspecified Mass}
\label{sec:invert-pend-with}

For the second experiment, we use the nonlinear dynamical system of an inverted
pendulum. The state space is specified by $x =
\begin{bmatrix}
  \theta &
  \dot{\theta}
\end{bmatrix} \in \reals^{2}
$ where $\theta$ is the angle between the pendulum and the vertical axis. The
control input is $u = \tau \in \reals$ specifying the torque $\tau$ to be
applied at the base of the pendulum. The dynamics of the system are
given by the ODE,
%\begin{align*}
$\ddot{\theta} = \frac{\bar{\tau}}{m\ell^{2}} - \frac{g\sin(\theta)}{\ell}$
%\end{align*}
where $m$ is the mass of the pendulum, $\ell$ is the length of the pendulum, $g$
is the acceleration due to gravity, and
$\bar{\tau} = \max(\tau_{\min}, \min(\tau_{\max}, \tau))$ is the clipped torque
based on torque limits (more details in
Appendix~\ref{sec:nonl-invert-pend}).
We use an approximate model of the dynamics where the mass of the pendulum is
perturbed as $\hat{m} = m + \Delta m$.
This results in dynamics that are nonlinearly perturbed from the true dynamics.
Since the dynamics are nonlinear, we cannot obtain
optimal controls, and \MM{} controls in closed form. Instead, we approximate these
controllers by running iLQR~\cite{li04} (both forward and backward pass) on the
true dynamics and the approximate
dynamics respectively for $200$ iterations. To obtain \ILC{} control inputs, we run
iLQR with forward pass (or rollouts)
using the true dynamics, and backward pass computed using the approximate
dynamics at each iteration.
We chose step sizes for all iLQR runs using
backtracking line search.

\begin{figure}[t]
  \centering
  \includegraphics[width=.5\linewidth]{figures/ilc/pendulum.pdf}
  \caption{Cost suboptimality gap of CE and ILC with varying $\Delta m$ for a
    nonlinear inverted pendulum system.}
  \label{fig:pendulum}
\end{figure}

\Cref{fig:pendulum} shows the cost suboptimality gap of \MM{} and \ILC{} as the
perturbation $\Delta m$ varies. Similar to our previous experiment, we observe
that for small modeling errors $\Delta m < 0.07$ both \ILC{} and \MM{} perform
similarly with \ILC{} outperforming slightly. But as $\Delta m$ grows, the cost of
\MM{} quickly grows saturating at a suboptimality gap around $57$.
% Unlike
% Figure~\ref{fig:lds} we do not observe cost of CE blowing up due to the
% backtracking line search that does not allow control input updates that
% deteriorate performance on the true dynamics.
In contrast, we observe that \ILC{}
is still able to compute near-optimal controls until
$\Delta m = 0.15$ showcasing the robustness of \ILC{} to higher modeling errors.
Beyond $\Delta m = 0.15$, \ILC{} performance also degrades significantly as the
approximate model is not representative of the true dynamics anymore.
Although our analysis in the previous sections was restricted to linear
dynamical systems, we notice a similar trend between \ILC{} and \MM{} in the presence
of nonlinear dynamics namely, in the regime of large modeling errors, \ILC{} tends to
perform better than \MM{}.

\subsection{Nonlinear Planar Quadrotor Control in Wind}
\label{sec:nonl-plan-quadr}

In our final experiment, we compare \MM{} and \ILC{} on a planar quadrotor control
task in the presence of wind. A similar setting was used
in~\cite{DBLP:conf/icml/AgarwalHMS21}. The quadrotor is controlled using two propellers
that provide upward
thrusts $(u_{1}, u_{2})$ and allows movement in the $3$D planar space
described as
$(p_{x}, p_{y}, \theta)$ where $p_{x}, p_{y}$ are X, Y positions, and
$\theta$ is the yaw of the quadrotor. The dynamics of the planar quadrotor
is specified
using a state vector $x \in \reals^{6}$, and control input $u \in
\reals^{2}$ (more details in Appendix~\ref{sec:nonl-plan-quadr-1}).
The quadrotor is flying in the presence of wind which is not captured in modeled
dynamics, but affects the true dynamics of the quadrotor as a dispersive force
field $(\eta p_{x}\mathbf{i} + \eta p_{y} \mathbf{j})$ resulting in
overall dynamics given by:
\begin{align*}
  \ddot{p}_{x} = \frac{1}{m}(u_{1} + u_{2})\sin(\theta) + \eta p_{x}~~~~~~~
  \ddot{p}_{y} = \frac{1}{m}(u_{1} + u_{2})\cos(\theta) - g + \eta p_{y}
\end{align*}
where $\eta \in \reals^{+}$ is a constant that captures magnitude of the wind
force field.

The objective of the task is to move the quadrotor from an initial
state $x_0$
to a final state $x_f$. Similar to
previous experiment, the dynamics are nonlinear and we cannot obtain optimal
controls and
\MM{} controls in closed form. Thus, we again approximate these by running
iLQR on true dynamics and approximate dynamics respectively. We obtain \ILC{}
control inputs again by using iLQR with forward pass using true dynamics and
backward pass using approximate dynamics.
% Since the wind disturbance is
% repetitive, we expect ILC to perform better than feedback-based control in CE.
For all iLQR runs, we choose step sizes by performing backtracking line search
and we initialize the control inputs as the hover controls.
\begin{figure}[t]
  \centering
  \includegraphics[width=.5\linewidth]{figures/ilc/quadrotor.pdf}
  \caption{Cost suboptimality gap of \MM{} and \ILC{} for planar quadrotor control
    with varying magnitude of wind $\eta$}
  \label{fig:quadrotor}
\end{figure}
\Cref{fig:quadrotor} compares \MM{} and \ILC{} for planar quadratic control with
varying magnitude of wind $\eta$. For small wind magnitudes, we
observe that both \MM{} and \ILC{} have good performance. As the wind
magnitude increases, \MM{} quickly diverges and the cost of synthesized
control inputs blows up quickly as the modeled dynamics are incapable
of capturing the dispersive force field exerted by the wind. \ILC{}, on
the other hand, manages to keep the cost from blowing up even at large
wind magnitudes. This reinforces our conclusion that \ILC{} is robust to
large modeling errors while \MM{} can quickly result in the cost blowing
up when the model is highly inaccurate.

\section{Discussion}
\label{sec:discussion}

Iterative Learning Control is known popularly as a higher
performance and more robust
alternative to optimal control with misspecified model when given access to an
inaccurate dynamical model. Our work takes the first steps in laying
the theoretical evidence for why \ILC{} has better performance and is
more robust when compared to
\MM{}. Unlike past work on analyzing the performance of \MM{} in the
infinite-horizon setting, we establish our suboptimality bounds in the
finite horizon setting in which \ILC{} is typically used. We use ricatti perturbation
proof techniques to prove suboptimality bounds in terms of the
modeling error $\epsA, \epsB$ for both \ILC{} and \MM{},
enabling us to compare them. This allows us to identify the reasons for the
performance and robustness of \ILC{}.

Our analysis shows that the gap between \ILC{} and \MM{} is in higher
order terms that can become
significant when the modeling error $\epsA, \epsB$ is large. This is
backed by our empirical experiments where we observe that as the
magnitude of modeling error increases, the performance gap between \ILC{}
and \MM{} grows rapidly as \MM{} is incapable of handling large modeling
errors and the resulting cost diverges. Furthermore, the conditions
needed for stability of the system under synthesized control inputs,
are easier to satisfy for \ILC{} when compared to \MM{}, especially in the
regime of large modeling errors. This explains the robustness of \ILC{}
over \MM{} for complex control tasks when given access to highly
inaccurate dynamical models. We also identify scenarios where the
norms $\|A_t\|$ and $\|B_t\|$ are small, where \ILC{} is
provably more efficient and more robust to modeling errors, when
compared to \MM{}.

While our current analysis is restricted to the linear quadratic
control setting, exploring similar suboptimality bounds in more
complex and possibly, nonlinear settings is an exciting direction for
future work. Recent work by~\cite{DBLP:conf/icml/SimchowitzF20} uses a
self-bounding ODE method to establish perturbation bounds that
sharpens previous bounds in the infinite horizon setting by only depending on natural
control-theoretic quantities and not relying on controllability
assumptions. It remains to be seen if we can rely on similar
techniques to sharpen the bounds presented in this work.
It would also be interesting to know whether fast rates for
control are possible for cost functions other than quadratic
costs. Finally, comparing iterative learning control and robust
control approaches such as \cite{DBLP:journals/focm/DeanMMRT20} would allow us to understand
the regime of modeling errors in which \ILC{} is more suitable than
robust control approaches, and vice versa.

%%% Local Variables:
%%% mode: latex
%%% TeX-master: "../main"
%%% End:

%\chapter{Proposed Work}
\label{cha:proposed-work}

\epigraph{\textit{More work is needed before planning with learned
    models can be effective. Environment models should be
    constructed judiciously with regard to both their states and
    dynamics with the goal of optimizing the planning process.}}{Rich
  Sutton and Andrew Barto (2018)}

The algorithms presented in this thesis, so
far, have not required any updates to the dynamics of the model. In
contrast, most existing methods in the literature, such
as~\cite{DBLP:journals/ml/KearnsS02, DBLP:journals/jmlr/BrafmanT02,
  DBLP:conf/atal/JongS07, 
  DBLP:journals/pami/DeisenrothFR15, DBLP:conf/icml/AbbeelQN06, 
  DBLP:conf/aaai/Jiang18, rastogi2018sample}, use experience
acquired from executions to update the dynamics of the model or learn
a model from scratch.
Chapters~\ref{CHA:CMAX} and~\ref{cha:lever-exper} have
argued that updating the dynamics of the model requires a large amount
of experience in large state spaces and can be at the expense of
completing the task. While this is generally true, there are major
advantages of updating the dynamics of the model, especially in
domains where it 
is feasible to do it online, as it allows the planner to compute
solutions that exploit the true dynamics and potentially result in
solutions with very low costs. Furthermore even in application domains
where we require a large amount of experience to update the model,
the improvement in task performance from planning on a more accurate
model can outweigh the executions wasted to learn true dynamics. For
example, there might be regions in the state space where updating the
dynamics of the model can be done efficiently while in other regions
we can resort to methods that update the behavior of planner such as
\cmax{} and \cmaxpp{}. This motivates a trade-off between both sets of
approaches and understanding this trade-off can result in intelligent
use of online experience to achieve efficient planning and
execution. The first part of proposed work studies this trade-off
and aims to create a unified framework combining the best of both
sets of approaches.

In this thesis, we are also interested in studying the task
performance one can obtain using inaccurate models without
updating the dynamics of the model online. So far, we have presented
algorithmic contributions in Chapters~\ref{CHA:CMAX}
and~\ref{cha:lever-exper} that provide evidence that we can achieve 
good empirical performance without requiring any updates to the
model. However, we have mostly
restricted ourselves to the discrete setting where one can perform
optimal planning procedures and establish asymptotic guarantees, but
cannot perform any fine-grained analysis. More specifically, we would
like to answer the following question, \textit{What task performance
  can we expect using an inaccurate model with a finite amount of
  experience?} Our guarantees, so far, do not tackle this
question as they are asymptotic in nature. Moving to a continuous
setting allows us to perform such fine-grained analysis and establish
bounds on the task performance as a function of the amount of
experience. The second part of proposed work aims to answer the above
question in the setting of continuous linearized systems with convex
costs and model uncertainty.

\section{Combining Model Learning with Updating Behavior of Planner}
\label{sec:updat-dynam-model}

In the first part of our proposed work, we aim to combine the advantages of
existing methods that update the dynamics of the model, and methods
presented in this thesis that update the behavior of the planner, like
\cmax{} and \cmaxpp{}. The goal is to create a unified framework where
the robot, during the course of its execution, deals with the
inaccuracy in the dynamical model and completes the task by
intelligently switching between: 
\begin{enumerate}
\item Learning the true dynamics and updating the model
\item Learning a model-free estimate of the value function for
  inaccurately modeled transitions
\item Biasing the planner away from inaccurately modeled transitions
  by inflating their cost
\end{enumerate}

In the next few sections, we outline the challenges in creating such a
unified framework and present initial ideas that are promising to
explore.

\subsection{Local Incremental Modeling}
\label{sec:local-modeling}

Foremost among them is data efficiency. In the online setting, the
robot acquires training data for learning the true dynamics through
executions. To learn a good approximation of the true dynamics, we
require a large number of samples especially in high-dimensional state
spaces. On the other hand, the robot's goal is to complete the task
quickly without wasting executions. Thus, the objective of learning
true dynamics can be in conflict with the objective of completing the
task resulting in a trade-off between using online executions to learn
true dynamics and completing the task.

Most existing works in the model-based
reinforcement learning literature use a global function
approximator to learn the true dynamics from scratch. However, as
evidenced by our experiments in Chapters~\ref{CHA:CMAX}
and~\ref{cha:lever-exper}, global function approximators such as
neural networks require a large amount of executions before they can
approximate the true dynamics well enough to complete the task. In
contrast, local function approximation methods such as the ones
described in Section~\ref{sec:local-funct-appr-1} operate well in the
regime of less data, approximate the dynamics with higher accuracies, and are more
amenable to incremental online implementations. Classical works such as
LWR~\cite{DBLP:journals/air/AtkesonMS97},
LWPR~\cite{DBLP:conf/icml/VijayakumarS00} and more recent works such
as LGR~\cite{DBLP:journals/corr/MeierHS14} and incremental
LGR~\cite{DBLP:conf/nips/MeierHS14} have shown empirical success in
learning inverse dynamics online for torque
control. Our experiments in Chapters~\ref{CHA:CMAX} and~\ref{cha:lever-exper}
using the model KNN baseline, which is a local modeling method, have
provided early indications to the efficacy of these local modeling
methods. The success of these local modeling methods motivates us to
use similar methods in learning forward dynamics online. 

\subsection{Task-Aware Model Learning}
\label{sec:task-driven-learning}

A general approach to estimate the true dynamics is to frame it as a
regression problem where the loss function is typically the L$2$ norm
prediction error. For example, given a state $s \in \statespace
\subset \mathbb{R}^d$
and action $a \in \actionspace$,
if the true successor is $s' \in \statespace$ and the function
approximator class is parameterized by $\theta$ then the objective is
to minimize the following loss function
\begin{equation}
  \label{eq:16}
  \mathcal{L}(\theta) = \|s' - \hat{f}(s, a; \theta)\|_2^2
\end{equation}
where $\hat{f}$ is the function approximator. In stochastic dynamics
setting, a popular way to estimate the stochastic transition matrix is
to use maximum likelihood estimation (MLE.)

The common practice in model-based reinforcement learning is to
seperate the task of learning the dynamical model from its use in
planning. One can argue that the goal of learning the model is not in
optimizing prediction error as given in \eqref{eq:16}, but to learn
models that are directly useful for planning. It might be the case
that some aspects of the environment dynamics are irrelevant to find a
good plan. Thus, it is desirable to have a model learning procedure
that takes the planning problem into account and is more
task-aware.

Furthermore in real world domains it is often the case
that the function approximator class used is unable to capture the
true dynamics, either due to a small function class or the true
dynamics being extremely complex. In such cases, even with unlimited
amount of experience and computation, there is no guarantee that the
model with the least prediction error leads to a plan that completes
the task successfully. This motivates moving away from using
prediction error as the model selection metric to using a
task-specific metric that selects a model which results in plans that
complete the task.

Recent work such as~\cite{DBLP:conf/aistats/FarahmandBN17,
  Farahmand2018} have explored designing loss functions for model
learning that result in models which are useful for their subsequent
use in value-based planning. This line of work has been taken further
by~\cite{grimm2020value} establishing the value equivalence principle
which states that two models are equivalent with respect to a set of
function class if they yield the same Bellman updates. They illustrate
that by leveraging the value equivalence principle one may find
simpler models without compromising task performance, saving both
computation and memory. Another very relevant
work~\cite{DBLP:conf/icra/JosephGRHR13} presents a simple algorithm
that selects the model which achieves the highest expected reward, and
not the lowest prediction error. The algorithm guarantees that the
highest performing model from the function approximator class can be
found given unlimited data and computation.

In our online no-reset setting, it is extremely important to use the
limited data available to optimize for task completion rather than
minimizing prediction error. A naive way to achieve this is to
consider a weighted prediction error loss function that weights each
transition according to the usefulness of it for future replanning
queries. This enables us to build better models with less data and
without compromising task performance.

\subsection{Guaranteeting Task Completeness}
\label{sec:guar-task-compl}

During the process of updating the dynamics of the model, the
resulting model can have approximation errors that result in violating
assumptions required to guarantee task completeness. Most of the prior
works present such guarantees under an idealistic assumption of no
approximation errors which cannot be realized in practice. In our
proposed algorithm, we would like to retain the task completeness
guarantees despite the presence of approximation errors in the updated
dynamical model.

\subsection{Switching Between Model Learning and Updating
  Planner Behavior}
\label{sec:switch-betw-cmax}

To achieve our goal of combining advantages of methods that update the
dynamical mdoel online and methods that update the behavior of the
planner, we need an intelligent strategy to switch between these
methods during execution. An example of a switching strategy has
been presented in Section~\ref{sec:adaptive} where we presented an
algorithm that switches between \cmax{} and \cmaxpp{}. The goal is to
design a similar algorithm that switches between updating the model,
\cmax{} and \cmaxpp{}, with the objective of optimizing task
performance. This involves explicitly reasoning about when it is more
useful to simply bias the planner away from inaccurately modeled
regions and find alternative paths, versus when it is useful to learn
the true dynamics and update the model or learn the true value
function using a model-free update.

\section{Robust Control in Continuous Linearized Systems with Model
  Uncertainty}
\label{sec:robust-contr-cont}

In the second part of our proposed work, we study a simple
continuous control setting that is easy to analyze and provides
insights into the performance one can expect from using an approximate
model with finite amount of experience. More specifically, we
consider the setting of continuous linearized systems with model
uncertainty. The dynamics of this system are given as follows,
\begin{equation}
  \label{eq:17}
  x_{t+1} = A_tx_t + B_tu_t
\end{equation}
where $x_t \in \statespace \subset \reals^d$ is the state of the
system at time $t$, and $u_t \in \actionspace \subset \reals^p$ is the
control input at time $t$. The matrices $A_t$ and $B_t$ represents the
linearized 
dynamics of the system at time $t$. This setting is quite general as
we can linearize any non-linear dynamical system at each time step,
and approximate it as a continuous linearized system as given in
\eqref{eq:17}.


For the purposes of this work, we will deal with the class of
non-stationary linear controllers parameterized by $K = (K_1, \cdots,
K_{T-1})$ such that the control input at time $t$ is given by $u_t =
K_tx_t$. The objective is to
minimize the sum of convex costs $J(K) = \sum_{t=1}^{T-1} c_t(x_t, u_t) +
c_T(x_T)$ over a
trajectory of length $T$ starting from a fixed state $x_1$ subject to
the dynamics given in
\eqref{eq:17}. Thus, the objective of the controller is to optimize
\begin{equation}
  \label{eq:18}
  \begin{aligned}
    \min_{K_1, \cdots, K_{T-1}} \quad & J(K) \\
    \textrm{Subject to} \quad & x_{t+1} = A_tx_t + B_tu_t
  \end{aligned}
\end{equation}

In our proposed work, the matrices $A_t, B_t$ are not known to the
learner. Instead, the learner has access to nominal dynamics $\Ahat_t,
\Bhat_t$ which are approximations of the true dynamics. Specifically,
we assume that $\|A_t - \Ahat_t\|_2 = \|\Delta_t^A\|_2 \leq \epsilon_t^A$ and $\|B_t -
\Bhat_t\|_2  = \|\Delta_t^B\|_2 \leq \epsilon_t^B$. Note that the learner has to
synthesize a controller that performs well in the true dynamics given
by \eqref{eq:17} while only having access to approximate nominal
dynamics given by $\Ahat_t, \Bhat_t$. This is very similar to the
setting that has been considered in this thesis so far, but with an
important difference. The difference is that we are dealing
with a continuous state and action space. While this makes the
controller synthesis challenging, it is amenable to fine-grained
analysis on the amount of experience needed to achieve a specific
performance.

Consider the optimal robust controller $K^*$ that
is obtained by solving the following optimization problem
\begin{equation}
  \label{eq:19}
  \begin{aligned}
    \min_{K_1, \cdots, K_{T-1}} \max_{\substack{\|\Delta_t^A\|_2 \leq
      \epsilon_t^A \\ \|\Delta_t^B\|_2 \leq \epsilon_t^B}}\quad & J(K) \\
    \textrm{Subject to} \quad & x_{t+1} = (\Ahat_t + \Delta_t^A)x_t +
    (\Bhat_t + \Delta_t^B)u_t
  \end{aligned}
\end{equation}
We use $K^*$ as the controller against which we benchmark the
performance of our learner. This results in bounds on the
learner's performance in terms of how it compares against $K^*$
performance. A similar notion of robust controller was introduced
in~\cite{DBLP:journals/focm/DeanMMRT20} in the infinite-horizon linear
quadratic regulator problem. Similar notions of min-max control have
been explored in
classical works such as $H_\infty$ control~\cite{10.5555/225507}.

\subsection{Iterative Learning Control}
\label{sec:iter-learn-contr}

We formulate the problem as an iterative learning problem where the
learner updates its controller across rollouts performed in the real
system~\cite{DBLP:journals/jfr/MooreDB92}. Before the start of
each rollout $i$, the learner computes a non-stationary linear
controller $K^{(i)} = (K_1^{(i)}, \cdots, K_{T-1}^{(i)})$ that is used
to control the system during the rollout. Note that while the learner
only has access to nominal dynamics and experience from previous
rollouts to compute the controller $K^{(i)}$, it is always evaluated
on the real system with dynamics given by \eqref{eq:17}. Thus, after
each rollout $i$, we obtain a trajectory $x_1^{(i)}, u_1^{(i)},
\cdots, x_{T-1}^{(i)}, u_{T-1}^{(i)}, x_T^{(i)}$ and the learner
incurs cost given by $J(K^{(i)})$. It is important to observe that
unlike feedback controllers, the learner does not update the
controller during the rollout and instead, only updates it after each
rollout. 

We measure the performance of the learner using the notion of regret
with respect to $K^*$: the difference between the aggregate cost
incurred by the learner and 
that of the optimal robust controller $K^*$ over $N$ rollouts
\begin{equation}
  \label{eq:20}
  \mathsf{Regret} = \sum_{i=1}^N J(K^{(i)}) - \sum_{i=1}^N J(K^*)
\end{equation}

Our goal in the proposed work is to design a controller synthesis
procedure for the learner which at rollout $i$, only uses the nominal
dynamics $\Ahat_t, \Bhat_t$ and the experience from past
rollouts. Furthermore, we would also like to bound the regret of the
learner w.r.t $K^*$ in terms of the number of rollouts $N$. Such a bound
provides insights on the performance of the controller synthesized
using an
inaccurate model and finite amount of
experience from $N$ rollouts.

While the algorithms developed in this part of proposed work are quite
different from the algorithms presented in the rest of the thesis, we
believe that many of the ideas are similar. Similar to \cmax{} and
\cmaxpp{}, we are not interested in learning the true dynamics $A_t,
B_t$ but instead we are interested in optimizing the performance of
the synthesized controller while being robust to inaccuracies in the model.


\section{Schedule of Proposed Work}
\label{sec:sched-prop-work}

\begin{itemize}
\item \textbf{Spring 2021}
  \begin{itemize}
  \item Finish work on robust control in continuous linearized systems
    with model uncertainty described in Section~\ref{sec:robust-contr-cont}
  \item Design and implement a task-aware incremental model learning
    algorithm as described in Sections~\ref{sec:local-modeling} and~\ref{sec:task-driven-learning}
  \end{itemize}
\item \textbf{Summer 2021}
  \begin{itemize}
  \item Combine the incremental model learning algorithm with \cmax{}
    and \cmaxpp{} to create the unified framework described in
    Section~\ref{sec:updat-dynam-model}
  \item Demonstrate the resulting framework on simulated and real
    robot experiments
  \end{itemize}
\item \textbf{Fall 2021}
  \begin{itemize}
  \item Write and defend thesis
  \end{itemize}
\end{itemize}

% \begin{itemize}
% \item \textbf{Spring 2021}
%   \begin{itemize}
%   \item Formulate a task-aware loss function for training local residual
%     dynamical models
%   \item Design an online incremental algorithm to update dynamics of
%     the model and complete the task
%   \end{itemize}
% \item \textbf{Summer 2021}
%   \begin{itemize}
%   \item Create a unified framework where the robot, during execution,
%     switches between updating dynamics, learning model-free Q-value
%     estimates, and biasing planner away from inaccurately modeled
%     transitions
%   \item Demonstrate the framework on simulated and real robot
%     applications where we have access to an inaccurate dynamical model
%   \end{itemize}
% \item \textbf{Fall 2021}
%   \begin{itemize}
%   \item Write and defend thesis
%   \end{itemize}
% \end{itemize}

%%% Local Variables:
%%% mode: latex
%%% TeX-master: "../main"
%%% End:


\chapter{Task-Aware Model Learning}
\label{cha:task-aware-model}

%%% Local Variables:
%%% mode: latex
%%% TeX-master: "../main"
%%% End:


\chapter{Future Work and Conclusion}
\label{cha:future-work-concl}

This chapter concludes the thesis by laying out directions for future
work. The author has made some progress on some of these directions
while for others, pointers are given to related work so that the
reader can get started.

\section{A Unified Framework for Planning and Execution using
  Inaccurate Models}
\label{sec:unified-framework}

This thesis has presented two novel algorithms \cmax{} and \cmaxpp{} that update the
behavior of the planner, rather than updating the dynamics of the
model, to allow robots to complete the task despite using an
inaccurate model. \cmax{} enables the planner to stick to the
state-action space regions where the model is accurate and biases it
away from the inaccurately modeled regions. \cmaxpp{}, on the other
hand, learns model-free value estimates for inaccurately modeled
transitions and integrates them into a model-based planning procedure
with the inaccurate model. Both approaches require the inaccurate
model to be optimistic and have task-completeness guarantees. While
our experiments have shown that they work very well empirically, there
are domains where designing ``good'' optimistic models is
difficult. By good, we mean a non-trivial optimistic model (a trivial
optimistic model would be one that predicts any transition executed by
robot would complete the task) that is useful in most state-action
space regions. In such domains, updating the dynamics of the model
might be more efficient even in cases where the true dynamics does not
lie in the model class considered. This thesis has taken preliminary
steps in designing such an efficient model learning algorithm in
Chapter~\ref{CHA:TAML} where we presented \taml{} that performs better
than \cmax{} in domains where we can update the dynamics of the model
through low-dimensional parameterizations.

An important future direction would be to build upon \taml{} to design
efficient model learning algorithms that directly optimize task
performance and enable choosing models that are useful for planning
rather than prediction. Initial work in this direction has been done
in~\cite{grimm2020value, DBLP:journals/corr/abs-2106-10316,
  DBLP:conf/icml/AyoubJSWY20, DBLP:journals/corr/abs-2106-14080,
  DBLP:journals/corr/abs-2106-03273,
  DBLP:journals/corr/abs-2110-02758}.
Given such algorithms, the author envisons a
unified framework for planning and execution where the robot, at every
time step, chooses to either update the dynamics of the model from
executions (using algorithms like \taml{}) or updates the behavior of
the planner (using algorithms like \cmax{} and \cmaxpp{}.) This allows
us to combine the advantages of both family of algorithms while
retaining task completeness guarantees. One viable way of implementing
this would be by using the multi-heuristic A* (MHA*)
framework~\cite{DBLP:journals/ijrr/AineSNHL16} where we treat each
algorithm that the robot can use as a heuristic that it can follow to
reach the goal. This would involve maintaining a different set of
cost-to-go estimates for \taml{}, \cmax{} and \cmaxpp{}. Intuitively,
we prefer \cmax{} as it does not waste executions learning dynamics or
learning model-free value estimates and quickly finds an alternative
path. To encode this preference, we can design an anytime algorithm
similar to \acmaxpp{} (in Section~\ref{sec:adaptive}) where if the
cost-to-go following \cmax{} is not too worse compared to that of
\taml{} and \cmaxpp{}, we follow \cmax{}. Else, if the cost-to-go
following \cmaxpp{} is not too far from that of \taml{}, then we
follow \cmaxpp{}. If neither of those are true, then we follow
\taml{}. This encodes the preference that avoiding inaccurately
modeled transitions is easier to learn than model-free value
estimates which is easier to learn than the model dynamics. The goal is to create a unified
framework where the robot, during the course of its execution,
intelligently switches between (a) learning the true dynamics, (b)
learning a model-free value estimate, or (c)
biasing the planner away from an inaccurately modeled
transition to guarantee task completeness while reducing
the amount of real-world experience required.


\section{Online Model Learning with Misspecified Model Classes}
\label{sec:online-model-learn}

While \taml{} was a first step in the direction of online model
learning with misspecified model classes where we directly optimize
task performance rather than prediction error, the author believes
there is still a long way to go in this direction. Our main motivation
for this comes from the simulation lemma, which was first introduced
in~\cite{DBLP:journals/ml/KearnsS02}, and can be reformulated in the
undiscounted deterministic dynamics setting as follows:
\begin{lemma}[Undiscounted Deterministic Dynamics Simulation Lemma]
  Let $M, M'$ be two Markov Decision Processes with the same cost
  function. If we have a fixed 
  start state $s_0$, a deterministic policy $\pi:\statespace
  \rightarrow \actionspace$, and $M, M'$ have deterministic dynamics
  $f, f': \statespace \times \actionspace \rightarrow
  \statespace$. Then we have,
  \begin{align}
    \label{eq:23}
    J_M(\pi) &= J_{M'}(\pi) + \sum_{t=0}^\infty c(s_t^M, \pi(s_t^M)) +
               V_{M'}^\pi(s_{t+1}^M) - V_{M'}^\pi(s_t^M) \\
    &= J_{M'}(\pi) + \sum_{t=0}^\infty V_{M'}^\pi(s_{t+1}^M) -
      V^\pi_{M'}(f'(s_t^M, \pi(s_t^M)))
  \end{align}
  where $s_0^M = s_0$ and $s_t^M = f(s_{t-1}^M, \pi(s_{t-1}^M))$.
\end{lemma}

In the case where $M$ is the real world, and $M'$ is any dynamical
model that we consider, the above lemma states that the performance of
any policy $\pi$ in the real world $M$ is equal to the sum of the performance of the
policy in the model $M'$ and the \textit{model advantages} at each time step $V_{M'}^\pi(s_{t+1}^M) -
      V^\pi_{M'}(f'(s_t^M, \pi(s_t^M)))$. Thus, in order to find a
      model $M'$ that captures the performance of a policy as the same
      as that of its performance in the real world, we need to
      minimize model advantages. However, most existing works that
      perform maximum likelihood learning do not consider this
      objective function~\cite{DBLP:journals/arc/Ljung10,
        DBLP:conf/icml/AbbeelN05, DBLP:conf/icml/RossB12, 
  DBLP:journals/corr/abs-1907-02057} and instead use a prediction
error loss. For example, \cite{DBLP:conf/icml/RossB12} present a simple iterative
approach for agnostic system identification with strong guarantees
that do not scale with the size of the MDP when given access to a good
exploration distribution. The approach is very simple to implement and
iterates between collecting new data about the real world $M$ by executing
a good policy under the current model $M'$ as well as by sampling from
the exploration distribution, and updating the model with the new
data. The model is updated by minimizing negative log likelihood of
the data under the model.


To understand why prediction error or maximum likelihood objective
makes sense, let us take a
closer look at the model advantages:
\begin{align*}
  V_{M'}^\pi(s_{t+1}^M) - V^\pi_{M'}(f'(s_t^M, \pi(s_t^M))) &\leq
                                                              L\|s_{t+1}^M - f'(s_t^M, \pi(s_t^M))\| \\
  &\leq L\|f(s_t^M, \pi(s_t^M)) - f'(s_t^M, \pi(s_t^M))\|
\end{align*}
where we assumed that the value function of policy $\pi$ in the model
$M'$ is $L$-lipschitz (any bounded function on a bounded domain is
lipschitz.) Thus, instead of optimizing the model advantages one can
optimize the prediction error which is an upper bound on the
model advantage~\cite{DBLP:conf/icml/RossB12}. However, this can be a
very weak upper bound resulting in a high sample complexity
requirement.

There are two ways to tackle this: 1) directly optimize $J_{M}(\pi)$,
or 2) optimize the model advantages instead of prediction error. RBMS
and \taml{} take the first way by directly optimizing $J_M(\pi)$,
i.e. the performance of the policy in the real world $M$. The policy
class is parameterized by the model class (and the application of
planner $P$) and the performance of policy in $M$ is computed through
an off-policy evaluation procedure. While this works for simple
domains, off-policy evaluation is not always reliable as the data is
collected under a different policy than the policy that is being
evaluated resulting in a high variance estimate.

Another option is to take the second
way. \cite{DBLP:conf/aistats/VoloshinJY21} take this approach in the
offline setting where given an offline collected dataset $\mathcal{D}$
they find a model $M'$ that minimizes the model advantages as
evaluated on $\mathcal{D}$. Once they find the best model in the model
class, they use it for planning to obtain the policy that is then used
for execution in the real world $M$. While this works well in offline
settings, the online version would face similar difficulties as RBMS
where the online collected dataset $\mathcal{D}$ might not have good
coverage and can be highly correlated. Thus, there is a need for an
online model learning algorithm that optimizes model advantages.

The author envisions an online iterative approach similar
to~\cite{DBLP:conf/icml/RossB12} where the model is updated with the
new data by minimizing model advantages rather than minimizing
negative log-likelihood. The advantage of such an approach is evident
in domains where there are no models that capture the true dynamics
exactly in the model class (a.k.a misspecified) but there are several
models that are useful for planning. Using model advantages, instead
of prediction error, allows us to distinguish models that are useful
for planning from models that are good at capturing true dynamics.

\section{Extending \cmax{} and \cmaxpp{} to Stochastic Dynamics}
\label{sec:extend-cmax-cmaxpp}



%%% Local Variables:
%%% mode: latex
%%% TeX-master: "../main"
%%% End:


\chapter{Appendix}
\label{cha:appendix}

This chapter contains all the missing details from the previous chapters,
especially the proofs, experiment descriptions, and other relevant information.
This is done to ensure that the thesis is concise for all readers, and for
readers who are interested in low-level details they can refer to this chapter
as needed.

\section{Appendix for Chapter~\ref{CHA:ARS}}
\label{sec:append-chapt-ars}

\subsection{Proof of Theorem~\ref{thm:online_linear_regression}}
\label{sec:proofs_bandit}
\begin{proof}[Proof of Theorem~\ref{thm:online_linear_regression}]
%Result in Eq~\ref{eq:ogd} is directly from \cite{Zinkevich2003_ICML} with the fact that $\|w\|_2\leq\mathcal{W}$ and $\|\nabla_{w}\ell_t(w)\|_2\leq C\mathcal{X}$ to any $w$ and $t$.

To prove Eq.~\ref{eq:random_para} for Alg.~\ref{alg:random_search_OLR}, we use the proof techniques from \cite{flaxman2005online}. The proof is more simpler than the one in \cite{flaxman2005online} as we do not have to deal with shrinking and reshaping the predictor set ${\Theta}$.

Denote $u\sim \mathbb{B}_b$ as uniformly sampling $u$ from a $b$-dim unit ball, $u\sim\mathbb{S}_b$ as uniformly sampling $u$ from the $b$-dim unit sphere, and $\delta \in (0,1)$. Consider the loss function $\hat{c}_i(w_i) = \mathbb{E}_{v\sim \mathbb{B}_b}[c_i(\theta_i + \delta v)]$, which is a smoothed version of $c_i(w_i)$. It is shown in \cite{flaxman2005online} that the gradient of $\hat{c}_i$ with respect to $\theta$ is:
\begin{align*}
   &\nabla_{\theta}\hat{c}_i(\theta)|_{\theta = \theta_i} \\
   &= \frac{b}{\delta} \mathbb{E}_{u\sim\mathbb{S}_b}[c_i(\theta_i +\delta u)u]\\
   &= \frac{b}{\delta}\mathbb{E}_{u\sim \mathbb{S}_b}[((\theta_i +\delta u)^T s_i - a_i)^2 u].
\end{align*} Hence, the descent direction we take in Alg.~\ref{alg:random_search_OLR} is actually an unbiased estimate of $\nabla_{\theta}\hat{c}_i(\theta)|_{\theta=\theta_i}$. So Alg.~\ref{alg:random_search_OLR} can be considered as running OGD with an unbiased estimate of gradient on the sequence of loss $\hat{c}_i(\theta_i)$. It is not hard to show that for an unbiased estimate of $\nabla_{\theta}\hat{c}_i(\theta)|_{\theta=\theta_i}$ = $\frac{b}{\delta} ((\theta_i + \delta u)^T s_i - a_i)^2 u$, the norm is bounded as $b(C^2 + C_{s}^2)/\delta$. Now we can directly applying Lemma 3.1 from \cite{flaxman2005online}, to get:
\begin{align}
\label{eq:regret_on_surrogate}
   \mathbb{E}\left[\sum_{i=1}^T \hat{c}_i(\theta_i)\right] - \min_{\theta^\star\in{\Theta}}\sum_{i=1}^T \hat{c}_i(\theta^\star) \leq \frac{C_{\theta}b(C^2+C_{s}^2)}{\delta}\sqrt{T}.
\end{align} We can bound the difference between $\hat{c}_i(\theta)$ and ${c}_i(\theta)$ using the Lipschitiz continuous property of $c_i$:
\begin{align}
|\hat{c}_i(\theta) - c_i(\theta) | & = |\mathbb{E}_{v\sim \mathbb{B}_b}[c_i(\theta+\delta v) - c_i(\theta)]| \nonumber\\
&\leq \mathbb{E}_{v\sim \mathbb{B}_b}[|c_i(\theta+\delta v) - c_i(\theta)|] \leq L\delta.
\end{align} Substitute the above inequality back to Eq.~\ref{eq:regret_on_surrogate}, rearrange terms, we get:
\begin{align}
&\mathbb{E}\left[ \sum_{i=1}^T c_i(\theta_i)  \right]  - \min_{\theta^\star\in{\Theta}} \sum_{i=1}^T c_i(w^\star)\leq \frac{C_{\theta}b(C^2+C_{s}^2)}{\delta}\sqrt{T} + 2LT\delta.
\end{align} By setting $\delta = T^{-0.25}\sqrt{\frac{C_{\theta}b(C^2+C_{s}^2)}{2L}}$, we get:
\begin{align*}
   &\mathbb{E}\left[ \sum_{i=1}^T c_i(\theta_i)  \right]  - \min_{w^\star\in{\Theta}} \sum_{i=1}^T c_i(w^\star) \leq \sqrt{C_{\theta}b(C^2+C_{s}^2)L} T^{3/4}.
\end{align*}

To prove Eq.~\ref{eq:random_action} for Alg.~\ref{alg:random_search_action}, we follow the similar strategy in the proof of Alg.~\ref{alg:random_search_OLR}.

Denote $\epsilon \sim [-1,1]$ as uniformly sampling $\epsilon$ from the interval $[-1,1]$, $e\sim \{-1,1\}$ as uniformly sampling $e$ from the set containing $-1$ and $1$. Consider the loss function $\tilde{c}_i(\theta) = \mathbb{E}_{\epsilon\sim [-1,1]}[(\theta^T s_i + \delta \epsilon - a_i)^2]$. One can show that the gradient of $\tilde{c}_i(\theta)$ with respect to $\theta$ is:
\begin{align}
    \nabla_{\theta}\tilde{c}_i(\theta) = \frac{1}{\delta}\mathbb{E}_{e\sim \{-1,1\}}[e(\theta^{\top} s_i + \delta e - a_i)^2 s_i].
\end{align} As we can see that the descent direction we take in Alg.~\ref{alg:random_search_action} is actually an unbiased estimate of $\nabla_{\theta}\tilde{c}_i(\theta)|_{\theta=\theta_i}$. Hence Alg.~\ref{alg:random_search_action} can be considered as running OGD with unbiased estimates of gradients on the sequence of loss functions $\tilde{c}_i(\theta)$. For an unbiased estimate of the gradient, $\frac{1}{\delta} e(\theta_i^{\top} s_i +\delta e - a_i)^2 s_i$, its norm is bounded as $(C^2 + 1)C_{s}/\delta$. Note that different from Alg.~\ref{alg:random_search_OLR}, here the maximum norm of the unbiased gradient \emph{is independent of feature dimension $b$}. Now we apply Lemma 3.1 from \cite{flaxman2005online} on $\tilde{c}_i$, to get:
\begin{align}
\label{eq:tilde_random_action}
    \mathbb{E}\left[ \sum_{i=1}^T \tilde{c}_i(\theta_i)\right] - \min_{\theta^\star\in{\Theta}}\sum_{i=1}^T \tilde{c}_i(\theta^*) \leq \frac{C_{\theta}(C^2 + 1)C_{s}}{\delta}\sqrt{T}.
\end{align}
Again we can bound the difference between $\tilde{c}_i(\theta)$ and $c_i(\theta)$ for any $\theta$ using the fact that $(\hat{a}_i - a_i)^2$ is Lipschitz continuous with respect to prediction $\hat{a}_i$ with Lipschitz constant $C$:
\begin{align}
    |\tilde{c}_i(\theta) - c_i(\theta)| &= |\mathbb{E}_{\epsilon\sim [-1,1]} [(\theta^{\top} s_i + \delta\epsilon - a_i)^2 - (\theta^{\top} s_i - a_i)^2]|  \nonumber\\
    &\leq \mathbb{E}_{\epsilon\sim [-1,-1]}[C\delta |\epsilon|] \leq C\delta.
\end{align} Substitute the above inequality back to Eq.~\ref{eq:tilde_random_action}, rearrange terms:
\begin{align*}
    &\mathbb{E}\left[\sum_{i=1}^T \tilde{c}_i(\theta_i)\right] - \min_{\theta^\star\in{\Theta}}\sum_{i=1}^T \tilde{c}_i(\theta^*)\leq \frac{C_{\theta}(C^2+1)C_{s}}{\delta}\sqrt{T} + 2C\delta T.
\end{align*}
Set $\delta = T^{-0.25}\sqrt{\frac{C_{\theta}(C^2+1)C_{s}}{2C}}$, we get:
\begin{align*}
  &\mathbb{E}\left[\sum_{i=1}^T \tilde{c}_i(\theta_i)\right] - \min_{\theta^*\in{\Theta}}\sum_{i=1}^T \tilde{c}_i(\theta^*)\leq \sqrt{C_{\theta}(C^2+1)C_{s}C}T^{3/4}.
\end{align*}
\end{proof}




\subsection{Proof of Theorem~\ref{theorem:parameter-convergence}}
\label{sec:proofs_RL}

We first present some useful lemmas below.


Consider the smoothed objective given by $\hat{J}(\theta) =
\mathbb{E}_{v \sim \mathbb{B}_d}[J(\theta + \delta v)]$ where
$\mathbb{B}_d$ is the unit ball in $d$ dimensions and $\delta$ is a
positive constant. Using the assumptions stated in Section
\ref{sec:assumptions_parameter}, we obtain the following useful lemma:
\begin{lemma}
  \label{lemma:grad-diff-parameter}
  If the objective $J(\theta)$ satisfies the assumptions in Section
  \ref{sec:assumptions_parameter} and the smoothed objective
  $\hat{J}(\theta)$
  is as given above,
  %is given by $\hat{J}(\theta) = \mathbb{E}_{v \sim
  %  \mathbb{B}_d}[J(\theta + \delta v)]$ where $\delta > 0$ and
  %  $\mathbb{B}_d$ is the unit ball in $d$ dimensions
  then we have that
  \begin{enumerate}
  \item $\hat{J}(\theta)$ is also $G$-Lipschitz and $L$-smooth
  \item For all $\theta \in \mathbb{R}^d$, $\|\nabla_\theta J(\theta)
    - \nabla_\theta \hat{J}(\theta)\| \leq L\delta$
  \end{enumerate}
\end{lemma}



\begin{proof}[Proof of Lemma \ref{lemma:grad-diff-parameter}]
  Consider for any $\theta_1, \theta_2 \in \mathbb{R}^d$,
\begin{align*}
    |\hat{J}(\theta_1) - \hat{J}(\theta_2)| &= |\mathbb{E}_{v \sim \mathbb{B}_d}[J(\theta_1+\delta v) - J(\theta_2 + \delta v)]| \nonumber \\
    &\leq \mathbb{E}_{v \sim \mathbb{B}_d}[|J(\theta_1+\delta v) - J(\theta_2 + \delta v)|] \nonumber \\
    &\leq \mathbb{E}_{v \sim \mathbb{B}_d}[G\|\theta_1 - \theta_2\|] \nonumber \\
    &= G\|\theta_1 - \theta_2\|
\end{align*}
The above inequalities are due to the fact that expectation of absolute value is greater than absolute value of expectation, and the $G$-lipschitz assumption on $J(\theta)$. Thus, the smoothened loss function $\hat{J}(\theta)$ is also $G$-lipschitz. Similarly consider,
\begin{align*}
  \|\nabla_\theta\hat{J}&(\theta_1) - \nabla_\theta\hat{J}(\theta_2)\| \\
  &= \|\nabla_\theta \mathbb{E}_{v \sim \mathbb{B}_d}[J(\theta_1 + \delta v)] - \nabla_\theta \mathbb{E}_{v \sim \mathbb{B}_d}[J(\theta_2+\delta v)]\| \nonumber \\
    &= \|\mathbb{E}_{v \sim \mathbb{B}_d}[\nabla_\theta J(\theta_1+\delta v)
      - \nabla_\theta J(\theta_2 + \delta v)]\| \nonumber \\
    &\leq \mathbb{E}_{v \sim \mathbb{B}_d}[\|\nabla_\theta J(\theta_1+\delta
      v) - \nabla_\theta J(\theta_2 + \delta v)\|] \nonumber \\
    &\leq \mathbb{E}_{v \sim \mathbb{B}_d}[L\|\theta_1 - \theta_2\|] \nonumber \\
    &= L\|\theta_1 - \theta_2\|
\end{align*}
The above inequalities are due to the fact that expectation of norm is
greater than norm of expectation, and the $L$-smoothness assumption on
$J(\theta_1)$. We interchange the expectation and derivative using the
assumptions on $J(\theta_1)$ and the dominated convergence
theorem. Thus, the smoothened loss function $\hat{J}(\theta_1)$ is
also $L$-smooth.


We know,
  \begin{align*}
    \nabla_\theta \hat{J}(\theta) &= \nabla_\theta\mathbb{E}_{v \sim \mathbb{B}_d}[J(\theta +
                             \delta v)] \nonumber \\
    &= \mathbb{E}_{v \sim \mathbb{B}_d}[\nabla_\theta J(\theta + \delta v)]
  \end{align*}
  Note that the expectation and derivative can be interchanged using
  the dominated convergence theorem. Hence, we have
  \begin{align*}
    \|\nabla_\theta \hat{J}(\theta) - \nabla_\theta J(\theta)\| &= \|\mathbb{E}_{u \sim
                                                  \mathbb{B}_d}[\nabla_\theta
                                                  J(\theta + \delta v)]
                                                  - \nabla_\theta J(\theta)\|
                                                  \nonumber \\
                                                &\leq \mathbb{E}_{u \sim
                                                  \mathbb{B}_d}\|\nabla_\theta
                                                  J(\theta + \delta v) -
                                                  \nabla_\theta J(\theta)\|
                                                  \nonumber \\
                                                &\leq \mathbb{E}_{u
                                                  \sim \mathbb{B}_d}[L
                                                  ||\delta v||]
                                                  \nonumber \\
                                                &\leq L \delta
  \end{align*}
\end{proof}

The above lemma will be very useful later when we try to relate the
convergence rate for the smoothed objective and the true objective. It is shown in
\citep{flaxman2005online, agarwal2010optimal} that the gradient estimate $g_i$ is an
unbiased estimator of the gradient $\nabla_\theta
\hat{J}(\theta_i)$. Hence, Algorithm \ref{alg:random_search_parameter}
is performing SGD on the smoothed objective $\hat{J}(\theta)$. Using
this insight, we can use the convergence rate of SGD for nonconvex
functions to stationary points from \citep{ghadimi2013stochastic} which is given as
follows
\begin{lemma}[\citep{ghadimi2013stochastic}]
  \label{lemma:sgd-parameter}
  Consider running SGD on the objective $\hat{J}(\theta)$ that is
  $L$-smooth and $G$-Lipschitz for $T$ steps. Fix initial solution
  $\theta_0$ and denote $\Delta_0 = \hat{J}(\theta_0) -
  \hat{J}(\theta^*)$ where $\theta^*$ is the point at which
  $\hat{J}(\theta)$ attains global minimum. Also, assume that the
  gradient estimate $g_i$ is unbiased and has a bounded variance,
  i.e. for all $i$, $\mathbb{E}_i[\|g_i - \nabla_\theta
  \hat{J}(\theta_i)\|_2^2] \leq V \in \mathbb{R}^+$ where
  $\mathbb{E}_i$ denotes expectation with randomness only at iteration
  $i$ conditioned on history upto iteration $i-1$. Then we have,
  \begin{equation}
    %\label{eq:sgd-parameter}
    \frac{1}{T} \sum_{i=1}^T \mathbb{E}\|\nabla_\theta
    \hat{J}(\theta_i)\|_2^2 \leq \frac{2\sqrt{2\Delta_0L(V+G^2)}}{\sqrt{T}}
  \end{equation}
\end{lemma}
For completeness, we include a proof of the above lemma below.
\begin{proof}[Proof of Lemma \ref{lemma:sgd-parameter}]
  Denote $\xi_i = g_i - \nabla_\theta {\hat{J}}(\theta_i)$.  Note that $\mathbb{E}_{i} [\xi_i] =
0$ since the stochastic gradient $g_i$ is unbiased.
From  $\theta_{i+1} = \theta_i - \alpha g_i$, we have:
\begin{align*}
  \hat{J}(\theta_{i+1}) & = \hat{J}(\theta_{i} - \alpha g_i)\\
  &\leq\hat{J}(\theta_i) - \nabla_\theta \hat{J}(\theta_i)^{\top} (\alpha g_i) + \frac{L\alpha^2}{2}\| g_i\|_2^2  \\
    & = \hat{J}(\theta_i) - \alpha \nabla_\theta \hat{J}(\theta_i)^{\top} g_i + \frac{L\alpha^2}{2} \|\xi_i + \nabla_\theta \hat{J}(\theta_i)\|^2_2 \\
    & = \hat{J}(\theta_i) - \alpha \nabla_\theta
      \hat{J}(\theta_i)^{\top} g_i + \frac{L\alpha^2}{2}(\|\xi_i\|_2^2
  + 2\xi_i^{\top}\nabla_\theta \hat{J}(\theta_i) + \|\nabla_\theta \hat{J}(\theta_i)\|_2^2 )
\end{align*} The first inequality above is obtained since the loss
function $\hat{J}(\theta)$ is $L$-smooth. Adding $\mathbb{E}_i$ on both sides and using the fact that $\mathbb{E}_i [\xi_i] = 0$, we have:
\begin{align*}
    \mathbb{E}_i [\hat{J}(\theta_{i+1})] &= \hat{J}(\theta_i) - \alpha
                                           \|\nabla_\theta
                                           \hat{J}(\theta_i)\|_2^2  +\frac{L\alpha^2}{2}\left( \mathbb{E}_i [\|\xi_i\|_2^2] + \|\nabla_\theta \hat{J}(\theta_i)\|_2^2  \right)  \\
    &\leq \hat{J}(\theta_i) - \alpha \|\nabla_\theta
      \hat{J}(\theta_i)\|_2^2 + \frac{L\alpha^2}{2}\left( \mathbb{E}_i [\|\xi_i\|_2^2] + G^2  \right)
\end{align*}
where the inequality is due to the lipschitz assumption. Rearranging terms, we get:
\begin{align*}
    \alpha\|\nabla_\theta \hat{J}(\theta_i)\|_2^2 &= \hat{J}(\theta_i)
      - \mathbb{E}_i [\hat{J}(\theta_{i+1})] + \frac{L\alpha^2}{2} (\mathbb{E}_i [\|\xi_i\|_2^2] + G^2) \\
    & \leq \hat{J}(\theta_i) - \mathbb{E}_i[ \hat{J}(\theta_{i+1})] + \frac{L\alpha^2}{2} (V + G^2)
\end{align*}
Sum over from time step $1$ to $T$, we get:
\begin{align*}
    \alpha \sum_{t=1}^T \mathbb{E}\|\nabla_\theta
  \hat{J}(\theta_i)\|_2^2 &\leq \mathbb{E} [\hat{J}(\theta_0) -
                            \hat{J}(\theta_T)] + \frac{LT\alpha^2}{2}(V+G^2)
\end{align*} Divide $\alpha$  on both sides, we get:
\begin{align*}
    \sum_{t=1}^{T} \mathbb{E}&\|\nabla_\theta \hat{J}(\theta_i)\|_2^2 \leq \frac{1}{\alpha} \mathbb{E}[\hat{J}(\theta_0) - \hat{J}(\theta_T)] + {LT\alpha} (V+G^2) \\
    & \leq \frac{1}{\alpha} \mathbb{E}[\hat{J}(\theta_0) - \hat{J}(\theta^*)] + {LT\alpha} (V+G^2)  \\
    & = \frac{1}{\alpha} \Delta_0 + {LT\alpha} (V+G^2)  \\
    & \leq \sqrt{\frac{\Delta_0LT(V+G^2)}{2}} + \sqrt{2\Delta_0
      LT(V+G^2)} \\
      &\leq 2\sqrt{2\Delta_0 LT(V+G^2)}
\end{align*} with $\alpha = \sqrt{\frac{2\Delta_0}{LT(V+G^2)}}$.
Hence, we have:
\begin{align*}
    \frac{1}{T}\sum_{t=1}^T\mathbb{E}\|\nabla_\theta \hat{J}(\theta_i)\|_2^2 \leq \frac{2\sqrt{2\Delta_0 L(V+G^2)}}{\sqrt{T}}
\end{align*}
\end{proof}



The above lemma is useful as it gives us the following result:
\begin{align}
  \label{eq:stationary-point}
  \min_{1 \leq i \leq T} \mathbb{E}\|\nabla_\theta\hat{J}(\theta_i)\|_2^2 &\leq \frac{1}{T} \sum_{i=1}^T \mathbb{E}\|\nabla_\theta
                                                               \hat{J}(\theta_i)\|_2^2
                                                               \nonumber
  \\
                                                             &\leq \frac{2\sqrt{2\Delta_0L(V+G^2)}}{\sqrt{T}}
\end{align}
since the minimum is always less than the average. We have then that
using SGD to minimize a nonconvex objective finds a $\theta_i$ that is
`almost' a stationary point in bounded number of steps provided the
stochastic gradient estimate has bounded variance.

We now show that the gradient estimate $g_i$ used in Algorithm
\ref{alg:random_search_parameter} indeed has a bounded variance. Observe that
the estimate $g_i$ in the algorithm is a two-point estimate, which
should have substantially less variance than one-point
estimates \citep{agarwal2010optimal}. However, the two evaluations, resulting in $J_i^+$ and
$J_i^-$, have different independent noise. This is due to the
fact that in policy search, stochasticity arises from the
environment and cannot be controlled and we cannot obtain the
significant variance reduction that is typical of two-point
estimators. The following lemma quantifies the bound on the variance of
gradient estimate $g_i$:
\begin{lemma}
\label{lemma:grad-variance-parameter}
  Consider a smoothed objective $\hat{J}(\theta) = \mathbb{E}_{v \sim
    \mathbb{B}_d}[J(\theta + \delta v)]$ where $\mathbb{B}_d$ is the
  unit ball in $d$ dimensions, $\delta > 0$ is a scalar and the true
  objective $J(\theta)$ is $G$-lipschitz. Given gradient estimate $g_i
  = \frac{d(J_i^+ - J_i^-)}{2\delta}u$ where $u$ is sampled uniformly
  from a unit sphere $\mathbb{S}_d$ in $d$ dimensions, $J^+_i =
  J(\theta_i + \delta u) + \eta^+_i$ and $J_i^- = J(\theta - \delta u)
  + \eta_i^-$ for zero mean random i.i.d noises $\eta_i^+, \eta_i^-$, we have
  \begin{equation}
    \label{eq:grad-variance-parameter}
    \mathbb{E}_i[\|g_i - \nabla_\theta \hat{J}(\theta_i)\|_2^2] \leq
    2d^2G^2 + 2\frac{d^2\sigma^2}{\delta^2}
  \end{equation}
  where {$\sigma^2$ is the variance of the random noise $\eta$.}
\end{lemma}


\begin{proof}[Proof of Lemma \ref{lemma:grad-variance-parameter}]
  From \cite{shamir2017optimal}, we know that $g_i$ is an unbiased estimate of the gradient of $\hat{J}(\theta_i)$, i.e. $\mathbb{E}_{u_i \sim \mathbb{S}_d}[g_i] = \nabla\hat{J}(\theta_i)$. Thus, we have
\begin{align*}
    \mathbb{E}_{u_i \sim \mathbb{S}_d}&\|g_i -
      \nabla\hat{J}(\theta_i)\|^2 \\
      &= \mathbb{E}_{u_i \sim \mathbb{S}_d}[\|g_i\|^2 + \|\nabla \hat{J}(\theta)_i\|^2 - 2g_i^T\nabla \hat{J}(\theta_i)] \\
    &= \mathbb{E}_{u_i \sim \mathbb{S}_d}\|g_i\|^2 + \|\nabla \hat{J}(\theta_i)\|^2 - 2\|\nabla \hat{J}(\theta_i)\|^2 \\
    &= \mathbb{E}_{u_i \sim \mathbb{S}_d}\|g_i\|^2 - \|\nabla \hat{J}(\theta_i)\|^2 \\
    &\leq \mathbb{E}_{u_i \sim \mathbb{S}_d}\|g_i\|^2 \\
    &= \frac{d^2}{4\delta^2}\mathbb{E}_{u_i \sim
      \mathbb{S}_d}\|(J(\theta_i + \delta u_i) - J(\theta_i - \delta
      u_i) + (\eta_i^+ - \eta_i^-))u_i\|^2 \\
  &\leq \frac{d^2}{2\delta^2}[\mathbb{E}_{u_i \sim \mathbb{S}_d}\|(J(\theta_i + \delta u_i) - J(\theta_i - \delta
    u_i)u_i\|_2^2 + \mathbb{E}_{u_i \sim \mathbb{S}_d}\|(\eta_i^+ -
    \eta_i^-))u_i\|^2] \\
    &\leq \frac{d^2}{2\delta^2}[\mathbb{E}_{u_i \sim \mathbb{S}_d}
      4G^2\delta^2 \|u_i\|^2 +  4\mathbb{E}_{u_i \sim
      \mathbb{S}_d}\|\eta_i^+\|_2^2 \|u_i\|_2^2]\\
    &= 2d^2G^2  + 2\frac{d^2\sigma^2}{\delta^2}
\end{align*}
where the second inequality is true as $\|a+b\|_2^2 \leq 2(\|a\|_2^2 +
\|b\|_2^2)$ and the last inequality is due to the Lipschitz assumption
on $J(\theta)$.
\end{proof}

We are ready to prove Theorem~\ref{theorem:parameter-convergence}.
\begin{proof}[Proof of Theorem \ref{theorem:parameter-convergence}]
  Fix initial solution $\theta_0$ and denote $\Delta_0 =
  \hat{J}(\theta_0) - \hat{J}(\theta^*)$ where $\hat{J}(\theta)$ is
  the smoothed objective and $\theta^*$ is the point at which
  $\hat{J}(\theta)$  attains global minimum.
  Since the gradient estimate $g_i$ used in Algorithm
  \ref{alg:random_search_parameter} is an unbiased estimate of the
  gradient $\nabla_\theta \hat{J}(\theta_i)$, we know that Algorithm
  \ref{alg:random_search_parameter} performs SGD on the smoothed
  objective. Moreover, from Lemma \ref{lemma:grad-variance-parameter},
  we know that the variance of the gradient estimate $g_i$ is
  bounded. Hence, we can use Lemma \ref{lemma:sgd-parameter} on the
  smoothed objective $\hat{J}(\theta)$ to get
  \begin{align}
    \label{eq:sgd-parameter}
    \frac{1}{T} \sum_{i=1}^T \mathbb{E}\|\nabla_\theta
    \hat{J}(\theta_i)\|_2^2 \leq \frac{2\sqrt{2\Delta_0L(V+G^2)}}{\sqrt{T}}
  \end{align}
  where $V \leq 2d^2G^2 + 2\frac{d^2\sigma^2}{\delta^2}$ (from Lemma
  \ref{lemma:grad-variance-parameter}). We can relate $\nabla_\theta
  \hat{J}(\theta)$ and $\nabla_\theta J(\theta)$ - the quantity that
  we ultimately care about, as follows:
  \begin{align*}
    \frac{1}{T} &\sum_{i=1}^T \mathbb{E}\|\nabla_\theta
    J(\theta_i)\|_2^2\\
    &= \frac{1}{T} \sum_{i=1}^T
                        \mathbb{E}\|\nabla_\theta J(\theta_i) -
                        \nabla_\theta \hat{J}(\theta_i) +
      \nabla_\theta \hat{J}(\theta_i)\|_2^2 \\
    &\leq \frac{2}{T} \sum_{i=1}^T \mathbb{E}\|\nabla_\theta J(\theta_i) -
                        \nabla_\theta \hat{J}(\theta_i)\|_2^2 + \mathbb{E}\|\nabla_\theta \hat{J}(\theta_i)\|_2^2
  \end{align*}
  We can use Lemma \ref{lemma:grad-diff-parameter} to bound the first
  term and Equation \ref{eq:sgd-parameter} to bound the second
  term. Thus, we have
  \begin{align*}
    \frac{1}{T} \sum_{i=1}^T \mathbb{E}\|\nabla_\theta
    J(\theta_i)\|_2^2 \leq \frac{2}{T}[TL^2\delta^2 + 2\sqrt{2\Delta_0L(V+G^2)T}]
  \end{align*}
  Substituting the bound for $V$ from Lemma
  \ref{lemma:grad-variance-parameter}, using the inequality
  $\sqrt{a+b} \leq \sqrt{a} + \sqrt{b}$ for $a, b \in \mathbb{R}^+$,
  optimizing over $\delta$, and using $\Delta_0 \leq \Qbound$ we get
  \begin{equation*}
    \frac{1}{T} \sum_{i=1}^T \mathbb{E}\|\nabla_\theta
    J(\theta_i)\|_2^2 \leq \mathcal{O}(\Qbound^{\frac{1}{2}}dT^{\frac{-1}{2}} + \Qbound^{\frac{1}{3}}d^{\frac{2}{3}}T^{\frac{-1}{3}}\sigma)
  \end{equation*}
\end{proof}

\subsection{Proof of Theorem~\ref{theorem:action-convergence}}
\label{sec:proof-action-convergence}

The bound on the bias of the gradient estimate is given
by the following lemma:
\begin{lemma}
  \label{lemma:bias-bound-action}
  If the assumptions in Section \ref{sec:assumptions_action} are
  satisfied, then for  the gradient estimate $g_i$ used in Algorithm
  \ref{alg:random_search_action} and the gradient of the objective
  $J(\theta)$ given in equation \ref{eq:dpg-gradient}, we have
  \begin{equation}
    \label{eq:bias-bound-action}
    \|\mathbb{E}[g_i] - \nabla_\theta J(\theta_i)\| \leq KUH\delta
  \end{equation}
\end{lemma}


\begin{proof}[Proof of Lemma \ref{lemma:bias-bound-action}]
  To prove that the bias is bounded, let's consider for any $i$
  \begin{align*}
      &\|\mathbb{E}[g_i] - \nabla_\theta J(\theta_i)\|_2 = \|\sum_{t=0}^{H-1} \mathbb{E}_{s_t\sim
        d_{\pi_{\theta_i}}^t}[\nabla_\theta \pi(\theta_i, s_t) \nabla_a (\mathbb{E}_{v \sim \mathbb{B}_p}
        Q_{\pi_{\theta_i}}^t(s_t, \pi(\theta_i, s_t) + \delta v) -
          Q_{\pi_{\theta_i}}^t(s_t, \pi(\theta_i, s_t)))]\|_2 \\
    &\leq \sum_{t=0}^{H-1} \mathbb{E}_{s_t\sim d_{\pi_{\theta_i}}^t, v
      \sim \mathbb{B}_p}\|\nabla_\theta \pi(\theta_i, s_t)\|_2 \|[\nabla_a
      Q_{\pi_{\theta_i}}^t(s_t, \pi(\theta_i, s_t) + \delta v) -
      \nabla_a Q_{\pi_{\theta_i}}^t(s_t, \pi(\theta_i, s_t))]\|_2 \\
    &\leq \sum_{t=0}^{H-1} KU\delta \mathbb{E}_{v \sim
      \mathbb{B}_p}\|v\|_2 \\
    &\leq KUH\delta
  \end{align*}
  The first inequality above is obtained by using the fact that
  $\|\mathbb{E}[X]\|_2 \leq \mathbb{E}\|X\|_2$, and the second
  inequality using the $K$-lipschitz assumption on $\pi(\theta, s)$
  and $U$-smooth assumption on $Q_{\pi_\theta}^t(s, a)$ in $a$. Also,
  observe that we interchanged the derivative and expectation above by
  using the assumptions on $Q_{\pi_\theta}^t$ as stated in Section
  \ref{sec:assumptions_action}.
\end{proof}

We will now show that the gradient estimate $g_i$ used in Algorithm
\ref{alg:random_search_action} has a bounded variance. Note that the
gradient estimate constructed in Algorithm
\ref{alg:random_search_action} is a one-point estimate, unlike policy
search in parameter space where we had a two-point estimate. Thus, the variance would be higher and the bound
on the variance of such a one-point estimate is given below
\begin{lemma}
  \label{lemma:grad-variance-action}
  Given a gradient estimate $g_i$ as shown in Algorithm
  \ref{alg:random_search_action}, the variance of the estimate can be
  bounded as
  \begin{equation}
    \label{eq:grad-variance-action}
    \mathbb{E}\|g_i - \mathbb{E}[g_i]\|_2^2 \leq
    \frac{2H^2p^2K^2}{\delta^2} ((\Qbound + W\delta)^2 + \sigma^2)
  \end{equation}
  where $\sigma^2$ is the variance of the random noise $\tilde{\eta}$.
\end{lemma}



\begin{proof}[Proof of Lemma \ref{lemma:grad-variance-action}]
  To bound the variance of the gradient estimate $g_i$ in Algorithm
  \ref{alg:random_search_action}, lets consider
  \begin{align*}
    &\mathbb{E}_i\|g_i - \mathbb{E}[g_i]\|_2^2 = \mathbb{E}_i\|g_i\|_2^2 -
                                              \|\mathbb{E}_i[g_i]\|_2^2
                                            \leq \mathbb{E}_i\|g_i\|_2^2 \\
    &= \frac{H^2p^2}{\delta^2} \mathbb{E}_i\|\nabla_\theta
      \pi(\theta_i, s_t)
    (Q_{\pi_{\theta_i}}^t(s_t, \pi(\theta_i, s_t)
      + \delta u) + \tilde{\eta}_i) u\|_2^2 \\
    &\leq \frac{K^2p^2H^2}{\delta^2} \mathbb{E}_{i}\|Q_{\pi_{\theta_i}}^t(s_t, \pi(\theta_i,
      s_t) + \delta u)u + \tilde{\eta}_iu\|_2^2
  \end{align*}
  where $\mathbb{E}_i$ denotes expectation with respect to the
  randomness at iteration $i$ and the inequality is obtained using
  $K$-lipschitz assumption on $\pi(\theta, s)$. Note that we can
  express $Q_{\pi_{\theta_i}}^t(s_t, \pi(\theta_i, s_t) + \delta u)
  \leq Q_{\pi_{\theta_i}}^t(s_t, \pi(\theta_i, s_t)) + W\delta\|u\|_2
  \leq \Qbound + W\delta$ where we used the $W$-lipschitz assumption on
  $Q_{\pi_{\theta}}^t(s, a)$ in $a$ and that it is bounded everywhere
  by constant $\Qbound$. Thus, we have
  \begin{align*}
    &\mathbb{E}_i\|g_i - \mathbb{E}[g_i]\|_2^2 \\
    &\leq \frac{K^2p^2H^2}{\delta^2} \mathbb{E}_{i}\|(\Qbound + W\delta)u +
      \tilde{\eta}_iu\|_2^2 \\
    &\leq \frac{2K^2p^2H^2}{\delta^2}
      (\mathbb{E}_i\|(\Qbound+W\delta)u\|_2^2 +
      \mathbb{E}_i\|\tilde{\eta}_iu\|_2^2 \\
    &\leq \frac{2K^2p^2H^2}{\delta^2} ((\Qbound+W\delta)^2 + \sigma^2)
  \end{align*}
\end{proof}

We are now ready to prove theorem \ref{theorem:action-convergence}
\begin{proof}[Proof of Theorem \ref{theorem:action-convergence}]
  Fix initial solution $\theta_0$ and denote $\Delta_0 =
  J(\theta_0) - J(\theta^*)$ where $\theta^*$ is the point at which
  $J(\theta)$ attains global minimum. Denote $\xi_i = g_i - \mathbb{E}_i[g_i]$ and $\beta_i =
  \mathbb{E}_i[g_i] - \nabla_\theta J(\theta_i)$. From Lemma
  \ref{lemma:bias-bound-action}, we know $\|\beta_i\| \leq KUH\delta$
  and from lemma \ref{lemma:grad-variance-action}, we know
  $\mathbb{E}\|\xi_i\|_2^2 = V \leq \frac{2K^2p^2H^2}{\delta^2} ((\Qbound +
  W\delta)^2 + \sigma^2)$ and $\mathbb{E}_i[\xi_i] = 0$ from definition. From $\theta_{i+1} = \theta_i - \alpha g_i$
  we have:
  \begin{align*}
    J(\theta_{i+1}) &= J(\theta_i - \alpha g_i) \\
    &\leq J(\theta_i) - \alpha \nabla_\theta J(\theta_i)^Tg_i +
      \frac{L\alpha^2}{2}\|g_i\|_2^2 \\
    &= J(\theta_i) - \alpha \nabla_\theta J(\theta_i)^T g_i +
      \frac{L\alpha^2}{2}\|\xi_i + \mathbb{E}_i[g_i]\|_2^2 \\
                    &= J(\theta_i) - \alpha \nabla_\theta J(\theta_i)^T g_i +
      \frac{L\alpha^2}{2}(\|\mathbb{E}_i[g_i]\|_2^2 + \|\xi_i\|_2^2 + 2\mathbb{E}_i[g_i]^T\xi_i)
  \end{align*}
  Taking expectation on both sides with respect to randomness at
  iteration $i$, we have
  \begin{align*}
    &\mathbb{E}_i[J(\theta_{i+1})] = J(\theta_i) - \alpha\nabla_\theta
    J(\theta_i)^T\mathbb{E}_i[g_i] + \frac{L\alpha^2}{2}(\|\mathbb{E}_i [g_i]\|_2^2 +
    \mathbb{E}_i\|\xi_i\|_2^2 +
      2\mathbb{E}_i[g_i]^T\mathbb{E}_i[\xi_i]) \\
    &\leq J(\theta_i) - \alpha\nabla_\theta J(\theta_i)^T (\beta_i +
      \nabla_\theta J(\theta_i)) +\frac{L\alpha^2}{2}(\|\beta_i + \nabla_\theta
      J(\theta_i)\|_2^2 + V) \\
    &= J(\theta_i) - \alpha\|\nabla_\theta J(\theta_i)\|_2^2 +
      \frac{L\alpha^2}{2}(\|\nabla_\theta J(\theta_i)\|_2^2 + V +
      \|\beta_i\|_2^2) + (L\alpha^2 - \alpha)\nabla_\theta J(\theta_i)^T\beta_i \\
    &\leq J(\theta_i) - \alpha\|\nabla_\theta J(\theta_i)\|_2^2 +
      \frac{L\alpha^2}{2}(G^2 + V + K^2H^2U^2\delta^2) + (L\alpha^2 - \alpha)\nabla_\theta J(\theta_i)^T\beta_i \\
    &\leq J(\theta_i) - \alpha\|\nabla_\theta J(\theta_i)\|_2^2 +
      \frac{L\alpha^2}{2}(G^2 + V + K^2H^2U^2\delta^2) + (L\alpha^2 + \alpha)\|\nabla_\theta
      J(\theta_i)\|\|\beta_i\| \\
    &\leq J(\theta_i) - \alpha\|\nabla_\theta J(\theta_i)\|_2^2 +
      \frac{L\alpha^2}{2}(G^2 + V + K^2H^2U^2\delta^2) + (L\alpha^2 + \alpha)GKUH\delta
  \end{align*}
  Rearranging terms and summing over timestep $1$ to $T$, we get
  \begin{align*}
    &\alpha\sum_{i=1}^T \|\nabla_\theta J(\theta_i)\|_2^2 \leq J(\theta_0) -
      \mathbb{E}_T[J(\theta_T)] + \frac{LT\alpha^2}{2}(G^2 + V +
      K^2H^2U^2\delta^2) + (L\alpha^2 + \alpha)GKUHT\delta \\
    &\leq \Delta_0 + \frac{LT\alpha^2}{2}(G^2 + V +
      K^2H^2U^2\delta^2) + (L\alpha^2 + \alpha)GKUHT\delta \\
    &\sum_{i=1}^T \|\nabla_\theta J(\theta_i)\|_2^2 \leq
      \frac{\Delta_0}{\alpha} + \frac{LT\alpha}{2}(G^2 + V +
      K^2H^2U^2\delta^2) + (L\alpha + 1)GKUHT\delta \\
    &\leq \frac{\Delta_0}{\alpha} + \frac{LT\alpha}{2}(G^2 +
      K^2H^2U^2\delta^2 + 2GKUH\delta)+ GKUHT\delta + \frac{LT\alpha}{2}V \\
    &\leq \frac{\Delta_0}{\alpha} + \frac{LT\alpha}{2}(G+KHU\delta)^2
    + GKUHT\delta +
      \frac{LT\alpha K^2p^2H^2}{\delta^2}((\Qbound + W\delta)^2 + \sigma^2) \\
      &\leq  \frac{\Delta_0}{\alpha} + LT\alpha(G^2 + K^2H^2U^2\delta^2) + GKUHT\delta + 2\frac{LT\alpha K^2p^2H^2}{\delta^2}(\Qbound^2 + W^2\delta^2 +\sigma^2)
  \end{align*}
  Using $\Delta_0 \leq \Qbound$ and optimizing over $\alpha$ and $\delta$, we get $\alpha = \mathcal{O}(\Qbound^{\frac{3}{4}}T^{-\frac{3}{4}}H^{-1}p^{-\frac{1}{2}}(\Qbound^2 + \sigma^2)^{-\frac{1}{4}})$ and $\delta = \mathcal{O}(T^{-\frac{1}{4}}p^{\frac{1}{2}}(\Qbound^2 + \sigma^2)^{\frac{1}{4}})$. This gives us
  \begin{equation}
      \frac{1}{T}\sum_{i=1}^T \|\nabla_\theta J(\theta_i)\|_2^2 \leq \mathcal{O}(T^{-\frac{1}{4}}Hp^{\frac{1}{2}}(\Qbound^3 + \sigma^2\Qbound)^{\frac{1}{4}})
  \end{equation}

\end{proof}

\iffalse
\subsection{Lower Bound Construction and Proof}

\begin{corollary}
to do
\end{corollary}
\begin{proof}
Since the convex decision set $\mathrm{W} = \{w: \|w\|_{2}\leq 1\}$ is bounded and compact, we can pick two points $w$ and $w'$ from $\mathrm{W}$ such that $\| w - w'\|_2 = 1$. Since $\|w - w'\|_2 = \sup_{l:\|l\|_2\leq 1}\langle l, w - w' \rangle$, and the set $\{l: \|l\|_2 \leq \}$ is compact, there exists $l$ such that $l^{\top}(w- w') = 1$ and $\|l\|_2 = 1$.

At any round $t$, we construct a linear loss function as follows. We sample a Rademacher variable $Z_t \in \{-1,1\}$ uniformly randomly, and the loss function $\ell_t(w) = (c Z_t l)^{\top} w$ with $c\in\mathbb{R}^+$.  Note that for any $t$, $\mathbb{E}_t[\ell_t(w)] = \mathbf{0}^{\top} w \triangleq \ell(w)$, as $\mathbb{E}_t [c Z_t] = \mathbf{0}$. Also, $\nabla_{w} \ell_t(w) = c Z_t l$ and $\mathbb{E}_{t}[\nabla_{w}\ell_t(w)] = \mathbf{0}$, which is equal to $\nabla_{w} \ell(w)$. Hence, running Online gradient descent on the sequence of loss function $\{\ell_t(w)\}$ is equivalent to running SGD with stochastic gradient $\nabla_{w} \ell_t(w)$ on the loss function $\ell(w)$.

Define regret as:
\begin{align*}
    \mathrm{Regret} = \sum_{t=1}^T \ell_t(w_t) - \min_{w\in\mathrm{W}} \sum_{t=1}^{T} \ell_t(w)
\end{align*} Note that for any sequence of $\{\ell_t\}$, we have:
\begin{align*}
    w^{\star} = \arg\min_{w\in\mathrm{W}} \sum_{t=1}^T \ell_t(w) = \frac{-\sum_{t=1}^{T} c Z_t l}{\|\sum_{t=1}^{T} c Z_t l\|_2} = -c l \frac{\sum_{t=1}^T Z_t} {\lvert\sum_{t=1}^T Z_t\rvert}.
\end{align*} Hence, $\min_{w\in\mathrm{W}} \sum_{t=1}^T \ell_t(w)$ is equal to:
\begin{align*}
   \min_{w\in\mathrm{W}} \sum_{t=1}^T \ell_t(w) = -c^2 \left\lvert \sum_{t=1}^T Z_t  \right\rvert.
\end{align*}



\end{proof}
\fi

\subsection{Implementation Details}
\label{sec:impl-deta}

\subsubsection{One-step Control Experiments}
\label{sec:one-step-control-1}

\paragraph{Tuning Hyperparameters for ARS}
We tune the hyperparameters for ARS \citep{mania2018simple} in both MNIST and linear regression experiments, by choosing a candidate set of values for each hyperparameter: stepsize, number of directions sampled, number of top directions chosen and the perturbation length along each direction. The candidate hyperparameter values are shown in Table \ref{tab:hyperparam}.

\begin{table}[ht]
    \centering
    \begin{tabular}{|c|c|}
      \hline
      \textbf{Hyperparameter} & \textbf{Candidate Values}\\
    \hline
    Stepsize &  $0.001, 0.005, 0.01, 0.02, 0.03$\\
    \hline
    \# Directions &  $10, 50, 100, 200, 500$\\
    \hline
    \# Top Directions & $5, 10, 50, 100, 200$\\
    \hline
    Perturbation & $0.001, 0.005, 0.01, 0.02, 0.03$ \\
    \hline
    \end{tabular}
    \caption{Candidate hyperparameters used for tuning in ARS  experiments}
    \label{tab:hyperparam}
\end{table}

We use the hyperparameters shown in Table \ref{tab:chosen-hyperparams} chosen through this tuning for each of the experiments in this work. The hyperparameters are chosen by averaging the test squared loss across three random seeds (different from the 10 random seeds used in actual experiments) and chosing the setting that has the least mean test squared loss after 100000 samples.

\begin{table}[ht]
    \centering
    \begin{tabular}{|c|c|c|c|c|}
    \hline
    \textbf{Experiment} & \textbf{Stepsize} & \textbf{\# Dir}. & \textbf{\# Top Dir.} & \textbf{Perturbation}\\
    \hline
    MNIST     &  0.02 & 50 & 20 & 0.03\\
    \hline
    LR $d=10$     & 0.03 & 10 & 10 & 0.03 \\
    \hline
    LR $d=100$ & 0.03 & 10 & 10 & 0.02 \\
    \hline
    LR $d=1000$ & 0.03 & 200 & 200 & 0.03 \\
    \hline
    \end{tabular}
    \caption{Hyperparameters chosen for ARS in each experiment. LR is short-hand for Linear Regression.}
    \label{tab:chosen-hyperparams}
\end{table}

\begin{table}[ht]
    \centering
    \begin{tabular}{|c|c|c|}
    \hline
    \textbf{Experiment}     &  \textbf{Learning Rate} & \textbf{Batch size}\\
    \hline
    MNIST     &  0.001 & 512\\
    \hline
    LR $d=10$ & 0.08 & 512\\
    \hline
    LR $d=100$ & 0.03 & 512\\
    \hline
    LR $d=1000$ & 0.01 & 512\\
    \hline
    \end{tabular}
    \caption{Learning rate and batch size used for REINFORCE experiments. We use an ADAM \citep{kingma2014adam} optimizer for these experiments.}
    \label{tab:hyperparam-reinforce}
\end{table}

\begin{table}[ht]
    \centering
    \begin{tabular}{|c|c|c|}
    \hline
    \textbf{Experiment}     &  \textbf{Learning Rate} & \textbf{Batch size}\\
    \hline
    LR $d=10$ & 2.0 & 512\\
    \hline
    LR $d=100$ & 2.0 & 512\\
    \hline
    \end{tabular}
    \caption{Learning rate and batch size used for Natural REINFORCE experiments. Note that we decay the learning rate after each batch by $\sqrt{T}$ where $T$ is the number of batches seen.}
    \label{tab:hyperparam-nreinforce}
\end{table}

\paragraph{MNIST Experiments}
\label{sec:mnist-details}

The CNN architecture used is as shown in Figure \ref{fig:arch}\footnote{This figure is generated by adapting the code from \url{https://github.com/gwding/draw_convnet}}. The total number of parameters in this model is $d=21840$. For supervised learning, we use a cross-entropy loss on the softmax output with respect to the true label. To train this model, we use a batch size of 64 and a stochastic gradient descent (SGD) optimizer with learning rate of 0.01 and a momentum factor of 0.5. We evaluate the test accuracy of the model over all the $10000$ images in the MNIST test dataset.

\begin{figure}[H]
    \centering
    \includegraphics[width=0.9\linewidth]{figures/aistats19/conv.png}
    \caption{CNN architecture used for the MNIST experiments}
    \label{fig:arch}
\end{figure}

For REINFORCE, we use the same architecture as before. We train the model by sampling from the categorical distribution parameterized by the softmax output of the model and then computing a $\pm 1$ reward based on whether the model predicted the correct label. The loss function is the REINFORCE loss function given by,
\begin{equation}
    J(\theta) = \frac{1}{N} \sum_{i=1}^N r_i \log(\mathbb{P}(\hat y_i|x_i, \theta))
\end{equation}
where $\theta$ is the parameters of the model, $r_i$ is the reward obtained for example $i$, $\hat y_i$ is the predicted label for example $i$ and $x_i$ is the input feature vector for example $i$. The reward $r_i$ is given by $r_i = 2*\mathbb{I}[\hat y_i = y_i] - 1$, where $\mathbb{I}$ is the $0-1$ indicator function and $y_i$ is the true label for example $i$.

For ARS, we use the same architecture and reward function as before. The hyperparameters used are shown in Table \ref{tab:chosen-hyperparams} and we closely follow the algorithm outlined in \citep{mania2018simple}.

\paragraph{Linear Regression Experiments}
\label{sec:linreg-details}

We generate training and test data for the linear regression experiments as follows: we sampled a random $d+1$ dimensional vector $w$ where $d$ is the input dimensionality. We also sampled a random $d \times d$ covariance matrix $C$. The training and test dataset consists of $d+1$ vectors $x$ whose first element is always $1$ (for the bias term) and the rest of the $d$ terms are sampled from a multivariate normal distribution with mean $\mathbf{0}$ and covariance matrix $C$. The target vectors $y$ are computed as $y = w^Tx + \epsilon$ where $\epsilon$ is sampled from a univariate normal distribution with mean $0$ and standard deviation $0.001$.

We implemented both SGD and Newton Descent on the mean squared loss, for the supervised learning experiments. For SGD, we used a learning rate of $0.1$ for $d=10, 100$ and a learning rate of $0.01$ for $d=1000$, and a batch size of 64. For Newton Descent, we also used a batch size of 64. To frame it as a one-step MDP, we define a reward function $r$ which is equal to the negative of mean squared loss. Both REINFORCE and ARS use this reward function. To compute the REINFORCE loss, we take the prediction of the model $\hat{w}^Tx$, add a mean $0$ standard deviation $\beta = 0.5$ Gaussian noise to it, and compute the reward (negative mean squared loss) for the noise added prediction. The REINFORCE loss function is then given by
\begin{equation}
    J(w) = \frac{1}{N} \sum_{i=1}^N r_i \frac{- (y_i - \hat{w}^Tx_i)^2}{2\beta^2}
\end{equation}
where $r_i = -(y_i - \hat y_i)^2$, $\hat y_i$ is the noise added prediction and $\hat{w}^Tx_i$ is the prediction by the model. We use an Adam optimizer with learning rate and batch size as shown in Table \ref{tab:hyperparam-reinforce}. For the natural REINFORCE experiments, we estimate the fisher information matrix and compute the descent direction by solving the linear system of equations $Fx = g$ where $F$ is the fisher information matrix and $g$ is the REINFORCE gradient. We use SGD with a $O(1/\sqrt{T})$ learning rate, where $T$ is the number of batches seen, and batch size as shown in Table \ref{tab:hyperparam-nreinforce}.

For ARS, we closely follow the algorithm outlined in \citep{mania2018simple}.

\subsubsection{Multi-step Control Experiments}
\label{sec:multi-step-control-1}

\paragraph{Tuning Hyperparameters for ARS}
\label{sec:tuning-hyperp-ars}

We tune the hyperparameters for ARS \citep{mania2018simple} in both
mujoco and LQR experiments, similar to the one-step control
experiments. The candidate hyperparameter values are shown in Tables
\ref{tab:hyperparam-multi-ars-mujoco} and \ref{tab:hyperparam-multi-ars-lqr}. We have observed that using all the
directions in ARS is always preferable under the low horizon settings
that we explore. Hence, we do not conduct a hyperparameter search over
the number of top directions and instead keep it the same as the
number of directions.

\begin{table}[ht]
    \centering
    \begin{tabular}{|c|c|c|}
    \hline
    \textbf{Hyperparameter} & \textbf{Swimmer-v2} &
                                                    \textbf{HalfCheetah-v2}\\
    \hline
    Stepsize &  $0.03, 0.05, 0.08, 0.1, 0.15$ & $0.001, 0.003,
                                                0.005,0.008, 0.01$ \\
    \hline
    \# Directions &  $5, 10, 20$ & $5, 10, 20$\\
    \hline
    Perturbation & $0.05, 0.1, 0.15, 0.2$ & $0.01, 0.03, 0.05, 0.08$\\
    \hline
    \end{tabular}
    \caption{Candidate hyperparameters used for tuning in ARS experiments}
    \label{tab:hyperparam-multi-ars-mujoco}
\end{table}

\begin{table}[ht]
    \centering
    \begin{tabular}{|c|c|}
    \hline
    \textbf{Hyperparameter} & \textbf{LQR}\\
    \hline
    Stepsize &  $0.0001, 0.0003, 0.0005, 0.0008, 0.001, 0.003, 0.005,
               0.008, 0.01$ \\
    \hline
    \# Directions &  $10$ \\
    \hline
    Perturbation & $0.01, 0.05, 0.1$ \\
    \hline
    \end{tabular}
    \caption{Candidate hyperparameters used for tuning in ARS experiments}
    \label{tab:hyperparam-multi-ars-lqr}
  \end{table}

We use the hyperparameters shown in Tables
\ref{tab:chosen-hyperparam-multi-ars-swimmer} and \ref{tab:chosen-hyperparam-multi-ars-halfcheetah} chosen through tuning for each
of the multi-step experiments. The hyperparameters are chosen by
averaging the total reward obtained across three random seeds
(different from the 10 random seeds used in experiments presented in
Figure~\ref{fig:multistep}) and
chosing the setting that has the highest total reward after $10000$
episodes of training..

\begin{table}[ht]
    \centering
    \begin{tabular}{|c|c|c|c|}
    \hline
      \textbf{Horizon} & \textbf{Stepsize} &
                                             \textbf{\#
                                             Directions} &
                                                           \textbf{Perturbation}\\
      \hline
      $H = 1$ & 0.15  & 5  & 0.2 \\
      \hline
      $H = 2$ & 0.08  & 5 &  0.2\\
      \hline
      $H = 3$ & 0.15  & 5 &  0.2\\
      \hline
      $H = 4$ & 0.08  & 5 & 0.2 \\
      \hline
      $H = 5$ & 0.05  & 5 &  0.2\\
      \hline
      $H = 6$ & 0.08  &5  & 0.2 \\
      \hline
      $H = 7$ & 0.08  & 5 & 0.2 \\
      \hline
      $H = 8$ & 0.08  & 5 & 0.2 \\
      \hline
      $H = 9$ & 0.1  &5  & 0.2 \\
      \hline
      $H = 10$ & 0.08  &5  & 0.2 \\
      \hline
      $H = 11$ & 0.08  &5  & 0.2 \\
      \hline
      $H = 12$ & 0.1  &5  & 0.2 \\
      \hline
      $H = 13$ & 0.08  & 5 & 0.2 \\
      \hline
      $H = 14$ & 0.08  & 5 &0.2  \\
      \hline
      $H = 15$ & 0.08  & 10 & 0.2 \\
      \hline
    \end{tabular}
    \caption{Hyperparameters chosen for multi-step experiments for ARS
    in Swimmer-v2}
    \label{tab:chosen-hyperparam-multi-ars-swimmer}
\end{table}

\begin{table}[ht]
    \centering
    \begin{tabular}{|c|c|c|c|}
    \hline
      \textbf{Horizon} & \textbf{Stepsize} &
                                             \textbf{\#
                                             Directions} &
                                                           \textbf{Perturbation}\\
%0.001, 0.008, 0.008, 0.003, 0.003, 0.003, 0.008, 0.008, 0.01 ,
       %0.005, 0.008, 0.005, 0.008, 0.01 , 0.008
      \hline
      $H = 1$ & 0.001  & 20 & 0.08 \\
      \hline
      $H = 2$ & 0.008  & 5 &  0.08\\
      \hline
      $H = 3$ &  0.008  & 10 & 0.08 \\
      \hline
      $H = 4$ &  0.003  & 5 &  0.05\\
      \hline
      $H = 5$ &  0.003  & 5 &  0.05\\
      \hline
      $H = 6$ &  0.003  & 10 &  0.05\\
      \hline
      $H = 7$ &  0.008  & 20 &  0.05\\
      \hline
      $H = 8$ &  0.008  & 5 &  0.05\\
      \hline
      $H = 9$ &  0.01  & 20 &  0.03\\
      \hline
      $H = 10$ &   0.005 & 10 &  0.03\\
      \hline
      $H = 11$ &  0.008  & 20 &  0.03\\
      \hline
      $H = 12$ &  0.005  & 5 &  0.05\\
      \hline
      $H = 13$ &  0.008  & 20 &  0.03\\
      \hline
      $H = 14$ &  0.01  & 10 &  0.03\\
      \hline
      $H = 15$ &  0.008  & 20 &  0.03\\
      \hline
    \end{tabular}
    \caption{Hyperparameters chosen for multi-step experiments for ARS
    in HalfCheetah-v2}
    \label{tab:chosen-hyperparam-multi-ars-halfcheetah}
\end{table}


\paragraph{Tuning Hyperparameters for ExAct}
\label{sec:tuning-hyperp-exact}

We tune the hyperparameters for ExAct (Algorithm
\ref{alg:random_search_action}) in both mujoco and LQR experiments,
similar to ARS. The candidate hyperparameter values are shown in
Tables \ref{tab:hyperparam-multi-exact-mujoco} and
\ref{tab:hyperparam-multi-exact-lqr}. Similar to ARS, we do not
conduct a hyperparameter search over the number of top directions and
instead keep it the same as the number of directions.

\begin{table}[H]
    \centering
    \begin{tabular}{|c|c|c|}
    \hline
    \textbf{Hyperparameter} & \textbf{Swimmer-v2} &
                                                    \textbf{HalfCheetah-v2}\\
    \hline
    Stepsize &  $0.005, 0.008, 0.01, 0.015, 0.02, 0.025, 0.03$ & $0.0001, 0.0003,
                                                0.0005,0.0008, 0.001,
                                                                 0.002,
                                                                 0.003$ \\
    \hline
    \# Directions &  $5, 10, 20$ & $5, 10, 20$\\
    \hline
    Perturbation & $0.15, 0.2, 0.3, 0.5$ &  $0.15, 0.2, 0.3, 0.5$\\
    \hline
    \end{tabular}
    \caption{Candidate hyperparameters used for tuning in ExAct experiments}
    \label{tab:hyperparam-multi-exact-mujoco}
\end{table}

\begin{table}[ht]
  \centering
  \begin{tabular}{|c|c|}
    \hline
    \textbf{Hyperparameter} & \textbf{LQR}\\
    \hline
    Stepsize &  $0.0001, 0.0003, 0.0005, 0.0008, 0.001, 0.003, 0.005,
               0.008, 0.01$ \\
    \hline
    \# Directions &  $10$ \\
    \hline
    Perturbation & $0.01, 0.05, 0.1$ \\
    \hline
  \end{tabular}
  \caption{Candidate hyperparameters used for tuning in ExAct experiments}
  \label{tab:hyperparam-multi-exact-lqr}
\end{table}

We use the hyperparameters shown in Tables
\ref{tab:chosen-hyperparam-multi-exact-swimmer} and \ref{tab:chosen-hyperparam-multi-exact-halfcheetah} chosen through tuning for
each of the multi-step experiments, similar to ARS.

\begin{table}[ht]
    \centering
    \begin{tabular}{|c|c|c|c|}
    \hline
      \textbf{Horizon} & \textbf{Stepsize} &
                                             \textbf{\#
                                             Directions} &
                                                           \textbf{Perturbation}\\
      \hline
      $H = 1$ & 0.02  &5  & 0.2 \\
      \hline
      $H = 2$ & 0.02  & 5 & 0.2 \\
      \hline
      $H = 3$ & 0.015  & 10 & 0.2 \\
      \hline
      $H = 4$ & 0.015  & 10 & 0.2 \\
      \hline
      $H = 5$ & 0.01  & 10 & 0.2 \\
      \hline
      $H = 6$ & 0.015  & 10 & 0.2 \\
      \hline
      $H = 7$ & 0.01  & 20 & 0.2 \\
      \hline
      $H = 8$ & 0.015  & 20 & 0.2 \\
      \hline
      $H = 9$ & 0.02  & 20 & 0.2 \\
      \hline
      $H = 10$ & 0.008  & 5 & 0.2 \\
      \hline
      $H = 11$ & 0.02  & 5 & 0.15 \\
      \hline
      $H = 12$ & 0.02  & 20 & 0.2 \\
      \hline
      $H = 13$ & 0.015  & 5 & 0.15 \\
      \hline
      $H = 14$ & 0.02  & 10 &0.15  \\
      \hline
      $H = 15$ & 0.01  & 5 & 0.1 \\
      \hline
    \end{tabular}
    \caption{Hyperparameters chosen for multi-step experiments for ExAct
    in Swimmer-v2}
    \label{tab:chosen-hyperparam-multi-exact-swimmer}
\end{table}

\begin{table}[ht]
    \centering
    \begin{tabular}{|c|c|c|c|}
    \hline
      \textbf{Horizon} & \textbf{Stepsize} &
                                             \textbf{\#
                                             Directions} &
                                                           \textbf{Perturbation}\\

      \hline
      $H = 1$ &0.0001   &20  &  0.2\\
      \hline
      $H = 2$ &  0.001 & 5 &  0.2\\
      \hline
      $H = 3$ &  0.001  & 5 & 0.2\\
      \hline
      $H = 4$ &  0.001  & 5 & 0.2 \\
      \hline
      $H = 5$ &  0.001  &10  & 0.2 \\
      \hline
      $H = 6$ &  0.001  & 5 & 0.2 \\
      \hline
      $H = 7$ &  0.001  &10  & 0.2 \\
      \hline
      $H = 8$ &  0.001  & 5 & 0.2 \\
      \hline
      $H = 9$ &  0.001  & 5 & 0.2 \\
      \hline
      $H = 10$ &  0.001  & 5 & 0.2 \\
      \hline
      $H = 11$ & 0.0008   & 5 & 0.15 \\
      \hline
      $H = 12$ &  0.001  & 5 & 0.2\\
      \hline
      $H = 13$ &  0.001  & 10 & 0.2 \\
      \hline
      $H = 14$ &  0.001  & 5 & 0.2\\
      \hline
      $H = 15$ &  0.0008  & 10 & 0.2 \\
      \hline
    \end{tabular}
    \caption{Hyperparameters chosen for multi-step experiments for ExAct
    in HalfCheetah-v2}
    \label{tab:chosen-hyperparam-multi-exact-halfcheetah}
\end{table}


\paragraph{Mujoco Experiments}
\label{sec:mujoco-experiments}

For all the mujoco experiments, both ARS and ExAct use a linear
policy with the same number of parameters as the dimensionality of the
state space. The hyperparameters for both algorithms are chosen as
described above. Each algorithm is run on both
environments (Swimmer-v2
and HalfCheetah-v2) for $10000$ episodes of training across $10$
random seeds (different from the ones used for tuning). This is
repeated for each horizon value $H \in \{1, 2, \cdots, 15\}$. In each
experiment, we record the mean evaluation return obtained after
training and plot the results in Figure~\ref{fig:multistep}. For more details on the environments used, we
refer the reader to \citep{brockman2016openai}.

\paragraph{LQR Experiments}
\label{sec:lqr-experiments}

In the LQR experiments, we constructed a linear dynamical system
$x_{t+1} = Ax_t + Bu_t + \xi_t$ where $x_t \in \mathbb{R}^{100}$, $A \in \mathbb{R}^{100\times
  100}$, $B \in \mathbb{R}^{100}$, $u_t \in
\mathbb{R}$ and the noise $\xi_t \sim \mathcal{N}(0_{100}, cI_{100
  \times 100})$ with a small constant $c \in \mathbb{R}^+$. We
explicitly make sure that the maximum eigenvalue of $A$ is less than 1
to avoid instability. We fix a quadratic cost function $c(x, u) =
x^TQx + uRu$, where $Q = 10^{-3}I_{100 \times 100}$ and $R = 1$. The
hyperparameters chosen for both algorithms are chosen as described
above.

For each algorithm, we run it for noise covariance values $c \in
\{10^{-4}, 5\times 10^{-4}, 10^{-3}, 5\times 10^{-3}, 10^{-2}, 5\times 10^{-2},
10^{-1}, 5\times 10^{-1}\}$
until we reach a stationary point where $\|\nabla_\theta
J(\theta)\|_2^2 \leq 0.05$. The number of interactions with the
environment allowed is capped at $10^6$ steps for each run. This is
repeated across $10$ random seeds (different from the ones used for
tuning). The number of interactions needed to reach the stationary
point as the noise covariance is increased is recorded and shown in
Figure~\ref{fig:multistep}.
\clearpage
\newpage
\section{Appendix for Chapter~\ref{CHA:CMAX}}
\label{sec:append-chapt-cmax}

\subsection{A Closer Look at the Assumption}
\label{sec:closer-look-at}

Assumption~\ref{assumption:core} requires that there exists at least
one path from the \textit{current state} to a goal state that does not
contain any transition that is known to be incorrect. In other words, it should
hold for every time step $t$ before the robot reaches the goal. Hence,
it is an assumption on both the quality of the approximate model and
the execution trace (the states visted during execution) of
\textsc{Cmax}. This makes the assumption hard to verify prior to
execution as it is
dependent on the operation of the algorithm under true dynamics.

However, we can relax the assumption in small state spaces by using the
\textit{model optimism} assumption from
\citet{DBLP:conf/aaai/Jiang18}. Specifically, an optimistic model is an
approximate model $\hat{M}$ whose optimal cost-to-go under its dynamics
$\hat{f}$ at each state $s \in \statespace$ underestimates the optimal
cost-to-go under true dynamics $f$. Notice that in small state spaces,
we can afford to do full state space planning (using value iteration,
for example) at each time step $t$, thereby obtaining the best action
under the approximate model's optimal cost-to-go. Thus, if the initial
approximate model $\hat{M}$ in \textsc{Cmax} is optimistic and we
perform full state space planning at each time step $t$, then we are
guaranteed to reach the goal. In other words, optimistic intial model
$\hat{M}$ is \textit{sufficient} for completeness if we perform full
state space planning at each time step.

Unfortunately, performing full state space planning at each time step
is computationally very expensive in large state spaces (or
intractable in continuous state spaces.) Hence, we need to resort to
limited-expansion planning or sample-based
planning~\cite{DBLP:journals/ml/KearnsMN02} which can result in
suboptimal actions. In these cases, model optimism is not sufficient
anymore, and we require assumptions such as
Assumption~\ref{assumption:core} and \ref{assumption:core-large} that
rely on the execution trace to guarantee completeness of \textsc{Cmax}.

\subsection{4D Planar Pushing Experiment Details}
\label{sec:4d-planar-pushing}

In this experiment, the task is for a robotic gripper to push a cube
from a start location to a goal location in the presence of
static obstacles without any resets, as shown in
Figure~\ref{fig:search} (right). This can be represented as a
planning problem in 4D continuous state space $\statespace$ with any state represented as
the tuple $s = (g_x,
g_y, o_x, o_y)$ where $(g_x, g_y)$ are the xy-coordinates of the
gripper and $(o_x, o_y)$ are the xy-coordinates of the object. The
model $\hat{M}$ used for planning \textit{does not} have the static obstacles and the
robot can only discover the state-action pairs that are affected due
to the obstacles through real world executions. The
action space $\actionspace$ is a discrete set of 4 actions that move
the gripper end-effector in the 4 cardinal directions by a fixed
offset using an IK-based controller. The cost of each transition is
$1$ when the object is not at the goal location, and $0$
otherwise.

For all the approaches (except Q-learning), we use
the following neural network architecture for cost-to-go
approximation: a feedforward network with 3 hidden layers each of $64$
units, the network takes as input a $15$D feature representation of the 4D
state $s = (o_x, o_y, g_x, g_y)$ that is constructed as follows:
\begin{itemize}
\item Relative position of the object w.r.t gripper $\frac{\mathbf{o}
    - \mathbf{g}}{\|\mathbf{o} - \mathbf{g}\|_2}$, where $\mathbf{o} =
  (o_x, o_y)$ is the $2$D object position and $\mathbf{g} = (g_x,
  g_y)$ is the $2$D gripper position
\item Distance between position of the object and gripper
  $\|\mathbf{o} - \mathbf{g}\|_2$
\item Relative position of the object w.r.t. goal $\frac{\mathbf{o} -
    \mathbf{t}}{\|\mathbf{o} - \mathbf{t}\|_2}$ where $\mathbf{t} =
  (t_x, t_y)$ is the $2$D goal location
\item Distance between position of the object and goal location
  $\|\mathbf{o} - \mathbf{t}\|_2$
\item Relative position of the gripper w.r.t goal $\frac{\mathbf{g} -
    \mathbf{t}}{\|\mathbf{g} - \mathbf{t}\|_2}$
\item Distance between position of the gripper and goal location
  $\|\mathbf{g} - \mathbf{t}\|_2$
\item Relative position of the object w.r.t center of the table
  $\frac{\mathbf{o} - \mathbf{c}}{\|\mathbf{o} - \mathbf{c}\|_2}$
\item Distance between position of the object and center of the table
  ${\|\mathbf{o} - \mathbf{c}\|_2}$
\item Relative position of the gripper w.r.t center of the table
  $\frac{\mathbf{g} - \mathbf{c}}{\|\mathbf{g} - \mathbf{c}\|_2}$
\item Distance between position of the gripper and center of the table
  ${\|\mathbf{g} - \mathbf{c}\|_2}$
\end{itemize}

The output of the network is a single scalar value representing the
cost-to-go of the input state. We use ReLU activations after each
layer except the last layer. Instead of learning the cost-to-go from
scratch, we start with an initial cost-to-go estimate that is
hardcoded and the neural network function approximator is used to
learn a residual on top of it. The hardcoded initial cost-to-go
estimate is obtained as follows:
\begin{itemize}
\item For the given object position, construct a target position for
  the gripper to go to as follows:
  \begin{itemize}
  \item Get the angle of the vector pointing from the object to the
    goal location: $\theta = \tan^{-1}(\frac{t_x - o_x}{t_y - o_y})$
  \item The target position for gripper is then given by $\mathbf{gt}
    = (o_x - \frac{\sin(\theta)w}{2}, o_y - \frac{\cos(\theta)w}{2})$
    where $w$ is the width of the object
  \end{itemize}
\item We compute the manhattan distance from the gripper to its target
  position $M(\mathbf{g}, \mathbf{gt})$, and from the object to the
  goal location $M(\mathbf{o}, \mathbf{t})$
\item The hardcoded heuristic is obtained as $\hat{V}(s) =
  \frac{M(\mathbf{g}, \mathbf{gt}) + M(\mathbf{o}, \mathbf{t})}{d}$,
  where $d$ is the fixed offset distance the gripper moves for each action
\end{itemize}
The residual cost-to-go function approximator is initialized in such a
way that it outputs $0$ initially for all $s \in \statespace$. We use
a similar residual Q-value function approximator for Q-learning with
the same architecture but that takes as input the above feature
representation and outputs a vector in $\reals^{|\actionspace|}$,
where each element corresponds to the Q-value for that action in the
input state. We also use hardcoded initial Q-values that are
constructed in a similar fashion $\hat{Q}(s, a) = c(s, a) +
\hat{V}(\hat{f}(s, a))$. To ensure a fair comparison across all
baselines, we use the same neural network function approximator for
cost-to-go, and start with the same initial cost-to-go
estimates.

For the model learning baseline that uses Neural network function
approximator, we use a feedforward neural network with $2$ hidden
layers each of $32$ units, the network takes as input the 4D state $s$
and a one-hot encoding of the discrete action $a$ and outputs a 4D
residual vector. The residual vector is added to the next state
predicted by the model $\hat{f}(s, a)$ to get the learned next
state. The loss function used to train the residual is mean
squared loss.

For the model learning baseline that uses KNN function approximator,
we use a radius of $0.02$, and average the next state residual vector observed
for any state within this radius to obtain the prediction for a new
state residual vector. In the same way as above, this residual vector
is added to the next state predicted by the model $\hat{f}(s, a)$ to
obtain the learned next state.

For all the neural network function approximators, we use an Adam
optimizer with learning rate of $0.001$, and an L2 regularization
constant of $0.01$. We use a batch size of $64$ for training all the
neural network function approximators. For Q-learning, we use an
random exploration probability of $\epsilon = 0.1$ and change the
target network by a polyak coefficient of $0.9$.

For all the approaches, we use a
limited expansion search planner with $K = 5$ expansions, $N = 5$
planning updates, batch size $B = 64$, and an Adam optimizer \cite{DBLP:journals/corr/KingmaB14} with
learning rate $\eta = 0.001$.

In training the cost-to-go function approximation, we use the
hindsight experience replay trick with a probability of $0.8$ for
sampling any future state in the trajectory as the desired goal. This
helps in keeping the function approximation stable and also helps in
generalization.

\subsection{3D Pick-and-Place Experiment Details}
\label{sec:3d-pick-place}

The task of this physical robot experiment (Figure~\ref{fig:real-3d})
is to pick and place a heavy object using a PR2 arm from a start pick
location to a goal place
location while avoiding an obstacle. This can be represented as a
planning problem in 3D discrete state space $\statespace$ where
each state corresponds to the 3D location of the end-effector. In our
experiment, we discretize each dimension into $20$ bins and plan in
the resulting discrete state space of size $20^3$. Since it is a
relatively small state space, we use exact planning updates without
any function approximation following
Algorithm~\ref{alg:small-state-spaces} with $K=3$ expansions. The action space is a
discrete set of $6$ actions corresponding to a fixed offset movement
in positive or negative direction along each dimension. We use a
RRT-based motion planner \cite{DBLP:journals/ijrr/LaValleK01} to plan the path of the
arm between states, while avoiding collision with the obstacle. The
model $\hat{M}$ used by planning \textit{does not} model the object as
heavy and hence, does not capture the dynamics of the arm correctly when it
holds the heavy object. The cost of each transition is $1$ if
object is not at the goal place location, otherwise it is $0$.

\subsection{7D Arm Planning Experiment Details}
\label{sec:7d-arm-planning}

The task of this physical robot experiment (Figure~\ref{fig:real-7d}) is
to move the PR2 arm with a
non-operational joint from a start configuration so that the
end-effector reaches a goal location, specified as a 3D
region. We represent this as a planning problem in 7D
discrete statespace $\statespace$ where each dimension corresponds to
a joint of the arm bounded by its joint limits. Each dimension is
discretized into $10$ bins resulting in a large state space of
size $10^7$. The action space
$\actionspace$ is a discrete set of size
$14$ corresponding to moving each joint by a fixed offset in the
positive or negative direction. We use an IK-based controller to
navigate between discrete states. The model $\hat{M}$ used for
planning \textit{does not} know that a joint is non-operational and
assumes that the arm can attain any configuration within the joint
limits. In the real world, if the robot tries to move the
non-operational joint, the arm does not move. Thus, the robot realizes
unreachable states only through real world executions.

\clearpage
\newpage
\section{Appendix for Chapter~\ref{CHA:CMAXPP}}
\label{sec:append-chapt-cmaxpp}

\subsection{Sensitivity Experiments}
\label{sec:sens-exper}

In this section, we present the results of our sensitivity experiments
examining the performance of \acmaxpp{} with the choice of the
sequence $\{\alpha_i\}$. We compare the performance of different
choices of the sequence $\{\alpha_i\}$ on the $3$D mobile robot
navigation task. For each run, we average the results across $5$
instances with randomly placed ice patches and present the mean and
standard errors. To keep the figures concise, we plot the cumulative
number of steps taken to reach the goal from the start of the first
lap to the current lap across all laps. In all our runs, \acmaxpp{}
successfully completes all $200$ laps and hence, we do not report the
number of successful instances in our results.

We choose $4$ schedules for the sequence $\{\alpha_i\}$:
\begin{enumerate}
\item \textbf{Exponential Schedule}: In this schedule, we vary
  $\beta_{i+1} = \rho\beta_i$ where $\rho < 1$ is a constant that is
  tuned and $\alpha_i = 1 + \beta_i$. Observe that as $i\rightarrow
  \infty$, $\alpha_i \rightarrow 1$ and that the sequence
  $\{\alpha_i\}$ is a decreasing sequence.

  We vary both the initial $\beta_1$ chosen and the constant $\rho$ in
  our experiments. For $\beta_1$ we choose among values $[10, 100,
  1000]$ and $\rho$ is chosen among $[0.5, 0.7, 0.9]$. The results are
  shown in Figure~\ref{fig:exp}.
  \begin{figure}
  \centering
  \includegraphics[width=0.5\linewidth]{figures/cmaxpp/alpha_exp.pdf}
  \caption{Sensitivity experiments with an exponential schedule}
  \label{fig:exp}
\end{figure}

All choices have almost the same performance with $\beta_1 = 1000$ and
$\rho = 0.9$ having the best performance initially but has slightly
worse performance in the last several laps. The choice of $\beta_1 =
100$ and $\rho = 0.9$  seems to be a good choice with great performance in both
initial and final laps.

\item \textbf{Linear Schedule}: In this schedule, we vary $\beta_{i+1}
  = \beta_i - \eta$ where $\alpha_i = 1 + \beta_i$ and $\eta > 0$ is a
  constant that is determined
  so that $\beta_{200} = 0$, i.e. $\alpha_{200} = 1$. Hence, we have
  $\eta = \frac{\beta_1}{200}$.

  We vary the initial $\beta_1$ and choose among values $[10, 100,
  200]$. The results are shown in Figure~\ref{fig:linear}.
  \begin{figure}
  \centering
  \includegraphics[width=0.5\linewidth]{figures/cmaxpp/alpha_linear.pdf}
  \caption{Sensitivity experiments with a linear schedule}
  \label{fig:linear}
\end{figure}

All three choices have the same performance except in the last few
laps where $\beta_1 = 10$ degrades while the other two choices perform well.
\item \textbf{Time Decay Schedule}: In this schedule, we vary
  $\beta_{i+1} = \frac{\beta_1}{i+1}$ where $\alpha_i = 1 +
  \beta_i$. In other words, we decay $\beta$ at the rate of
  $\frac{1}{i}$ where $i$ is the lap number. Again, observe that as $i
  \rightarrow \infty$, we have $\alpha_i \rightarrow 1$.

  We vary the initial $\beta_1$ and choose among values $[10, 100,
  1000]$. The results are shown in Figure~\ref{fig:time}.
  \begin{figure}
  \centering
  \includegraphics[width=0.5\linewidth]{figures/cmaxpp/alpha_time.pdf}
  \caption{Sensitivity experiments with a time decay schedule}
  \label{fig:time}
\end{figure}

The choices of $\beta_1 = 100$  and $\beta_1 = 1000$ have the best
(and similar) performance while $\beta_1 = 10$ has a poor performance
as it quickly switches to \cmaxpp{} in the early laps and wastes
executions learning accurate $Q$-values.
\item \textbf{Step Schedule}: In this schedule, we vary $\beta$ as a
  step function with $\beta_{i+1} = \beta_i - \delta$ if $i$ is a
  multiple of $\xi$ where $\xi$ is the step frequency, $\alpha_i = 1 +
  \beta_i$ and $\delta$ is a constant that is determined so that
  $\beta_{200} = 0$, i.e. $\alpha_{200} = 1$. Hence, we have $\delta =
  \frac{\beta_1\xi}{200}$.

  We vary both the initial $\beta_1$ and the step frequency $\xi$. For
  $\beta_1$ we choose among values $[10, 100, 200]$ and for $\xi$ we
  choose among $[5, 10, 20]$. The results are shown in
  Figure~\ref{fig:step}.
  \begin{figure}
  \centering
  \includegraphics[width=0.5\linewidth]{figures/cmaxpp/alpha_step.pdf}
  \caption{Sensitivity experiments with a step schedule}
  \label{fig:step}
\end{figure}

All choices have the same performance and \acmaxpp{} seems to be
robust to the choice of step size frequency.
\end{enumerate}

For our final comparison, we will pick the best performing choice
among all the schedules and compare performance among these selected
choices. The results are shown in Figure~\ref{fig:best}.

\begin{figure}
  \centering
  \includegraphics[width=0.5\linewidth]{figures/cmaxpp/alpha_best.pdf}
  \caption{Sensitivity experiments with best choices among all
    schedules}
  \label{fig:best}
\end{figure}

We can observe that all schedules have the same performance except the
exponential schedule which has worse performance. This can be
attributed to the rapid decrease in the value of $\beta$ compared to
other schedules and thus, around lap $50$ \acmaxpp{} switches to
\cmaxpp{} resulting in a large number of executions wasted to learn
accurate $Q$-value estimates. This does not happen for other schedules
as they decrease $\beta$ gradually and thus, spreading out the
executions used to learn accurate $Q$-value estimates across several
laps and not performing poorly in any single lap.

\subsection{Experiment Details}
\label{sec:experiment-details}

All experiments were implemented using Python 3.6 and run on a
$3.1$GHz Intel Core i$5$ machine. We use
PyTorch~\cite{NEURIPS2019_9015} to train neural network function
approximators in our $7$D experiments, and use Box2D~\cite{catto2007box2d} for our 3D mobile
robot simulation (similar to OpenAI Gym~\cite{brockman2016openai}
\texttt{car\_racing} environment) and use
PyBullet~\cite{coumans2013bullet} for our $7$D PR2 experiments.

\subsubsection{3D Mobile Robot Navigation with Icy Patches}
\label{sec:3d-mobile-robot}

\begin{figure}[t]
  \centering
  \includegraphics[width=0.5\linewidth]{figures/cmaxpp/race_track_full_pic.png}
  \caption{$3$D Mobile Robot experiment example track}
  \label{fig:track}
\end{figure}

An example track used in the $3$D experiment is shown in
Figure~\ref{fig:track}. We generate $66$ motion primitives offline
using the following procedure: (a) We first define the primitive
action set for the robot by discretizing the steering angle into 3
cells, one corresponding to zero and the other two corresponding to
$+0.6$ and $-0.6$ radians. We also discretize the speed of the robot
to 2 cells corresponding to $+2$m/s and $-2$m/s, (b) We then
discretize the state space into a $100\times 100$ grid in $XY$ space
and $16$ cells in $\theta$ dimension. Thus, we have a $100 \times 100
\times 16$ grid in $XY\theta$ space., (c) We then initialize the robot
at $(0, 0)$ $xy$ location with different headings chosen among $[0,
\cdots, 15]$ and roll out all possible sequences of primitive actions
for all possible motion primitive lengths from $1$ to $15$ time steps,
(d) We filter out all motion primitives whose end point is very close
to a cell center in the $XY\theta$ grid. During execution, we use a
pure pursuit controller to track the motion primitive so that the robot
always starts and ends on a cell center. During planning, we simply
use the discrete offsets stored in the motion primitive to compute the
next state (and thus, the model dynamics are pre-computed offline
during motion primitive generation.)

The cost function used is as follows: for any motion primitive $a$ and
state $s$,
the cost of executing $a$ from $s$ is given by $c(s, a) = \sum_{s'}
c'(s')$ where $c'$ is a pre-defined cost map over the $100 \times 100
\times 16$ grid and $s'$ is all the intermediate states (including the
final state) that the robot
goes through while executing the motion primitive $a$  from $s$. The
pre-defined cost map is defined as follows: $c'(s) = 1$ if state $s$
lies on the track (i.e. $xy$ location corresponding to $s$ lies on the
track) and $c'(s) = 100$ otherwise (i.e. all $xy$ locations
corresponding to grass or wall has a cost of $100$). This encourages
the planner to come up with a path that lies completely on the track.

We define two checkpoints on the opposite ends of the track (shown as
blue squares in Figure~\ref{fig:track}.) The goal of the robot is to
reach the next checkpoint incurring least cost while staying on the
track. Note that this
requires the robot to complete laps around the track as quickly as
possible. Since the state space is small, we maintain value estimates
$V, Q, \tilde{V}$ using tables and update the appropriate table entry
for each value update. The tables are initialized with value estimates
obtained by planning in the model $\Mhat$ using a planner with $K=100$
expansions until the robot can efficiently complete the laps using the
optimal paths. However, this does not mean that the initial value
estimates are the optimal values for $\Mhat$ dynamics since the
planner looks ahead and can achieve optimal paths with underestimated
value functions. Nevertheless, these estimates are highly informative.

\subsubsection{7D Pick-and-Place with a Heavy Object}
\label{sec:7d-pick-place}

\begin{figure}[t]
  \centering
  \includegraphics[width=0.5\linewidth]{figures/cmaxpp/intro_grasp_new.png}
  \caption{$7$D Pick-and-Place Experiment}
  \label{fig:pr2}
\end{figure}

For our $7$D experiments, we make use of Bullet Physics Engine through
the pyBullet interface. For motion planning and other simulation
capabilities we make use of \texttt{ss-pybullet}
library~\cite{sspybullet}. The task is shown in
Figure~\ref{fig:pr2}. The goal is for the robot to pick the heavy
object from its start pose and place it at its goal pose while
avoiding the obstacle, without any resets. Since the object is heavy,
the robot fails to lift the object in certain configurations where it
cannot generate the required torque to lift the object. Thus, the
robot while lifting the object might fail to reach the goal waypoint
and onky reach an intermediate waypoint resulting in discrepancies
between modeled and true dynamics.


This is represented as a planning problem in $7$D statespace. The
first $6$ dimensions correspond to the $6$DOF pose of the object (or
gripper,) and the last dimension corresponds to the redundant DOF in
the arm (in our case, it is the upper arm roll joint.) Given a $7$D
configuration, we use IKFast library~\cite{diankov_thesis} to compute
the corresponding $7$D joint angle configuration. The action space
consists of $14$ motion primitives that move the arm by a fixed offset
in each of the $7$ dimensions in positive and negative directions. The
discrepancies in this experiment are only in the $Z$ dimension
corresponding to lifting the object. For planning, we simply use a
kinematic model of the arm and assume that the object being lifted is
extremely light. Thus, we do not need to explicitly consider dynamics
during planning. However, during execution we take the dynamics into
account by executing the motion primitives in the simulator. The cost
of any transition is $1$ if the object is not at goal pose, $0$ if the
object is at goal pose. We start the next repetition only if the robot
reached the goal pose in the previous repetition.


The $7$D state space is discretized into $10$ cells in each dimension
resulting in $10^7$ states. Since the state space is large we use
neural network function approximators to maintain the value functions
$V, Q, \tilde{V}$. For the state value functions $V, \tilde{V}$ we use
the following neural network approximator: a feedforward network with
$3$ hidden layers consisting of $64$ units each, we use ReLU
activations after each layer except the last layer, the network takes
as input a $34$D feature representation of the $7$D state computed as
follows:

\begin{itemize}
	\item For any discrete state $s$, we compute a continuous
          $10$D representation $r(s)$ that is used to construct the
          features

	\begin{itemize}
		\item The discrete state is represented as $(xd, yd,
                  zd, rd, pd, yd, rjointd)$ where $(xd, yd, zd)$
                  represents the $3$D discrete location of the object
                  (or gripper,) $(rd, pd, yd)$ represents the discrete
                  roll, pitch, yaw of the object (or gripper,) and
                  $rjointd$ represents the discrete redundant joint
                  angle

		\item We convert $(xd, yd, zd)$ to a continuous
                  representation by simply dividing by the grid size
                  in those dimensions, i.e. $(xc, yc, zc) = (xd/10,
                  yd/10, zd/10)$

		\item We do a similar construction for $rjointc$,
                  i.e. $rjointc = rjointd/10$

		\item However, note that $rd, pd, yd$ are angular
                  dimensions and simply dividing by grid size would
                  not encode the wrap around nature that is inherent
                  in angular dimensions (we did not have this problem
                  for $rjointd$ as the redundant joint angle has lower
                  and upper limits, and is always recorded as a value
                  between those limits.) To account for this, we use a
                  sine-cosine representation defined as $(rc1, rc2,
                  pc1, pc2. yc1, yc2) = (sin(rc), cos(rc), sin(pc),
                  cos(pc), sin(yc), cos(yc))$ where $rc, pc, yc$ are
                  the roll, pitch, yaw angles corresponding to the
                  cell centers of the grid cells $rd, pd, yd$.

		\item Thus, the final $10$D representation of state
                  $s$ is given by $r(s) = (xc, yc, zc, rc1, rc2, pc1,
                  pc2, yc1, yc2, rjointc)$

		\item We also define a truncated $9$D representation
                  $r'(s) = (xc, yc, zc, rc1, rc2, pc1, pc2, yc1, yc2)$
                  and a $3$D representation $r''(s) = (xc, yc, zc)$

	\end{itemize}
	\item The first feature is the $9$D relative position of the $6$D goal pose w.r.t the object $f1 = r'(g) - r'(s)$
	\item The second feature is the $10$D relative position of the object w.r.t the gripper home state $h$, $f2 = r(s) - r(h)$
	\item The third feature is the $9$D relative position of the goal w.r.t the gripper home state $h$, $f3 = r'(g) - r'(h)$
	\item The fourth feature is the $3$D relative position of the obstacle left top corner $o1$ w.r.t the object, $f4 = r''(o1) - r''(s)$
	\item The fifth and final feature is the $3$D relative position of the object right bottom corner $o2$ w.r.t. the object, $f5 = r''(o2) - r''(s)$
	\item Thus, the final $34$D feature representation is given by $f(s) = (f1, f2, f3, f4, f5)$.
\end{itemize}

The output of the network is a single scalar value representing the
cost-to-goal of the input state. Instead of learning the
cost-to-goal/value from scratch, we start with an initial value
estimate that is hardcoded (manhattan distance to goal in the $7$D
discrete grid) and the neural network approximator is used to learn a
residual on top of it. A similar trick was used in
\cmax{}~\cite{cmax}. The residual state value function
approximator was initialized to output $0$ for all $s \in
\statespace$. We use a similar architecture for the residual $Q$-value
function approximator but it takes as input the $34$D state feature
representation and outputs a vector in $\mathbb{R}^{|\actionspace|}$
(in our case, $\mathbb{R}^{14}$) to represent the cost-to-goal
estimate for each action $a \in \actionspace$. We also use the same
hardcoded value estimates as before in addition to the residual
approximator to construct the $Q$-values. All baselines and proposed
approaches use the same function approximator and same initial
hardcoded value estimates to ensure fair comparison. The value
function approximators are trained using mean squared loss.


The residual model learning baseline with neural network (NN) function
approximator uses the following architecture: $2$ hidden layers each
with $64$ units and all layers are followed by ReLU activations except
the last layer. The input of the network is the $34$D feature
representation of the state and a one-hot encoding of the action in
$\mathbb{R}^14$. The output of the network is the $7$D continuous
state which is added to the state predicted by the model $\Mhat$. The
loss function used to train the network is a simple mean squared
loss. The residual model learning baseline with K-Nearest Neighbor
regression approximator (KNN) uses a manhattan radius of $3$ in the
discrete $7$D state space. We compute the prediction by averaging the
next state residual vector observed in the past for any state that
lies within the radius of the current state. The averaged residual is
added to the next state predicted by model $\Mhat$ to obtain the
learned next state.


We use Adam optimizer~\cite{DBLP:journals/corr/KingmaB14} with a
learning rate of $0.001$ and a weight decay (L$2$ regularization
coefficient) of $0.001$ to train all the neural network function
approximators in all approaches. We use a batch size of $32$ for the
state value function approximators and a batch size of $128$ for the
$Q$-value function approximators. We perform $U = 3$ updates for state
value function and $U = 5$ updates for state-action value function for
each time step. We update the parameters of all neural network
approximators using a polyak averaging coefficient of $0.5$.


Finally, we use hindsight experience replay
trick~\cite{DBLP:conf/nips/AndrychowiczCRS17} in training all the
value function approximators with the probability of sampling any
future state in past trajectories as the goal set to $0.7$. This is
crucial as our cost function used is extremely sparse.

\clearpage
\newpage
\section{Appendix for Chapter~\ref{CHA:ILC}}
\label{sec:append-chapt-ilc}

\subsection{General Results}
\label{sec:general-results-1}

In this section, we will present general results that bound the cost suboptimality of
any time-varying controller $\Khat$ in terms of the norm differences
$||\KOPT_{t} - \Khat_{t}||$. Our first lemma makes use of
Assumption~\ref{assumption:stability} to show that if the norm differences
$||\KOPT_{t} - \Khat_{t}||$ are small, then the true system can be stable under
$\Khat$:
\begin{restatable}{lemma}{stabilityLemma}
  \label{lemma:stability}
  If Assumption~\ref{assumption:stability} holds and if $\Khat$ satisfies
  $||\KOPT_{i} - \Khat_{i}|| \le \frac{\delta}{2||B_{i}||}$ for all
  $i \in \{0, \cdots, H-1\}$, then we have
  \begin{equation}
    \label{eq:1}
    ||L_{t}(\Khat)|| \leq \left(1 - \frac{\delta}{2}\right)^{t+1} \leq e^{-\frac{\delta}{2}(t+1)}
  \end{equation}
\end{restatable}
\begin{proof}
  Observe that,
\begin{align*}
  ||L_t(\Khat)|| &= ||\prod_{i=0}^t M_i(\Khat)|| = ||\prod_{i=0}^t A_i
                   + B_i\Khat_i|| \\
  &= ||\prod_{i=0}^t A_i + B_i\KOPT_i + B_i(\Khat_i - \KOPT_i)|| =
    ||\prod_{i=0}^t M_i(\KOPT) + \Delta_i||
\end{align*}
where $\Delta_i = B_i(\Khat_i - \KOPT_i)$. Since the spectral norm is
sub-multiplicative we can see that
\begin{align*}
  ||\prod_{i=0}^t M_i(\KOPT) + \Delta_i|| &\leq \prod_{i=0}^t||M_i(\KOPT) +
                                        \Delta_i|| \\
  &\leq \prod_{i=0}^t(||M_i(\KOPT)|| + ||\Delta_i||)
\end{align*}
where we used the triangle inequality. Now note that
$||M_i(\KOPT)|| \leq 1-\delta$ from assumption~\ref{assumption:stability},
\begin{align*}
  \|\Delta_{i}\| &= \|B_{i}(\Khat_{i} - \KOPT_{i}) \| \leq \|B_{i}\|\|\Khat_{i} - \KOPT_{i}\| \\
                 &\leq \kappa \frac{\delta}{2\kappa} \\
  &\leq \frac{\delta}{2}
\end{align*}
The last inequality above is from our assumption on model errors in the lemma statement.
Combining all of this above, we get
\begin{align*}
  ||L_t(\Khat)|| \leq \prod_{i=0}^t\left(1 - \delta + \frac{\delta}{2}\right) \leq \left( 1-\frac{\delta}{2} \right)^{t+1}
\end{align*}
\end{proof}

The next lemma is very similar to the performance difference lemma that was
first proposed in~\cite{kakade2002approximately}. We borrow the version presented
in~\cite{fazel18} and extend it to the finite horizon setting below:
\begin{lemma}
  \label{lemma:performance-difference}
  Let $\xhat_0,
  \uhat_0, \cdots, \xhat_H, \uhat_H$ be the trajectory generated by
  controller
  $\Khat$ using the true dynamics such that $\xhat_0 = x_0$, $\uhat_t = \Khat_t\xhat_t$ for
  $t=0, \cdots, H-1$. Then
  we have:
  \begin{equation}
    \label{eq:2}
    \Vhat_0(x_0) - \VOPT_0(x_0) = \sum_{t=0}^{H-1} \AOPT_t(\xhat_t, \uhat_t) - \VOPT_H(\xhat_H)
  \end{equation}
  where $\Vhat_t$ is the cost-to-go using controller $\Khat$ from time
  step $t$, $\VOPT_t$ is the cost-to-go using the optimal controller $\KOPT$ from time
  step $t$, and $\AOPT_t(x, u) = \QOPT_t(x, u) - \VOPT_t(x)$ is the advantage of
  the controller $\KOPT$ at time step $t$. Furthermore, we have that for any $x$
  \begin{equation*}
    %\label{eq:3}
    \AOPT_t(x, \Khat_tx) = x^T(\Khat_t - \KOPT_t)^T(R + B_t^T\POPT_{t+1}B_t)
    (\Khat_t - \KOPT_t)x
  \end{equation*}
\end{lemma}
\begin{proof}
  For proof, we refer the readers to~\cite{fazel18}.
\end{proof}

We use the performance difference lemma, as stated above, in the finite horizon
LQR setup and make use of Lemma~\ref{lemma:stability} to establish the
suboptimality bound in terms of the norm differences $||\KOPT_{t} - \Khat_{t}||$:
\costTheorem*
\begin{proof}
  From Lemma~\ref{lemma:performance-difference} we have
  \begin{align*}
    A_{t}(\xhat_{t}, \Khat_{t}\xhat_{t}) &= \xhat_{t}^{T}(\Khat_{t} - \KOPT_{t})^{T}(R + B_{t}^{T}P_{t+1}B_{t})(\Khat_{t} - \KOPT_{t})\xhat_{t}
  \end{align*}
  We know $\xhat_{t} = L_{t-1}(\Khat)x_{0}$ and using the trace identity we get
  \begin{align*}
    A_{t}(\xhat_{t}, \Khat_{t}\xhat_{t}) &= \Tr(L_{t-1}(\Khat)x_{0}x_{0}^{T}(L_{t-1}(\Khat))^{T}(\Khat_{t} - \KOPT_{t})^{T}(R + B_{t}^{T}P_{t+1}B_{t})(\Khat_{t} - \KOPT_{t})) \\
                                         &\leq \|L_{t-1}(\Khat)x_{0}x_{0}^{T}\| \|R + B_{t}^{T}P_{t+1}B_{t}\| \|\Khat_{t} - \KOPT_{t}\|_{F}^{2} \\
    &\leq \|L_{t-1}(\Khat)\|^{2}\|x_{0}\|^{2} \|R + B_{t}^{T}P_{t+1}B_{t}\| \|\Khat_{t} - \KOPT_{t}\|_{F}^{2}
  \end{align*}
  We can bound $\|R + B_{t}^{T}P_{t+1}B_{t}\| \leq \Gamma^{3}$ and
  $\|\Khat_{t} - \KOPT_{t}\|_{F}^{2} \leq \min\{n, d\} \|\Khat_{t} - \KOPT_{t}\|^{2}$.
  We can also use Lemma~\ref{lemma:stability} to bound
  $\|L_{{t-1}}(\Khat)\| \leq \exp\left(-\frac{\delta}{2}t\right)$.
  Combining all
  of this above we get
  \begin{align*}
    A_{t}(\xhat_{t}, \Khat_{t}\xhat_{t}) &\leq \min\{n, d\}\Gamma^{3} \exp\left(-\delta t\right)\|\Khat_{t} - \KOPT_{t}\|^{2}\|x_{0}\|^{2}
  \end{align*}
  Summing over all time steps we
  obtain (using $d \leq n$)
  \begin{align*}
    \Vhat_0(x_0) - V_0(x_0) &\leq d\Gamma^{3}\|x_{0}\|^{2} \sum_{t=0}^{H-1} \exp\left(-\delta t\right)\|\Khat_{t} - \KOPT_{t}\|^{2}
  \end{align*}
\end{proof}

\subsection{Helpful Lemmas}
\label{sec:helpful-lemmas}

Before we dive into the results, let us present a helpful lemma
borrowed from \cite{mania19}:
\begin{lemma}
  Let $f_1, f_2$ be $\mu$-strongly convex twice differentiable
  functions. Let $x_1 = \arg\min_x f_1(x)$ and $x_2 = \arg\min_x
  f_2(x)$. Suppose $||\nabla f_1(x_2)|| \leq \epsilon$, then $||x_1 -
  x_2|| \leq \frac{\epsilon}{\mu}$
  \label{lemma:1}
\end{lemma}
\begin{proof}
  Taylor expanding $\nabla f_{1}$ we get
  \begin{align*}
    \nabla f_{1}(x_{2}) &= \nabla f_{1}(x_{1}) + \nabla^{2}f_{1}(\tilde{x})(x_{2} - x_{1}) \\
    &= \nabla^{2}f_{1}(\tilde{x})(x_{2} - x_{1})
  \end{align*}
  for some $\tilde{x} = tx_{1}+ (1-t)x_{2}$ where $t \in [0, 1]$. Thus we have
  \begin{align*}
    \|\nabla f_{1}(x_{2})\| = \|\nabla^{2}f_{1}(\tilde{x})(x_{2} - x_{1})\| \leq \epsilon
  \end{align*}
  But we know $\|\nabla^{2}f_{1}(\tilde{x})\| \geq \mu$ which gives us
  \begin{align*}
    \|x_{2} - x_{1}\| \leq \frac{\epsilon}{\mu}
  \end{align*}
\end{proof}

The next lemma is a useful fact about positive semi-definite matrices,
also from \cite{mania19},
\begin{lemma}
  \label{lemma:mania-appendix-original}
  Given matrices $A, \Ahat$ such that $\|A - \Ahat\| \leq \epsA$,
  and positive-semidefinite matrices $Q, S, \Shat$ we have
  \begin{equation}
    \label{eq:89}
    \|A^{T}Q(I + SQ)^{-1}A - \Ahat^{T}Q(I + \Shat Q)^{-1}\Ahat\| \leq \|A\|^{2}\|Q\|^{2}\|\Shat - S\| + 2\|A\|\|Q\|\epsA + \|Q\|\epsA^{2}
  \end{equation}
\end{lemma}
\begin{proof}
  We can rewrite the expression,
  \begin{align*}
    A^{T}Q(I + SQ)^{-1}A &- \Ahat^{T}Q(I + \Shat Q)^{-1}\Ahat = \\
                         &A^{T}Q(I + SQ)^{-1}(\Shat - S)Q(I+\Shat Q)^{-1}A - A^{T}Q(I + \Shat Q)^{-1}(\Ahat - A) \\
    &-(\Ahat - A)^{T}Q(I + \Shat Q)^{-1}A - (\Ahat - A)^{T}Q(I + \Shat Q)^{-1}(\Ahat - A)
  \end{align*}

  Now we make use of Lemma 7 from \cite{mania19} which states that for any two
  positive semidefinite matrices $M, N$ of the same dimension, we have
  $\|N(I + MN)^{-1}\| \leq \|N\|$. Thus, we have $\|Q(I+SQ)^{-1}\| \leq \|Q\|$
  and $\|Q(I + \Shat Q)^{-1}\| \leq \|Q\|$.

  Using the above facts we get,
  \begin{align*}
    \|A^{T}Q(I + SQ)^{-1}A - \Ahat^{T}Q(I + \Shat Q)^{-1}\Ahat\| &\leq \|A\|^{2}\|Q\|^{2}\|\Shat - S\| + 2\|A\|\|Q\|\epsA + \|Q\|\epsA^{2}
  \end{align*}
\end{proof}

Finally, we have a lemma that will be useful in proving ricatti
perturbation bounds,
\begin{lemma}
  Given positive semidefinite matrices $N_{1}, N_{2}, M$ of the same dimensions,
  we have
  \begin{equation}
    \label{eq:90}
    ||N_{1}(I + MN_{1})^{-1} - N_{2}(I + MN_{2})^{-1}|| \leq ||(I + MN_{1})^{-1}|| ||N_{1} - N_{2}|| ||(I + MN_{2})^{-1}||
  \end{equation}
  \label{lemma:stackexchange}
\end{lemma}
\begin{proof}
  We can rewrite the expression as,
  \begin{align*}
    &N_1(I+MN_1)^{-1}-N_2(I+MN_2)^{-1}\\
    &=\left[N_1(I+MN_1)^{-1}-N_1(I+MN_2)^{-1}\right]
    +\left[N_1(I+MN_2)^{-1}-N_2(I+MN_2)^{-1}\right]\\
    &=N_1(I+MN_1)^{-1}\left[(I+MN_2)-(I+MN_1)\right](I+MN_2)^{-1}
    +(N_1-N_2)(I+MN_2)^{-1}\\
    &=N_1(I+MN_1)^{-1}M(N_2-N_1)(I+MN_2)^{-1}
    +(N_1-N_2)(I+MN_2)^{-1}\\
    &=\left[I-N_1(I+MN_1)^{-1}M\right](N_1-N_2)(I+MN_2)^{-1} \\
    &=(I + N_{1}M)^{{-1}}(N_1-N_2)(I+MN_2)^{-1}
  \end{align*}
  The rest follows by taking norm on both sides, and using the submultiplicative
  property of the induced norm.
  \begin{align*}
    ||N_{1}(I + MN_{1})^{-1} - N_{2}(I + MN_{2})^{-1}|| &\leq ||(I + N_{1}M)^{-1}|| ||N_{1} - N_{2}|| ||(I + MN_{2})^{-1}|| \\
                                                        &= ||(I + N_{1}M)^{-T}||||N_{1} - N_{2}|| ||(I + MN_{2})^{-1}|| \\
    &= ||(I + MN_{1})^{-1}||||N_{1} - N_{2}|| ||(I + MN_{2})^{-1}||
  \end{align*}
\end{proof}

\subsection{Optimal Control with Misspecified Model Results}
\label{sec:cert-equiv-contr-2}

The next lemma, from \cite{mania19}, applies the above result to quadratic functions that
are observed in linear quadratic control:
\begin{lemma}
  \label{lemma:ce-quadratic}
  Define $f_1(x, u) = \frac{1}{2}u^TRu +
  \frac{1}{2}(A_1x+B_1u)^TP_1(A_1x+B_1u)$ and similarly define $f_2(x,
  u)$ where $R, P_1, P_2$ are positive-definite matrices. Let $K_1$ be
  such that $u_1 = \arg\min_u f_1(x, u) = K_1x$  for
  any vector $x$. Define the matrix $K_2$ in a similar fashion. Also,
  denote $\Gamma = 1 + \max\{||A_1||, \allowbreak||B_1||, ||P_1||,
  ||K_1||\}$. Suppose there exists $\epsA, \epsB, \epsP > 0$ (and
  $<\Gamma$) such that
  $||A_1 - A_2|| \leq \epsA$, $||B_1 - B_2|| \leq
  \epsB$, and $||P_1 - P_2|| \leq \epsP$. Then we have,
  \begin{equation}
    \label{eq:73}
    \|K_{1} - K_{2}\| \leq \frac{\Gamma^2\epsA + (3\Gamma^3 +
      2\Gamma^2)\epsB + 4(\Gamma^3 + \Gamma^2)\epsP}{\ubar{\sigma}(R)}
  \end{equation}
\end{lemma}

\begin{proof}
  Consider
  \begin{align*}
    \nabla_{u} f_{1}(x, u) &= (B_{1}^{T}P_{1}B_{1} + R)u + B_{1}^{T}P_{1}A_{1}x \\
    \nabla_{u} f_{2}(x, u) &= (B_{2}^{T}P_{2}B_{2} + R)u + B_{2}^{T}P_{2}A_{2}x
  \end{align*}

  Let us bound the difference $\|\nabla_{u} f_{1}(x,u) - \nabla_{u} f_{2}(x, u)\|$
  by bounding each term separately. First consider the term
  \begin{align*}
    \|B_{1}^{T}P_{1}B_{1} - B_{2}^{T}P_{2}B_{2}\| &= \|B_{1}^{T}P_{1}(B_{1} - B_{2}) + (B_{1} - B_{2})^{T}P_{1}B_{2} + B_{2}^{T}(P_{1} - P_{2})B_{2}\| \\
                                                  &\leq \|B_{1}^{T}P_{1}(B_{1} - B_{2})\| + \|(B_{1} - B_{2})^{T}P_{1}B_{2}\| + \|B_{2}^{T}(P_{1} - P_{2})B_{2}\| \\
                                                  &\leq \Gamma^{2}\epsB + \Gamma\epsB(\Gamma + \epsB) + (\Gamma + \epsB)^{2}\epsP \\
    &\leq \Gamma^2(3\epsB + 4\epsP)
  \end{align*}
  where we used the fact that $\|B_{2}\| \leq \Gamma + \epsB$.
  We can similarly bound the term
  \begin{align*}
    \|B_{1}^{T}P_{1}A_{1} - B_{2}^{T}P_{2}A_{2}\| \leq
    \Gamma^{2}(\epsA + 2\epsB + 4\epsP)
  \end{align*}


  Thus, we have for any vector $x$ such that $\|x\| \leq 1$
  \begin{align*}
    \|\nabla_{u} f_{1}(x,u) - \nabla_{u} f_{2}(x, u)\| \leq
    \Gamma^2(3\epsB + 4\epsP)\|u\| + \Gamma^2(\epsA + 2\epsB + 4\epsP)
  \end{align*}
  Substituting $u=u_{1}$ we get
  \begin{align*}
    \|\nabla_{u} f_{2}(x, u_{1})\| \leq \Gamma^2(3\epsB + 4\epsP)\|u_1\| + \Gamma^2(\epsA + 2\epsB + 4\epsP)
  \end{align*}

  We can bound $\|u_{1}\| \leq \|K_{1}\|\|x\| \leq \|K_{1}\| \leq \Gamma$. Then
  from Lemma~\ref{lemma:1} we have,
  \begin{align*}
    \|u_{1} - u_{2}\| &\leq \frac{\Gamma^3(3\epsB + 4\epsP) +
                        \Gamma^2(\epsA + 2\epsB + 4\epsP)}{\ubar{\sigma}(R)} \\
    \|K_{1} - K_{2}\| &\leq \frac{\Gamma^2\epsA + (3\Gamma^3 +
      2\Gamma^2)\epsB + 4(\Gamma^3 + \Gamma^2)\epsP}{\ubar{\sigma}(R)}
  \end{align*}
\end{proof}

Now we will prove Lemma~\ref{lemma:ce},
\ceLemma*
\begin{proof}
  Use Assumption~\ref{assumption:singularvalue} and
  Lemma~\ref{lemma:ce-quadratic} for every $t=0, \cdots, H-1$ with
  $\epsP = \fCE_{t+1}(\epsA, \epsB)$ and choosing $\epsilon_t =
  \max\{\epsA, \epsB, \fCE_{t+1}(\epsA, \epsB)\}$.
\end{proof}

All that is left is to prove Theorem~\ref{theorem:ce} which we will do
now,
\theoremCE*
\begin{proof}
  We know $\POPT_{t}$ satisfies,
\begin{align*}
  \POPT_{t} &= Q + A_t^{T}\POPT_{{t+1}}A_t - A_t^{T}\POPT_{t+1}B_t(R + B_t^{T}\POPT_{{t+1}}B_t)^{-1}B_t^{T}\POPT_{t+1}A_t \\
  &= Q + A_t^{T}\POPT_{t+1}(I + B_tR^{-1}B_t^{T}\POPT_{t+1})^{-1}A_t
\end{align*}
where we used the matrix inversion lemma.

Similarly we have,
\begin{align*}
  \PCE_{t} &= Q + \Ahat_t^{T}\PCE_{t+1}(I + \Bhat_t R^{-1}\Bhat_t^{T}\PCE_{t+1})^{{-1}}\Ahat_t
\end{align*}

Consider the difference,
\begin{align*}
  \POPT_{t} - \PCE_{t} &= A_t^{T}\POPT_{t+1}(I + B_tR^{-1}B_t^{T}\POPT_{t+1})^{-1}A_t - \Ahat_t^{T}\PCE_{t+1}(I + \Bhat_t R^{-1}\Bhat_t^{T}\PCE_{t+1})^{{-1}}\Ahat_t \\
                    &= A_t^{T}\POPT_{t+1}(I + B_tR^{-1}B_t^{T}\POPT_{t+1})^{-1}A_t - \Ahat_t^{T}\POPT_{t+1}(I + \Bhat_t R^{-1}\Bhat_t^{T}\POPT_{t+1})^{{-1}}\Ahat_t \\
  &+ \Ahat_t^{T}\left(\POPT_{t+1}(I + \Bhat_t R^{-1}\Bhat_t^{T}\POPT_{t+1})^{{-1}} - \PCE_{t+1}(I + \Bhat_t R^{-1}\Bhat_t^{T}\PCE_{t+1})^{{-1}} \right)\Ahat_t
\end{align*}
To bound the above expression, we will make use of
Lemma~\ref{lemma:mania-appendix-original} with $S = B_tR^{-1}B_t^{T}$,
$\Shat = \Bhat_t R^{-1}\Bhat_t^{T}$,
$Q = \POPT_{t+1}$ and observing that
$\|\Shat - S\| \leq 2\|B_t\|\|R^{-1}\|\epsB + \|R^{-1}\|\epsB^{2}$ we obtain

\begin{align*}
  \|\PCE_{t} - \POPT_{t}\| \leq& \|A_t\|^{2}\|\POPT_{t+1}\|^{2}(2\|B_t\|\|R^{-1}\|\epsB + \|R^{-1}\|\epsB^{2}) + 2\|A_t\|\|\POPT_{t+1}\|\epsA + \|\POPT_{t+1}\|\epsA^{2} \\
  &+ \|\Ahat_t^{T}\left(\POPT_{t+1}(I + \Bhat_t R^{-1}\Bhat_t^{T}\POPT_{t+1})^{{-1}} - \PCE_{t+1}(I + \Bhat_t R^{-1}\Bhat_t^{T}\PCE_{t+1})^{{-1}} \right)\Ahat_t\|
\end{align*}

%\iffalse % START CONDITION NUMBER PROOF
All that remains is to bound the second expression. We will use
Lemma~\ref{lemma:stackexchange} with $N_{1} = \POPT_{t+1}$, $N_{2} = \PCE_{t+1}$ and
$M = \Bhat_t R^{{-1}}\Bhat_t^{T}$ gives us,
\begin{align*}
  &||\POPT_{t+1}(I + \Bhat_t R^{-1}\Bhat_t^{T}\POPT_{t+1})^{{-1}} - \PCE_{t+1}(I + \Bhat_t R^{-1}\Bhat_t^{T}\PCE_{t+1})^{{-1}}|| \\
  &\leq ||(I + \Bhat_t R^{{-1}}\Bhat_t^{T}\POPT_{t+1})^{-1}|| ||\POPT_{t+1} - \PCE_{{t+1}}|| ||(I + \Bhat_t R^{{-1}}\Bhat_t^{T}\PCE_{t+1})^{-1}||
\end{align*}

Thus, we have
\begin{align*}
  ||\POPT_{t} - \PCE_{t}|| &\leq  \|A_t\|^{2}\|\POPT_{t+1}\|^{2}(2\|B_t\|\|R^{-1}\|\epsB + \|R^{-1}\|\epsB^{2}) + 2\|A_t\|\|\POPT_{t+1}\|\epsA + \|\POPT_{t+1}\|\epsA^{2} \\
  &+ \|\Ahat_t\|^{2}||(I + \Bhat_t R^{{-1}}\Bhat_t^{T}\POPT_{t+1})^{-1}|| ||\POPT_{t+1} - \PCE_{{t+1}}|| ||(I + \Bhat_t R^{{-1}}\Bhat_t^{T}\PCE_{t+1})^{-1}||
\end{align*}

Observe that we can bound
\begin{align*}
  ||(I + \Bhat_t R^{{-1}}\Bhat_t^{T}\POPT_{t+1})^{-1}|| &= ||(\POPT_{t+1})^{-1}\POPT_{t+1}(I + \Bhat_t R^{{-1}}\Bhat_t^{T}\POPT_{t+1})^{-1}|| \\
                                                &\leq ||(\POPT_{t+1})^{-1}|| ||\POPT_{t+1}(I + \Bhat_t R^{{-1}}\Bhat_t^{T}\POPT_{t+1})^{-1}|| \\
  &\leq ||(\POPT_{t+1})^{-1}|| ||\POPT_{t+1}|| = \kappa_{\POPT_{t+1}}
\end{align*}
where $\kappa_{\POPT_{t+1}}$ is the condition number of the matrix $\POPT_{t+1}$. This
gives us the bound
\begin{align*}
  ||\POPT_{t} - \PCE_{t}|| &\leq  \|A_t\|^{2}\|\POPT_{t+1}\|^{2}(2\|B_t\|\|R^{-1}\|\epsB + \|R^{-1}\|\epsB^{2}) + 2\|A_t\|\|\POPT_{t+1}\|\epsA + \|\POPT_{t+1}\|\epsA^{2} \\
                        &+
                          \|\Ahat_t\|^{2}\kappa_{\POPT_{t+1}}\kappa_{\PCE_{t+1}}
                          ||\POPT_{t+1} - \PCE_{{t+1}}||
\end{align*}

Using the fact that
$||\Ahat_t||^{2} \leq (\|A_t\| + \epsA)^2$ gives us
\begin{align}
  \label{eq:92}
  ||\POPT_{t} - \PCE_{t}|| &\leq  \|A_t\|^{2}\|\POPT_{t+1}\|^{2}(2\|B_t\|\|R^{-1}\|\epsB + \|R^{-1}\|\epsB^{2}) + 2\|A_t\|\|\POPT_{t+1}\|\epsA + \|\POPT_{t+1}\|\epsA^{2} \nonumber\\
                        &+ (\|A_t\| +  \epsA)^2\kappa_{\POPT_{t+1}}\kappa_{\PCE_{t+1}} ||\POPT_{t+1} - \PCE_{{t+1}}||
\end{align}
%\fi % END CONDITION NUMBER PROOF

If $\epsA, \epsB$ are small enough that $\|\POPT_{t+1} -
\PCE_{t+1}\|\|\POPT_{t+1}\| \leq 1$ then we can bound
\begin{align*}
  \|(\PCE_{t+1})^{-1} - (\POPT_{t+1})^{-1}\| &\leq
  \frac{\|(\POPT_{t+1})^{-1}\|}{1 - \|(\POPT_{t+1})^{-1}\|\|\POPT_{t+1}
                                               - \PCE_{t+1}\|} \\
  &\leq \frac{\|(\POPT_{t+1})^{-1}\|\|\POPT_{t+1}\|}{\|\POPT_{t+1}\| -
    \|(\POPT_{t+1})^{-1}\|} \\
  &\leq \frac{\kappa_{\POPT_{t+1}}}{\|\POPT_{t+1}\|^2 - \kappa_{\POPT_{t+1}}}
\end{align*}
The above result is from~\cite{horn12} (Section 5.8 page 381). Now we
can bound the condition number $\kappa_{\PCE_{t+1}}$ by observing that
$\|(\PCE_{t+1})^{-1}\| \leq \|(\POPT_{t+1})^{-1}\| +
\frac{\kappa_{\POPT_{t+1}}}{\|\POPT_{t+1}\|^2 - \kappa_{\POPT_{t+1}}}$
and $\|\PCE_{t+1}\| \leq \|\POPT_{t+1}\| + \|\PCE_{t+1} -
\POPT_{t+1}\| \leq \|\POPT_{t+1}\| + \frac{1}{\|\POPT_{t+1}\|}$ giving
us
\begin{align*}
  \kappa_{\POPT_{t+1}}\kappa_{\PCE_{t+1}} &= \kappa_{\POPT_{t+1}}\|(\PCE_{t+1})^{-1}\|\|\PCE_{t+1}\| \leq \kappa_{\POPT_{t+1}}(\|(\POPT_{t+1})^{-1}\| +
\frac{\kappa_{\POPT_{t+1}}}{\|\POPT_{t+1}\|^2 -
                        \kappa_{\POPT_{t+1}}})(\|\POPT_{t+1}\| +
                        \frac{1}{\|\POPT_{t+1}\|}) \\
  &= \kappa_{\POPT_{t+1}}^2 +
    \frac{\kappa_{\POPT_{t+1}}^2}{\|\POPT_{t+1}\|^2} + \frac{\kappa_{\POPT_{t+1}}^2}{\|\POPT_{t+1}\|^2 -
                        \kappa_{\POPT_{t+1}}}(\|\POPT_{t+1}\| +
                        \frac{1}{\|\POPT_{t+1}\|})
\end{align*}
Denoting $c_{\POPT_{t+1}}$ as the right hand side expression in the
above inequality we get the desired result.The example that
realizes the upper bound is given in Appendix~\ref{sec:scalar-example-that}.

\end{proof}

\subsection{Note on Assumption~\ref{assumption:psd}}
\label{sec:assumpt-refass}

Consider the cost-to-go matrix $\PILC_t$ given by
\begin{align*}
  \PILC_{t} &= Q + \Ahat_t^{T}\PILC_{{t+1}}A_t - \Ahat_t^{T}\PILC_{t+1}B_t(R + \Bhat_t^{T}\PILC_{{t+1}}B_t)^{-1}\Bhat_t^{T}\PILC_{t+1}A_t \\
  &= Q + \Ahat_t^{T}\PILC_{t+1}(I + B_tR^{-1}\Bhat_t^{T}\PILC_{t+1})^{-1}A_t
\end{align*}
and the cost-to-go from any state $x$ is given by
\begin{align*}
  V_t(x) = x^T\PILC_tx
\end{align*}
Since this is a quadratic, for it to be convex (and thus, have a
minima) we require the leading
coefficient to be positive semi-definite. In other words, $\PILC_t$
should have eigenvalues with non-negative real parts. Assuming
$\PILC_{t+1}$ to be positive semi-definite, and observing the fact
that $Q$ is a positive semi-definite matrix, we require that
$B_tR^{-1}\Bhat_t^T$ to have eigenvalues with non-negative real parts
for $\PILC_t$ to be positive semi-definite. Note that this is
trivially satisfied for \MM{} as the leading coefficient there
contains a similar term $B_tR^{-1}B_t^T$ which is positive
semi-definite.

Intuitively, if $B_tR^{-1}\Bhat_t^T$ does not have eigenvalues with
non-negative real parts, then the resulting quadratic cost-to-go
function need not be convex, and \ILC{} will not converge.

\subsection{Iterative Learning Control Results}
\label{sec:iter-learn-contr}

Our first lemma derives a similar result as
Lemma~\ref{lemma:ce-quadratic} but for the iterative learning control
setting,
\begin{lemma}
  \label{lemma:ilc-quadratic}
  Given functions $f_{1}(x, u)$ and $f_{2}(x, u)$ such that
  $\nabla_{u} f_{1}(x, u) = (B_{1}^{T}P_{1}B_{1} + R)u + B_{1}^{T}P_{1}A_{1}x$
  and
  $\nabla_{u} f_{2}(x, u) = (B_{2}^{T}P_{2}B_{1} + R)u + B_{2}^{T}P_{2}A_{1}x$
  where $R, P_{1}, P_{2}$ are positive-definite matrices. Let $K_{1}$ and
  $K_{2}$ be unique matrices such that $\nabla_{u} f_{1}(x, K_{1}x) = 0$ and
  $\nabla_{u} f_{2}(x, K_{2}x) = 0$ for any
  vector $x$. Also,
  denote $\Gamma = 1 + \max\{||A_1||, ||B_1||, ||P_1||,
  ||K_1||\}$. Suppose there exists $\epsA, \epsB, \epsP > 0$ (and $<\Gamma$) such that
  $||A_1 - A_2|| \leq \epsA$, and $||B_1 - B_2|| \leq
  \epsB$, and $||P_1 - P_2|| \leq \epsP$. Then we have,
  \begin{equation}
    \label{eq:81}
    \|K_{1} - K_{2}\| \leq \frac{2\Gamma^3(\epsB + 2\epsP)}{\ubar{\sigma}(R)}
  \end{equation}
\end{lemma}
\begin{proof}
  Let us bound the difference $\|\nabla_{u} f_{1}(x,u) - \nabla_{u} f_{2}(x, u)\|$
  by bounding each term separately. First consider the term
  \begin{align*}
    \|B_{1}^{T}P_{1}B_{1} - B_{2}^{T}P_{2}B_{1}\| &= \|(B_{1} - B_{2})^{T}P_{1}B_{1} + B_{2}^{T}(P_{1} - P_{2})B_{1}\| \\
                                                  &\leq \Gamma^{2}\epsB + \Gamma(\Gamma + \epsB)\epsP \\
    &\leq \Gamma^2(\epsB + 2\epsP)
  \end{align*}
  where we used the fact that $\|B_{2}\| \leq \Gamma + \epsB$.
  We can similarly bound the term
  \begin{align*}
    \|B_{1}^{T}P_{1}A_{1} - B_{2}^{T}P_{2}A_{1}\| \leq \Gamma^{2}(\epsB + 2\epsP)
  \end{align*}

  Thus, we have for any vector $x$ such that $\|x\| \leq 1$
  \begin{align*}
    \|\nabla_{u} f_{1}(x,u) - \nabla_{u} f_{2}(x, u)\| \leq \Gamma^{2}(\epsB + 2\epsP)(\|u\| + 1)
  \end{align*}
  Substituting $u=u_{1}$ we get
  \begin{align*}
    \|\nabla_{u} f_{2}(x, u_{1})\| \leq \Gamma^{2}(\epsB + 2\epsP)(\|u_{1}\| + 1)
  \end{align*}

  We can bound $\|u_{1}\| \leq \|K_{1}\|\|x\| \leq \|K_{1}\| \leq \Gamma$. Then
  from Lemma~\ref{lemma:1} we have,
  \begin{align*}
    \|u_{1} - u_{2}\| &\leq \frac{\Gamma^{2}(\epsB + 2\epsP)(\Gamma+1)}{\ubar{\sigma}(R)} \\
    \|K_{1} - K_{2}\| &\leq \frac{2\Gamma^{3}(\epsB + 2\epsP)}{\ubar{\sigma}(R)}
  \end{align*}
\end{proof}

Now we will prove Lemma~\ref{lemma:ilc},
\ilcLemma*
\begin{proof}
  Use Assumption~\ref{assumption:singularvalue} and
  Lemma~\ref{lemma:ilc-quadratic} for $t = 0, \cdots, H-1$ with
  $\epsP = \fILC_{t+1}(\epsA, \epsB)$ and choosing $\epsilon_t =
  \max\{\epsA, \epsB, \fILC_{t+1}(\epsA, \epsB)\}$.
\end{proof}

Our final task is to prove Theorem~\ref{theorem:ilc},
\ilcTheorem*
\begin{proof}
  We know $\PILC_{t}$ satisfies,
\begin{align*}
  \PILC_{t} &= Q + \Ahat_t^{T}\PILC_{{t+1}}A_t - \Ahat_t^{T}\PILC_{t+1}B_t(R + \Bhat_t^{T}\PILC_{{t+1}}B_t)^{-1}\Bhat_t^{T}\PILC_{t+1}A_t \\
  &= Q + \Ahat_t^{T}\PILC_{t+1}(I + B_tR^{-1}\Bhat_t^{T}\PILC_{t+1})^{-1}A_t
\end{align*}
where we used the matrix inversion lemma.

Consider the difference,
\begin{align*}
  \POPT_{t} - \PILC_{t} &= A_t^{T}\POPT_{t+1}(I + B_tR^{-1}B_t^{T}\POPT_{t+1})^{-1}A_t - \Ahat_t^{T}\PILC_{t+1}(I + B_tR^{-1}\Bhat_t^{T}\PILC_{t+1})^{-1}A_t \\
                      &= A^{T}\POPT_{t+1}(I + B_tR^{-1}B_t^{T}\POPT_{t+1})^{-1}A_t - \Ahat_t^{T}\POPT_{t+1}(I + B_tR^{-1}\Bhat_t^{T}\POPT_{t+1})^{-1}A_t \\
  &+ \Ahat_t^{T}\left(\POPT_{t+1}(I + B_tR^{-1}\Bhat_t^{T}\POPT_{t+1})^{-1} - \PILC_{t+1}(I + B_tR^{-1}\Bhat_t^{T}\PILC_{t+1})^{-1}\right)A_t
\end{align*}

Here again we can use Lemma~\ref{lemma:mania-appendix-original} with
$S = B_tR^{-1}B_t^{T}$, $\Shat = B_tR^{-1}\Bhat_t^{T}$, $Q = \POPT_{t+1}$ and observing that
$||\Shat - S|| \leq ||B_t||||R^{-1}||\epsB$ to get
\begin{align*}
  ||\PILC_{t} - \POPT_{t}|| &\leq ||A_t||^{2}||\POPT_{t+1}||^{2}||B_t||||R^{-1}||\epsB + ||A_t||||\POPT_{t+1}||\epsA \\
  &+ ||A_t||||\Ahat_t||||\POPT_{t+1}(I + B_tR^{-1}\Bhat_t^{T}\POPT_{t+1})^{-1} - \PILC_{t+1}(I + B_tR^{-1}\Bhat_t^{T}\PILC_{t+1})^{-1}||
\end{align*}

Here again we use Lemma~\ref{lemma:stackexchange} to bound the second expression
giving us
\begin{align*}
  ||\PILC_{t} - \POPT_{t}|| &\leq ||A_t||^{2}||\POPT_{t+1}||^{2}||B_t||||R^{-1}||\epsB + ||A_t||||\POPT_{t+1}||\epsA \\
  &+ ||A_t||||\Ahat_t|| ||(I + B_tR^{-1}\Bhat_t^{T}\POPT_{t+1})^{-1}|| ||\PILC_{t+1} - \POPT_{t+1}|| ||(I + B_tR^{-1}\Bhat_t^{T}\PILC_{t+1})^{-1}||
\end{align*}

This can be rewritten as the final bound,
\begin{align}
  \label{eq:91}
  ||\POPT_{t} - \PILC_{t}|| &\leq ||A_t||^{2}||\POPT_{t+1}||^{2}||B_t||||R^{-1}||\epsB + ||A_t||||\POPT_{t+1}||\epsA \nonumber\\
  &+ (||A_t||^{2} + \epsA||A_t||) \kappa_{\POPT_{t+1}}\kappa_{\PILC_{t+1}} ||\POPT_{t+1} - \PILC_{t+1}||
\end{align}

The constant $c_{\POPT_{t+1}}$ can be derived very similarly as we
have done in the proof of Theorem~\ref{theorem:ce}. The example that
realizes the upper bound is given in Appendix~\ref{sec:scalar-example-that}.
\end{proof}

\subsection{Scalar Example that Realizes Upper Bounds}
\label{sec:scalar-example-that}

\subsubsection{General Formulation}
\label{sec:general-formulation}
Consider a $1$D linear dynamical system given by,
\begin{equation}
  \label{eq:10}
  x_t = ax_t + bu_t
\end{equation}
where $x_t, u_t, a, b \in \reals$. The cost function is given by,
\begin{equation}
  \label{eq:11}
  V_0(x_0) = \sum_{t=0}^{H-1} qx_t^2 + ru_t^2 + qx_H^2
\end{equation}
We are given access to an approximate model specified using $\ahat,
\bhat \in \reals$.

The optimal cost-to-go is specified using
\begin{align}
  \label{eq:12}
  &\popt_H = q \\
  &\popt_t = q + \frac{a^2\popt_{t+1}}{1 + b^2r^{-1}\popt_{t+1}} = q + \frac{a^2r\popt_{t+1}}{r + b^2\popt_{t+1}}
\end{align}

For \MM{}, the cost-to-go is specified using
\begin{align}
  \label{eq:13}
  &\pce_H = q \\
  &\pce_t = q + \frac{\ahat^2r\pce_{t+1}}{r + \bhat^2\pce_{t+1}}
\end{align}

For ILC, the cost-to-go is specified using
\begin{align}
  \label{eq:8}
  &\pilc_h = q \\
  &\pilc_t = q + \frac{a\ahat r\pilc_{t+1}}{r + b\bhat\pilc_{t+1}}
\end{align}

In the next two subsections, we will show that an example dynamical
system where $\bhat = 0$, i.e. the approximate model thinks that the
system is not controllable will realize the worst case upper bounds
for both \MM{} and ILC as presented in Theorems~\ref{theorem:ce}
and~\ref{theorem:ilc} respectively.

\subsubsection{Optimal Control with Misspecified Model}
\label{sec:cert-equiv-contr-1}

Consider the difference
\begin{align*}
  \popt_t - \pce_t &= \frac{a^2r\popt_{t+1}}{r + b^2\popt_{t+1}} -
                       \frac{\ahat^2r\pce_{t+1}}{r +
                       \bhat^2\pce_{t+1}} \\
  &= \left( \frac{a^2r\popt_{t+1}}{r + b^2\popt_{t+1}} -
    \frac{\ahat^2r\popt_{t+1}}{r + \bhat^2\popt_{t+1}} \right) +
    \left( \frac{\ahat^2r\popt_{t+1}}{r + \bhat^2\popt_{t+1}} - \frac{\ahat^2r\pce_{t+1}}{r +
                       \bhat^2\pce_{t+1}} \right)
\end{align*}
Let us look at each term separately. The first term can be simplified
as
\begin{equation}
  \label{eq:17}
  \left( \frac{a^2r\popt_{t+1}}{r + b^2\popt_{t+1}} -
    \frac{\ahat^2r\popt_{t+1}}{r + \bhat^2\popt_{t+1}} \right) =
                                                                 \frac{\popt_{t+1}(a^2-
                                                                 \ahat^2)}{1
                                                                 +
                                                                 \bhat^2r^{-1}\popt_{t+1}}
                                                                 +
                                                                 \frac{a^2r^{-1}(\bhat^2
                                                                 -
                                                                 b^2)(\popt_{t+1})^2}{(1
                                                                 +
                                                                 b^2r^{-1}\popt_{t+1})(1
                                                                 +
                                                                 \bhat^2r^{-1}\popt_{t+1})}
\end{equation}
Similarly, the second term can be simplified as
\begin{equation}
  \label{eq:9}
  \left( \frac{\ahat^2r\popt_{t+1}}{r + \bhat^2\popt_{t+1}} - \frac{\ahat^2r\pce_{t+1}}{r +
                       \bhat^2\pce_{t+1}} \right) =
                   \frac{\ahat^2(\popt_{t+1} - \pce_{t+1})}{(1 +
                     \bhat^2r^{-1}\popt_{t+1})(1 + \bhat^2r^{-1}\pce_{t+1})}
\end{equation}

Now, consider the example dynamical system where $a - \ahat = \epsa$,
$b - \bhat = \epsb$, and $\bhat = 0$. Our upper bound in
Theorem~\ref{theorem:ce} states that,
\begin{equation}
  \label{eq:16}
  |\popt_{t} - \pce_t| \leq a^2r^{-1}(\popt_{t+1})^2(2b\epsb +
  \epsb^2) + \popt_{t+1}(2a\epsa + \epsa^2) + (a +
  \epsa)^2|\popt_{t+1} - \pce_{t+1}|
\end{equation}
For the example system equation~\eqref{eq:17} simplifies to,
\begin{align*}
  \frac{\popt_{t+1}(a^2-
                                                                 \ahat^2)}{1
                                                                 +
                                                                 \bhat^2r^{-1}\popt_{t+1}}
                                                                 +
                                                                 \frac{a^2r^{-1}(\bhat^2
                                                                 -
                                                                 b^2)(\popt_{t+1})^2}{(1
                                                                 +
                                                                 b^2r^{-1}\popt_{t+1})(1
                                                                 +
                                                                 \bhat^2r^{-1}\popt_{t+1})}
  = \popt_{t+1}(2a\epsa + \epsa^2)
                                                                 +
                                                                 \frac{a^2r^{-1}(\popt_{t+1})^2(2b\epsb
  + \epsb^2)}{(1
                                                                 +
                                                                 b^2r^{-1}\popt_{t+1})}
\end{align*}
which matches the first two terms in the upper bound
(equation~\eqref{eq:16}) upto a constant. Now, let's look at how
equation~\eqref{eq:9} simplifies
\begin{align*}
  \frac{\ahat^2(\popt_{t+1} - \pce_{t+1})}{(1 +
                     \bhat^2r^{-1}\popt_{t+1})(1 +
  \bhat^2r^{-1}\pce_{t+1})} = (a + \epsa)^2(\popt_{t+1} - \pce_{t+1})
\end{align*}
which matches the last term in the upper bound
(equation~\eqref{eq:16}) exactly. Thus, we found an example where
$|\popt_t - \pce_t|$ matches the upper bound specified in
Theorem~\ref{theorem:ce} upto a constant.

\subsubsection{Iterative Learning Control}
\label{sec:iter-learn-contr-2}

Consider the difference
\begin{align*}
  \popt_t - \pilc_t &= \frac{a^2r\popt_{t+1}}{r + b^2\popt_{t+1}} -
                       \frac{a\ahat r\pilc_{t+1}}{r +
                      b\bhat\pilc_{t+1}} \\
  &= \left( \frac{a^2r\popt_{t+1}}{r + b^2\popt_{t+1}} -
    \frac{a\ahat r\popt_{t+1}}{r + b\bhat\popt_{t+1}} \right) + \left(
    \frac{a\ahat r\popt_{t+1}}{r + b\bhat\popt_{t+1}} - \frac{a\ahat r\pilc_{t+1}}{r +
                      b\bhat\pilc_{t+1}}\right)
\end{align*}
Once again let us look at each term separately. The first term can be
simplified as
\begin{equation}
  \label{eq:20}
  \left( \frac{a^2r\popt_{t+1}}{r + b^2\popt_{t+1}} -
    \frac{a\ahat r\popt_{t+1}}{r + b\bhat\popt_{t+1}} \right) =
  \frac{a\popt_{t+1}(a - \ahat)}{(1 + b\bhat r^{-1}\popt_{t+1})} +
  \frac{a^2br^{-1}(\popt_{t+1})^2(\bhat - b)}{(1 + b^2r^{-1}\popt_{t+1})(1
    + b\bhat r^{-1}\popt_{t+1})}
\end{equation}
Similarly, the second term can be simplified as
\begin{equation}
  \label{eq:18}
  \left(
    \frac{a\ahat r\popt_{t+1}}{r + b\bhat\popt_{t+1}} - \frac{a\ahat r\pilc_{t+1}}{r +
                      b\bhat\pilc_{t+1}}\right) =
                  \frac{a\ahat(\popt_{t+1} - \pilc_{t+1})}{(1 + b\bhat
                    r^{-1}\popt_{t+1})(1 + b\bhat r^{-1}\pilc_{t+1})}
\end{equation}
Similar to \MM{} in the previous section, consider the example dynamical
system where $a - \ahat = \epsa$, $b - \bhat = \epsb$ and $\bhat =
0$. Our upper bound in Theorem~\ref{theorem:ilc} states that
\begin{equation}
  \label{eq:19}
  |\popt_t - \pilc_t| \leq a^2(\popt_{t+1})^2br^{-1}\epsb +
  a\popt_{t+1}\epsa + a(a + \epsa)|\popt_{t+1} - \pilc_{t+1}|
\end{equation}
For the example dynamical system, equation~\eqref{eq:20} simplifies to
\begin{align*}
  \frac{a\popt_{t+1}(a - \ahat)}{(1 + b\bhat r^{-1}\popt_{t+1})} +
  \frac{a^2br^{-1}(\popt_{t+1})^2(\bhat - b)}{(1 + b^2r^{-1}\popt_{t+1})(1
    + b\bhat r^{-1}\popt_{t+1})} = a\popt_{t+1}\epsa +
  \frac{a^2(\popt_{t+1})^2br^{-1}\epsb}{(1 + b^2r^{-1}\popt_{t+1})}
\end{align*}
which matches the first two terms in the upper bound
(equation~\eqref{eq:19}) upto a constant. Now, let's look at how
equation~\eqref{eq:18} simplifies
\begin{align*}
  \frac{a\ahat(\popt_{t+1} - \pilc_{t+1})}{(1 + b\bhat
                    r^{-1}\popt_{t+1})(1 + b\bhat r^{-1}\pilc_{t+1})}
  = a(a + \epsa)(\popt_{t+1} - \pilc_{t+1})
\end{align*}
which matches the last term in the upper bound
(equation~\eqref{eq:19}) exactly. Thus, we found that the same example
also matches the upper bound specified in Theorem~\ref{theorem:ilc}
upto a constant.

\subsection{Experiment Details}
\label{sec:experiment-details}

\subsubsection{Linear Dynamical System with Approximate Model}
\label{sec:line-dynam-syst-1}

We use a horizon $H = 10$ and initial state $x_0 = \begin{bmatrix}0.1
  \\ 0.1 \end{bmatrix}$.

\subsubsection{Nonlinear Inverted Pendulum with Misspecified Mass}
\label{sec:nonl-invert-pend}

For the second experiment, we use the nonlinear dynamical system of an inverted
pendulum. The state space is specified by $x =
\begin{bmatrix}
  \theta \\
  \dot{\theta}
\end{bmatrix} \in \reals^{2}
$ where $\theta$ is the angle between the pendulum and the vertical axis. The
control input is $u = \tau \in \reals$ specifying the torque $\tau$ to be
applied at the base of the pendulum. The dynamics of the system are
given by the ODE,
%\begin{align*}
$\ddot{\theta} = \frac{\bar{\tau}}{m\ell^{2}} - \frac{g\sin(\theta)}{\ell}$
%\end{align*}
where $m$ is the mass of the pendulum, $\ell$ is the length of the pendulum, $g$
is the acceleration due to gravity, and
$\bar{\tau} = \max(\tau_{\min}, \min(\tau_{\max}, \tau))$ is the clipped torque
based on torque limits. We use $\ell = 1$m, $\tau_{\max} = 8$Nm,
$\tau_{\min} = -8$Nm, and $m = 1$kg.

We use a per time step cost function defined as $c(\theta, \tau) =
0.1\tau^{2} + \theta^{2}$
where $\theta \in [-\pi, \pi]$, an initial state $x_{0} =
\begin{bmatrix}
  \frac{\pi}{2} \\ 0.5
\end{bmatrix}
$, and a horizon $H = 20$. For all algorithms, we start with an
initial control sequence consisting of zero torques for the entire horizon.

\subsubsection{Nonlinear Planar Quadrotor Control in Wind}
\label{sec:nonl-plan-quadr-1}

In our final experiment, we compare \MM{} and \ILC{} on a planar quadrotor control
task in the presence of wind. The quadrotor is controlled using two propellers
that provide upward
thrusts $(u_{1}, u_{2})$ and allows movement in the $3$D planar space
described as
$(p_{x}, p_{y}, \theta)$ where $p_{x}, p_{y}$ are X, Y positions, and
$\theta$ is the yaw of the quadrotor. The dynamics of the planar quadrotor
is specified
using a state vector $x \in \reals^{6}$, control input $u \in \reals^{2}$ as
\begin{align*}
  x =
  \begin{bmatrix}
    p_{x} \\ p_{y} \\ \theta \\ \dot{p}_{x} \\ \dot{p}_{y} \\ \dot{\theta}
  \end{bmatrix}, u =
  \begin{bmatrix}
    u_{1} \\ u_{2}
  \end{bmatrix},
  \dot{x} =
  \begin{bmatrix}
    \dot{p}_{x} \\ \dot{p}_{y} \\ \dot{\theta} \\ \frac{1}{m}(u_{1} + u_{2})\sin(\theta) \\ \frac{1}{m}(u_{1} + u_{2})\cos(\theta) - g \\ \frac{\ell}{2J}(u_{2} - u_{1})
  \end{bmatrix}
\end{align*}
where $m$ is the mass of the quadrotor, $\ell$ is the distance between the
propellers, $g$ is acceleration due to gravity, and $J$ is the moment of
inertia of the quadrotor. We use $m = 1$kg, $\ell = 0.3$m, and
$J = 0.2m\ell^{2}$. The objective of the task is to move the quadrotor from an initial
state $x_0$
at $(-3, 1)$ with zero velocity to a final state $x_f$ at $(3, 1)$ with zero
velocity. This is achieved using the per time-step cost function
$c(x, u) = (x - x_{f})^{T}Q(x - x_{f}) + (u - u_{h})^{T}R(u - u_{h})$
where $u_{h} = [\frac{1}{2}mg, \frac{1}{2}mg]$ are the hover controls. We use a
horizon of $H = 60$ with a step size of $0.025$ for RK4 integration.

%%% Local Variables:
%%% mode: latex
%%% TeX-master: "../main"
%%% End:


\chapter*{Bibliography}
\addcontentsline{toc}{chapter}{Bibliography}

\vspace{-25mm}
This bibliography contains \total{citenum} references.
\vspace{10mm}

\printbibliography[heading=none]

\end{document}

%%% Local Variables:
%%% mode: latex
%%% TeX-master: t
%%% End:
